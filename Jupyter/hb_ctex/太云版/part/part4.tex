% !Mode:: "TeX:UTF-8"
\section{复旋量场: 1/2-自旋有质量有荷, 狄拉克费米子}
\subsection{狄拉克方程及其共轭方程, 泡利-狄拉克表象, 外尔方程的拉氏密度}
前文群论一章中, 我们已严格导出旋量满足的方程~$(\gamma^\mu p_\mu-m)\phi=0$. 作代换~$p_\mu=i\partial_\mu$, 便可得到量子方程
\begin{align}
(i\gamma^\mu\partial_\mu-m)\psi=0;
\end{align}
称为狄拉克方程. 能导致狄拉克方程的拉氏密度为
\begin{align}
\mathcal{L}=\bar{\psi}(i\gamma^\mu\partial_\mu-m)\psi;
\end{align}
%恰如前文分析, 此拉氏密度在值为零处, 取得运动方程; 或曰
可见真实运动使场的拉氏密度为零. 将其对~$\bar{\psi}$ 应用拉格朗日方程即可得前述狄拉克方程; 将其对~$\psi$ 应用拉氏方程, 可得
\begin{align}
i\partial_\mu\bar{\psi}\gamma^\mu+m\bar{\psi}=0;
\end{align}
此即狄拉克方程的共轭方程. 当然, 上式也可由狄拉克方程作演换获得. 我们曾经说过, 任何场都是要满足~Klein-Gordon 方程的. 对狄拉克方程我们验证如下~$-(i\gamma^\nu\partial_\nu+m)(i\gamma^\mu\partial_\mu-m)\psi=(\gamma^\mu\gamma^\nu\partial_\mu\partial_\nu+m^2)\psi=(\partial^2+m^2)\psi=0$.

我们已经证明了狄拉克方程对离散变换是对称的; 下面, 我们来证明狄拉克方程在连续洛伦兹变换下的不变性. 事实上, 原则地, 一个量只要由两个洛伦兹矢量构成, 则就是对洛伦兹变换不变的标量. 不过现在我们还是具体做一遍. 若体系或坐标系发生一个洛伦兹转动, 则有
\begin{align}
(i\gamma^\mu\partial_\mu-m)\bar{\psi}=0&\rightarrow\left(i\gamma^\mu{\Lambda_\mu}^\nu\partial_\nu-m\right)\Lambda_{\frac{1}{2}}\psi(\Lambda^{-1}x)\nonumber\\
&=\left[i\gamma^\mu{(\Lambda^{-1})^\nu}_\mu\partial_\nu-m\right]\Lambda_{\frac{1}{2}}\psi(\Lambda^{-1}x)\nonumber\\
&=\left[i\Lambda_{\frac{1}{2}}\gamma^\nu\Lambda^{-1}_{\frac{1}{2}}\partial_\nu-m\right]\Lambda_{\frac{1}{2}}\psi(\Lambda^{-1}x)\nonumber\\
&=\Lambda_{\frac{1}{2}}(i\gamma^\nu\partial_\nu-m)\psi(\Lambda^{-1}x)=0;
\end{align}
由此可见, 在连续洛伦兹变换下, 狄拉克方程的确是不变的.

此外, 在群论一章中推导时, $\psi$ 的上下分量是具有确定手征性左右旋量; 相应得到的~$\gamma^\mu$ 或方和的矩阵形式, 称是外尔表象或手征表象下的. 数学上我们知道, 每一组满足~$\gamma^\mu$ 的代数的矩阵, 都是狄拉克方程的一个表示. 若取~$\gamma^0=\left[\begin{array}{cc}1&0\\0&-1\end{array}\right]$, 而取~$\gamma^i$ 的形式与外尔表象中的相同, 则不难发现, 这样的一组矩阵, 亦是满足~$\gamma^\mu$ 的代数的. 产生此表示的表象, 我们称为泡利-狄拉克表象. 此时, 狄拉克方程的矩阵形式如下
\begin{align}
\left[\begin{array}{cc}
E-m&-\bm{\sigma}\cdot\bm{p}\\
\bm{\sigma}\cdot\bm{p}&-E-m
\end{array}\right]
\left[\begin{array}{c}\varphi\\\chi\end{array}\right]=0.
\end{align}
我们已知, 当粒子质量为零时, 狄拉克方程退化为两个外尔方程, 狄拉克旋量退化为两个分别描述左右旋态的外尔旋量. 也就是说, 当粒子的动能远高于静质量时, 采用外尔表象是方便的. 同样可看出, 当粒子动能较小, 例如远小于静质量时, 采用狄拉克表象是方便的.

容易用左右手外尔旋量的形式表达出狄拉克场拉氏密度:
\begin{align}
\mathcal{L}=\psi^\dag_L i\bar{\sigma}^\mu\partial_\mu\psi_L+\psi^\dag_R i\sigma^\mu\partial_\mu\psi_R-m(\psi_R^\dag\psi_L+\psi^\dag_L\psi_R).
\end{align}
显然, 质量造成了左右外尔旋量的耦合. 在质量为零时, 狄拉克场拉氏密度退化为外尔场的拉氏量, 后者即给出分别描述左右手旋量运动的两个方程.

我们可以将狄拉克方程写为~$i\gamma^0\partial_0\psi=(-i\gamma^i\partial_i+m)\psi$, 从而~$i\partial_0\psi=(-i\gamma^0\gamma^i\partial_i+\gamma^0m)\psi$; 若命~$\gamma^0=\beta,~\alpha^i=\gamma^0\gamma^i$, 则就得~$i\partial_0\psi=(-i\alpha^i\partial_i+\beta m)\psi$. 此即狄拉克最早写下的形式.







最后, 我们来谈一下马约拉纳费米子. %狄拉克旋量是复的, 狄拉克表示是洛伦兹群的一个复表示; 这源自于我们为~$\gamma$ 算符所满足的~Clifford 代数所选定的相应矩阵.
若我们为~Clifford 代数选取下述表示矩阵: $\gamma^0=
\left[\begin{array}{cc}
0&i\sigma^2\\
i\sigma^2&0
\end{array}\right],~
\gamma^1=
\left[\begin{array}{cc}
i\sigma^3&0\\
0&i\sigma^3
\end{array}\right],~
\gamma^2=
\left[\begin{array}{cc}
0&-\sigma^2\\
\sigma^2&0
\end{array}\right],~
\gamma^3=
\left[\begin{array}{cc}
-i\sigma^1&0\\
0&-i\sigma^1
\end{array}\right],$
称为马约拉纳基, 则可发现我们得到了洛伦兹群的一个实旋量表示; 相应的表示基函数就称为马维拉纳旋量. 显然, 马维拉纳旋量的荷共轭, 即~$C$ 变换结就是其自身~$\psi=\left[\begin{array}{c}\psi_L\\i\sigma^2\psi^*_L\end{array}\right]$. 我们反粒子是其自身的粒子, 为马约拉纳费米子.

总之, 有质量复旋量场描述的粒子, 1/2-自旋有质量有荷, 如电子, 夸克等, 称为狄拉克费米子; 无质量复旋量场描述的粒子, 1/2-自旋无质量有荷, 称为外尔费米子; 有质量实旋量场描述的粒子, 1/2-自旋有质量无荷, 称为马约拉纳费米子.









\subsection{狄拉克方程的解}



狄拉克方程的解是
\begin{align}
\psi(x)=\int\frac{d^3\bm{p}}{(2\pi)^3\sqrt{E_{\bm{p}}}}\sum_s\left(a^s_{\bm{p}}u^s(p)e^{-ipx}+b^{s\dag}_{\bm{p}}v^s(p)e^{ipx}\right).
\end{align}

















