% !Mode:: "TeX:UTF-8"

\section{分析力学: 场的力学量的获得}
\subsection{欧拉-拉格朗日方程}
我们设场~$\phi$ 的拉氏密度仅与场及其一次导数项有关~$\mathcal{L}=\mathcal{L}(\phi,\partial_\mu\phi)$. 总拉氏量与拉氏密度的关系为~$L=\int d^3\bm{x}\mathcal{L}(\phi,\partial_\mu\phi)$; 场的作用量为~$S=\int dtL=\int d^4x\mathcal{L}(\phi,\partial_\mu\phi)$. 根据最小作用量原理:
\begin{align}
0=&\delta S\nonumber\\
=&\int d^4x\left[\frac{\partial\mathcal{L}}{\partial\phi}\delta\phi+\frac{\partial\mathcal{L}}{\partial\partial_\mu\phi}\delta(\partial_\mu\phi)\right]\nonumber\\
=&\int d^4x\left[\frac{\partial\mathcal{L}}{\partial\phi}\delta\phi-\partial_\mu\frac{\partial\mathcal{L}}{\partial\partial_\mu\phi}\cdot\delta\phi+
\partial_\mu\left(\frac{\partial\mathcal{L}}{\partial\partial_\mu\phi}\delta\phi\right)\right],
\end{align}
可得场的运动方程即\footnote{作为对比, 三维情况的~E-L 方程为
\begin{align}
\frac{d}{dt}\left(\frac{\partial L}{\partial\dot{q}}\right)-\frac{\partial L}{\partial q}=0.
\end{align}}
\begin{align}
\frac{\partial\mathcal{L}}{\partial\phi}-\partial_\mu\frac{\partial\mathcal{L}}{\partial\partial_\mu\phi}=0;
\end{align}
称为欧拉-拉格朗日方程. 同时我们知道
\begin{gather}
\pi=\frac{\partial\mathcal{L}}{\partial\dot{\phi}},~\mathcal{H}=\pi\dot{\phi}-\mathcal{L},~H=\int d^3\bm{x}\mathcal{H};\\
\dot{\pi}=-\frac{\partial\mathcal{H}}{\partial\phi},~\dot{\phi}=\frac{\partial\mathcal{H}}{\partial\pi}.
\end{gather}
显然~$\mathcal{L}$ 以~$\phi,~\partial_\mu\phi$ 为独立变量; $\mathcal{H}$ 以~$\phi,~\pi$ 为独立变量.


\subsection{诺特定理: 对称性与守恒律}


参照系变换, 即数学上的坐标变换\footnote{这一点是很重要的; 尤其是对于广义相对论: 正是据此, 我们得出引力即时空弯曲.}; 一般地~(即对任意度规成立地), 此即~$x_\nu\rightarrow x'_\mu=x'_\mu(x_\nu)$. 狭义相对论说的即是, 在狭义相对论这一框架下, 能允许的最一般的坐标变换, 应退化到/满足式~$x'_\mu={\Lambda_\mu}^\nu x_\nu+\xi_\mu$. 此处, 我们将考察在时空平移或旋转的操作下, 最小作用量原理能给出什么.

显然, 此种情况下, 作用量变分变为
\begin{align}
\delta S=\int\delta(d^4x)\mathcal{L}+\int d^4x\delta_{xx'}(\mathcal{L});
\end{align}
显然, 我们要分别求出~$\delta(d^4x)$ 与~$\delta_{xx'}(\mathcal{L})$. 首先, 我们来求~$\delta(d^4x)$. 为之, 我们将时空平移或旋转操作的无穷小表为~$x'_\mu=x_\mu+\delta x_\mu$. 作为例子, 如对于旋转, 前式中即有~$\delta x_\mu={\omega_\mu}^\nu x_\nu$. 然后, $dx'^\mu=d(x^\mu+\delta x^\mu)=(\delta^{\mu\nu}+\partial_\nu\delta x^\mu)dx^\nu=dx^\mu+\partial_\nu\delta x^\mu dx^\nu$. 我们知道~$d^4x'=J(x'/x)d^4x$, 由前述推导知~$J=\det(\frac{\partial x'^\mu}{\partial x^\nu})=1+\partial_\mu \delta x^\mu$, 故知
\begin{align}
\delta(d^4x)=(\partial_\mu\delta x^\mu)d^4x.
\end{align}

\noindent 下面, 我们计算作用量变分中的第二项:~$\delta_{xx'}\mathcal{L}(x)=\mathcal{L}'(x')-\mathcal{L}(x)=\mathcal{L}'(x')-\mathcal{L}(x')+\mathcal{L}(x')-\mathcal{L}(x)=\delta\mathcal{L}(x')+\partial_\mu\mathcal{L}(x)\delta x^\mu=\delta\mathcal{L}(x)+\partial_\mu\mathcal{L}(x)\delta x^\mu$. 同理可得~$\delta_{xx'}\phi(x)=\delta\phi(x)+\partial_\mu\phi(x)\delta x^\mu$. 于是
\begin{align}
\delta S=&\int\delta(d^4x)\mathcal{L}+\int d^4x\delta_{xx'}(\mathcal{L})\nonumber\\
=&\int d^4x\partial_\mu(\delta x^\mu)\mathcal{L}+\int d^4x\delta\mathcal{L}+\int d^4x\partial_\mu\mathcal{L}(x)\cdot\delta x^\mu\nonumber\\
=&\int d^4x\left[\frac{\partial\mathcal{L}}{\partial\phi}\delta\phi-\partial_\mu\frac{\partial\mathcal{L}}{\partial\partial_\mu\phi}\cdot\delta\phi+
\partial_\mu\left(\frac{\partial\mathcal{L}}{\partial\partial_\mu\phi}\delta\phi\right)+\partial_\mu(\delta x^\mu)\mathcal{L}+\partial_\mu\mathcal{L}\cdot\delta x^\mu\right]\nonumber\\
=&\int d^4x\left[\frac{\partial\mathcal{L}}{\partial\phi}\delta\phi-\partial_\mu\frac{\partial\mathcal{L}}{\partial\partial_\mu\phi}\cdot\delta\phi+
\partial_\mu\left(\frac{\partial\mathcal{L}}{\partial\partial_\mu\phi}\delta\phi+\mathcal{L}(x)\delta x^\mu\right)\right];
\end{align}
最小作用量原理要求上述结果为零; 于是上述结果中前两项即给出欧拉-拉格朗日方程, 后面圆括号项给出~$\partial_\mu j^\mu\equiv\partial_\mu\left(\frac{\partial\mathcal{L}}{\partial\partial_\mu\phi}\delta\phi+\mathcal{L}(x)\delta x^\mu\right)=0$. 也就是说, 相应于每一个能保持体系作用量不变的连续时空操作~$\delta x^\mu$, 都有一个
\begin{align}
j^\mu=&\frac{\partial\mathcal{L}}{\partial\partial_\mu\phi}\delta\phi+\mathcal{L}(x)\delta x^\mu\nonumber\\
=&\frac{\partial\mathcal{L}}{\partial\partial_\mu\phi}\delta_{xx'}\phi-\left(\frac{\partial\mathcal{L}}{\partial\partial_\mu\phi}\partial_\nu\phi(x)-\mathcal{L}(x){g^\mu}_\nu\right)\delta x^\nu
\end{align}
为守恒流密度. 这, 称作诺特定理. 当然, 守恒荷密度即四维守恒流矢量的零分量~$j^0$.

\subsubsection{能动张量, 时空平移对称性与能动张量守恒}


我们若作命令
\begin{align}
\mathcal{T}^{\mu\nu}=\frac{\partial\mathcal{L}}{\partial\partial_\mu\phi}\partial^\nu\phi(x)-\mathcal{L}(x)g^{\mu\nu},
\end{align}
--由此可知~$j^\mu=\frac{\partial\mathcal{L}}{\partial\partial_\mu\phi}\delta_{xx'}\phi-\mathcal{T}^{\mu\nu}\delta x_\nu$, -- 则可发现, $\mathcal{T}^{00}=\pi\dot{\phi}-\mathcal{L}=\mathcal{H}$; 此即场在某点附近的能量密度. 也就是说,
\begin{align}
\mathcal{P}^\nu\equiv\mathcal{T}^{0\nu}=\pi\partial^\nu\phi-\mathcal{L}g^{0\nu}
\end{align}
即场的四维动量密度. 当然~$\mathcal{P}^i=\mathcal{T}^{0i}=\pi\partial^i\phi=-\pi\partial_i\phi$ 即场的三维动量密度. 因此, 我们称前述所命的~$\mathcal{T}^{\mu\nu}$ 为能动张量.

若发生的能保持体系作用量不变的操作为时空的平移, 即~$\delta x^\mu=\epsilon^\mu$, 则可知, 对任意场, 即无论是标量场矢量场还是旋量场, 都有~$\delta_{xx'}\phi=0$; 于是有代入诺特守恒流就有~$j^\mu=-\mathcal{T}^{\mu\nu}\epsilon_\nu,~-\partial_\mu j^\mu=\partial_\mu\mathcal{T}^{\mu\nu}=0$. 也就是说, 时空平移对称性导致能动张量守恒.



\subsubsection{空间旋转对称性与角动量守恒, 自旋的出现}

若发生的能保持体系作用量不变的操作为空间的旋转, 则~$\delta x^\mu=\delta x^i={\omega^i}_jx^j=\varepsilon_{ijk}\theta^k x^j$; 前式中已设体系的转动是逆着~$k$ 轴看来顺时针进行的. 对于标量场, 在旋转下亦有~$\delta_{xx'}\phi=0$, 故我们可拿出其对应于空间旋转对称性的守恒量如下
\begin{align}
j^0=&\frac{\partial\mathcal{L}}{\partial\partial_0\phi}\delta_{xx'}\phi-\left(\frac{\partial\mathcal{L}}{\partial\partial_0\phi}\partial_i\phi(x)-\mathcal{L}(x){g^0}_i\right)\delta x^\nu\nonumber\\
=&\mathcal{P}^i\varepsilon_{ijk}\theta^k x^j\\
=&-\mathcal{M}^k\theta^k.
\end{align}
显然, 此即场的轨道角动量密度. 对于矢量场, 这时我们有~$\delta_{xx'}A^i=\varepsilon_{ijk}\theta^k A^j$. 于是对应的守恒量就是
\begin{align}
j^0=&\frac{\partial\mathcal{L}}{\partial\partial_0A^i}\delta_{xx'}A^i-\left(\frac{\partial\mathcal{L}}{\partial\partial_0A^i}\partial_iA^i-\mathcal{L}{g^0}_i\right)\delta x^\nu\nonumber\\
=&\pi^i \varepsilon_{ijk}\theta^k A^j+\mathcal{P}^i\varepsilon_{ijk}\theta^k x^j\nonumber\\
=&-(\mathcal{S}^k+\mathcal{M}^k)\theta^k.
\end{align}
上述结果中, 后一项是轨道角动量密度, 前一项~$\mathcal{S}^k\equiv \varepsilon_{ijk}A^i\pi^j$ 称为自旋角动量密度. 由此看到, 对照于轨道角动量的产生原因是体系的转动, 自旋角动量的产生原因, 是因体系转动而导致的描述体系的场的变化. --当然, 与此同时, 标量场没有自旋角动量, 也就是可以理解的了.

下面我们研究旋量场. 当作空间旋转时, 对旋量场我们有~$\delta_{xx'}\psi=\frac{i}{2}\omega_{\mu\nu}S^{\mu\nu}\psi=\frac{i}{2}\omega_{ij}S^{ij}\psi=-\frac{i}{2}\Sigma^k\theta^k\psi$, 故可得
\begin{align}
j^0=&\frac{\partial\mathcal{L}}{\partial\partial_0\psi}\delta_{xx'}\psi-\left(\frac{\partial\mathcal{L}}{\partial\partial_0\psi}\partial_i\psi(x)-\mathcal{L}(x){g^0}_i\right)\delta x^\nu\nonumber\\
=&\pi\frac{-i}{2}\Sigma^k\theta^k\psi+\mathcal{P}^i\varepsilon_{ijk}\theta^k x^j\nonumber\\
=&-(\mathcal{S}^k+\mathcal{M}^k)\theta^k.
\end{align}
由此可见, 旋量场有角动量密度~$\mathcal{S}^k=\pi\frac{i}{2}\Sigma^k\psi$. 先前由群分析我们知道旋量场自旋为~$1/2$, 在此亦可见一致.



\subsection{规范不变性原理}
\subsubsection{荷场的整体规范变换~(相位平移) 对称性及其守恒荷}


对复数场, 我们来考虑它的相位变换:
\begin{align}\label{U1}
\phi\rightarrow\phi'=e^{iq\gamma}\phi;
\end{align}
此即规范变换之所谓. 如果相位~$\gamma$ 是常数, 上式就称为整体或第一类规范变换; 若是时空坐标的实函数, 就称为定域或第二类规范变换.

首先, 我们来考察场在定域规范对称性下, 将发生什么. 此时, $\delta\phi=iq\gamma\phi,~\delta\phi^\dag=-iq\gamma\phi^\dag$. 将之代入一般的诺特流式, 就得到
\begin{align}
j^\mu=iq\gamma\left(\frac{\partial\mathcal{L}}{\partial\partial_\mu\phi}\phi-\phi^\dag\frac{\partial\mathcal{L}}{\partial\partial_\mu\phi^\dag}\right):=-j^\mu_{(q)}\gamma;
\end{align}
推导上式时要注意复场情况下~$\mathcal{L}=\mathcal{L}(\phi,\partial_\mu\phi;\phi^\dag,\partial_\mu\phi^\dag)$. 上式中,
\begin{align}
j^\mu_{(q)}=\frac{q}{i}\left(\frac{\partial\mathcal{L}}{\partial\partial_\mu\phi}\phi-\phi^\dag\frac{\partial\mathcal{L}}{\partial\partial_\mu\phi^\dag}\right);
\end{align}
即复场的对应于整体规范对称性的守恒流密度; 其中~$q$ 即对应守恒荷. 由本节稍后的内容我们将确认/由以后的具体的场量子化将进一步确认, 此荷, 就是电荷, 弱同位旋, 或色荷等.

当然, 因为简单性, 我们也可以通过直接对哈氏密度变分来求得守恒流:
\begin{align}
\delta\mathcal{L}=&\frac{\partial\mathcal{L}}{\partial\phi}\delta\phi+\delta\phi^\dag\frac{\partial\mathcal{L}}{\partial\phi^\dag}
+\frac{\partial\mathcal{L}}{\partial\partial_\mu\phi}\delta(\partial_\mu\phi)+\delta(\partial_\mu\phi^\dag)\frac{\partial\mathcal{L}}{\partial\partial_\mu\phi^\dag}\nonumber\\
=&iq\gamma\left(\frac{\partial\mathcal{L}}{\partial\phi}\phi-\phi^\dag\frac{\partial\mathcal{L}}{\partial\phi^\dag}
+\frac{\partial\mathcal{L}}{\partial\partial_\mu\phi}\partial_\mu\phi-\partial_\mu\phi^\dag\frac{\partial\mathcal{L}}{\partial\partial_\mu\phi^\dag}\right)\nonumber\\
=&iq\gamma\partial_\mu\left(\frac{\partial\mathcal{L}}{\partial\partial_\mu\phi}\phi-\phi^\dag\frac{\partial\mathcal{L}}{\partial\partial_\mu\phi^\dag}\right)
=-\partial_\mu j^\mu_{(q)}\cdot\gamma;\label{haha}
\end{align}
其中用到了~E-L 方程~$\partial_\mu\frac{\partial\mathcal{L}}{\partial\partial_\mu\phi}-\frac{\partial\mathcal{L}}{\partial\phi}=0$, 并注意~$\delta(\partial_\mu\phi)=iq\gamma\partial_\mu\phi,~\delta(\partial_\mu\phi^\dag)=-iq\gamma\partial_\mu\phi^\dag$. 从上式读出的结果守恒流, 与我们通过一般的诺特流式来做是一模一样的.

具体地, 下面举两个例子. 对复标量场, $\mathcal{L}=\partial_\mu\phi^\dag\partial^\mu\phi-m^2\phi^\dag\phi$, 于是可知其守恒流为~$j^\mu_{(q)}=iq(\phi^\dag\partial^\mu\phi-\phi\partial_\mu\phi^\dag)$. 对于狄拉克场, $\mathcal{L}=\bar{\psi}(i\gamma^\mu\partial_\mu-m)\psi$, 其守恒流即为~$j^\mu_{(q)}=q\bar{\psi}\gamma^\mu\psi$.





%一点评注. 动量守恒导致于空间平移对称性; 这说明体系可以在空间无损耗地自由移动, 从任何地方描述体系都是一样的. 相似地, 荷守恒导致于规范--相位平移--对称性; 说明体系函数可以采取任何相位, 从任何相位看体系都是一样的.

\subsubsection{定域规范变换对称性, 协变导数~$D_\mu$, 荷场与规范场的耦合}


定域情况下, $\gamma=\gamma(x)$, 于是我们有~$\delta(\partial_\mu\phi)=iq\partial_\mu(\gamma\phi)=iq\gamma\partial_\mu\phi+iq\phi\partial_\mu\gamma$; 此时继续对~$\mathcal{L}$ 进变分, 我们就可以得到
\begin{align}
\delta\mathcal{L}=-\partial_\mu \left(\gamma j^\mu_{(q)}\right).
\end{align}
而此变分, 显然是不能再被要求为零的了; 即~$\gamma j^\mu_{(q)}$ 并不能被看作是系统对应于定域规范变换不变性的守恒流. %同时, 从理性上猜测, $j^\mu_{(q)}$ 或许仍是体系的对应于定域规范不变性的守恒流, 但变分的结果却并不是~$\partial_\mu j^\mu_{(q)}=0$ 的形式; 这, 就使我们无法确定先前的猜测.


不难发现, 之所以不能得到类似整体规范时的结论, 是因为定域情况下~($\partial_\mu\phi)'=\partial_\mu\phi'=\partial_\mu(e^{iq\gamma}\phi)=e^{iq\gamma}(\partial_\mu+iq\partial_\mu\gamma)\phi$, 即~$\partial'_\mu$ 无法自由地穿过~$e^{iq\gamma}$ 进而变成~$\partial_\mu$. 病因既明, 医方即出. 我们改造偏微分算符--事实上即是引入与原荷场耦合的某种新场, 称为规范场, --以使当新偏微分算符作到撇的变化时, 这被引入的新场也作具有某种相应行为的到撇的变化, 而这后者正好抵掉~$iq\gamma$ 这一项. 我们将发现, 这不仅是可以做到的, 而且是具有物理意义的. 我们命
\begin{align}
D_\mu:=\partial_\mu+iqA_\mu,
\end{align}
称为协变导数; 并要求当其作~$D_\mu\rightarrow D'_\mu=\partial_\mu+iqA'_\mu$ 时, 被引入的矢量规范场~$A_\mu$ 的变化行为是~$A_\mu\rightarrow A'_\mu=A_\mu-\partial_\mu\gamma$; 如此, 便有
\begin{align}
(D_\mu\phi)'=D'_\mu \phi'=e^{iq\gamma}D_\mu\phi
\end{align}
了. 也就是说, 此时我们就可以得出~$\delta(D_\mu\phi)=iq\gamma D_\mu\phi$ 了.


\paragraph{规范场, 量子电动力学, 量子色动力学}~

通过要求荷场具有定域规范不变性, 我们得到的与之耦合的场, 正是电磁场. 事实上, 不仅电磁场与带电荷场的耦合具有定域规范不变性, 胶子与夸克的耦合, $W^\pm,~Z^0$ 与轻子的耦合等, 也是如此. 于是, 反过来, 我们一般把定域规范不变性当做一条基本原理, 并用它来引入带荷场与规范场之间的相互作用. 这在具体的方法上, 只要把一般偏微分算符~$\partial_\mu$ 代换为协变导数~$D_\mu$ 就可以了. 如同描述物质运动的最小作用量原理, 介定物理规范时空背景框架的相对性原理一样, 规范不变性原理, 亦是物质世界中最重要最深刻的基本原理之一.

这里我们再对规范场的质量问题作一说明. 我们知道, 作为与电荷场耦合的电磁场, 是无静质量的. 事实上, 以后我们将看到, 有质量的场将不具有规范冗余, 即不能参与上述规范变换, 或曰不能作为中间玻色场出现. 而实验表明, 作为弱作用中间玻色子的~$W^\pm,~Z^0$ 粒子却就是具有质量的, 而且还十分地重. 调和这两件事, 将导致对称破缺机制的出场.


进一步, 变换~(\ref{U1}) 是一维复空间何持矢量长度不变的变换, 记为~U(1), 是故以此被引入的电磁场就被称为~U(1) 规范场. U(1) 群元是对易的, 故~U(1) 规范场又称为阿贝尔规范场. 以后我们将明白, 描述弱与强相互作用的分别是~U(2) 与~U(3) 群, 其生成元不对易; 故由此引入的场称为非阿贝尔规范场, 或更常见地, 杨-Mills 规范场.


最后, 我们具体写出荷场与规范场耦合的总拉氏密度. 带电荷的场与电磁场的相互作用, 我们称为量子电动力学. 分别地, 复标量场与电磁场的相互作用, 称为标量电动力学; 按照代换原则, 其拉氏密度即:
\begin{align}
\mathcal{L}=D_\mu\phi^\dag D^\mu\phi-m^2\phi^\dag\phi-\frac{1}{4}F_{\mu\nu}F^{\mu\nu}.
\end{align}
狄拉克旋量场与电磁场的相互作用, 称为旋量电动力学; 其拉氏密度为
\begin{align}
\mathcal{L}=\bar{\psi}(i\gamma^\mu D_\mu-m)\psi-\frac{1}{4}F_{\mu\nu}F^{\mu\nu}=\bar{\psi}(i\gamma^\mu \partial_\mu-m)\psi-q\bar{\psi}\gamma^\mu\psi A_\mu-\frac{1}{4}F_{\mu\nu}F^{\mu\nu}.
\end{align}
另外, 夸克色与胶子场的相互作用, 称为量子色动力学. 这些, 我们都在将以后逐一研究.




\paragraph{协变导数~$D_\mu$ 的运算性质}
~

下面我们继续研究协变导数~$D_\mu$ 的一些重要性质. 首先, 为了使~$D_\mu(\phi\varphi^\dag)$ 这样的式子获得有意义的运算, 我们需要将协变导数的前述定义调整为
\begin{align}
D_\mu\psi=(\partial_\mu+iqA_\mu)\psi,~D_\mu\psi^\dag=(\partial_\mu-iqA_\mu)\psi^\dag.
\end{align}
此时, 我们就可获得~$D_\mu(\phi\varphi^\dag)=\phi D_\mu\varphi^\dag+D_\mu\phi\cdot\varphi^\dag=\partial_\mu(\phi\varphi^\dag)$. 其次, 引入耦合项后的拉氏密度, 即将偏微分改为协变导数后的拉氏密度为~$\mathcal{L}=\mathcal{L}(\phi,D_\mu\phi;\phi^\dag,D_\mu\phi^\dag)$; 这时候, 我们可以通过
\begin{align}
D_\mu\frac{\partial\mathcal{L}}{\partial D_\mu\phi}-\frac{\partial\mathcal{L}}{\partial\phi}=0
\end{align}
来求得荷场的运动方程. 只不过, 此时求出的方程已是含激发源的非齐次方程, 而非自由场方程. 当然, 将协变导数分写开来, 方程~$\partial_\mu\frac{\partial\mathcal{L}}{\partial\partial_\mu\phi}-\frac{\partial\mathcal{L}}{\partial\phi}=0$ 仍是适用的; 求出的方程, 也是含激发源的非齐次方程.

有了以上知识, 我们就可以写出引入耦合后的拉氏量的变分如下
\begin{align}
\delta\mathcal{L}=&iq\gamma(x) D_\mu\left(\frac{\partial\mathcal{L}}{\partial D_\mu\phi}\phi-\phi^\dag\frac{\partial\mathcal{L}}{\partial D_\mu\phi^\dag}\right)
=iq\gamma(x) \partial_\mu\left(\frac{\partial\mathcal{L}}{\partial D_\mu\phi}\phi-\phi^\dag\frac{\partial\mathcal{L}}{\partial D_\mu\phi^\dag}\right)\nonumber\\
=&iq\gamma(x) \partial_\mu\left(\frac{\partial\mathcal{L}}{\partial \partial_\mu\phi}\phi-\phi^\dag\frac{\partial\mathcal{L}}{\partial \partial_\mu\phi^\dag}\right):=-\partial_\mu J^\mu_{(q)}\cdot\gamma(x);
\end{align}
其中~$J^\mu_{(q)}$ 为荷场对应于定域规范不变性的守恒流, 称为协变守恒流.





最后, 我们主要以复标量场为例, 对上述一般手续作一次具体展示, 并由此进一步确信引入耦合的代换原则的合理性.

在上述自由复标量场拉氏密度中作代换, 则引入耦合项的拉氏密度即~$\mathcal{L}=D_\mu\phi^\dag D^\mu\phi-m^2\phi^\dag\phi=\mathcal{L}_{free}+iqA^\mu\phi\partial_\mu\phi^\dag-iqA_\mu\phi^\dag\partial^\mu\phi+q^2A^2\phi^\dag\phi=\mathcal{L}_{free}-A_\mu j^\mu_{(q)}+q^2A^2\phi^\dag\phi$. 场的运动方程就是~$(D_\mu D^\mu+m^2)\phi=0\rightarrow(\partial_\mu\partial^\mu+m^2)\phi=q^2A^2\phi-2iqA^\mu\partial_\mu\phi-iq\phi\partial^\mu A_\mu$. 而将电磁场的拉氏量与其与复标量场的耦合项合写起来, 对电磁场运用拉氏方程, 我们就可得到~$\partial_\mu F^{\mu\nu}=-J^\mu_{(q)}$.

我们重点考察守恒流. 复场的对应于定域规范不变性的协变守恒流是
\begin{align}
J^\mu_{(q)}=&\frac{q}{i}\left(\frac{\partial\mathcal{L}}{\partial D_\mu\phi}\phi-\phi^\dag\frac{\partial\mathcal{L}}{\partial D_\mu\phi^\dag}\right)=iq(\phi^\dag D^\mu\phi-\phi D^\mu\phi^\dag)\nonumber\\
=&j^\mu_{(q)}-2q^2A^\mu\phi^\dag\phi.
\end{align}
当然, 上述是的~$j^\mu_{(q)}$ 只是形式与自由场时的守恒流相同; 其中的~$\phi$ 满足的具体运动方程是不同的: 一个是自由场的齐次方程, 另一是含激发源的非齐次方程. 以对~$\mathcal{L}=\mathcal{L}(\phi,\partial_\mu\phi;\phi^\dag,\partial_\mu\phi^\dag;A_\mu)$ 进行直接变分的方法, 也可以求出协变守恒流:
\begin{align}
\delta\mathcal{L}=&\frac{\partial\mathcal{L}}{\partial\phi}\delta\phi+\delta\phi^\dag\frac{\partial\mathcal{L}}{\partial\phi^\dag}
+\frac{\partial\mathcal{L}}{\partial\partial_\mu\phi}\delta(\partial_\mu\phi)+\delta(\partial_\mu\phi^\dag)\frac{\partial\mathcal{L}}{\partial\partial_\mu\phi^\dag}+\frac{\partial\mathcal{L}}{\partial A_\mu}\delta A_\mu\nonumber\\
=&-\gamma(x)\partial_\mu \left(j^\mu_{(q)}-2q^2A^\mu\phi^\dag\phi\right).
\end{align}
注意上述推导中我们亦用到了~EL 方程~$\partial_\mu\frac{\partial\mathcal{L}}{\partial\partial_\mu\phi}-\frac{\partial\mathcal{L}}{\partial\phi}=0$; 但此时的拉氏密度已是引入耦合项后的拉氏密度, 而非推出式~(\ref{haha}) 过程中所采用的自由场拉氏密度. --或曰场~$\phi$ 满足的具体运动方程与先前采用的是不同的. 对于狄拉克场, 不难发现其协变守恒流~$J^\mu_{(q)}=q\bar{\psi}\gamma^\mu\psi$; 当然其亦仅在形式上与自由场时之守恒流是相同的.

上述一系列计算, 用协变导数做以及用偏微分对全拉氏量去做, 的确是等价的. 由此我们确信, 用由偏微分到协变导数的代换来引入相互作用, 的确是自洽而简洁的. 而它的最终合理性, 也将由实验得到完全证明.






