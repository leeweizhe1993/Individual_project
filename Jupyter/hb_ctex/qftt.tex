\documentclass[UTF8,a4paper]{article}
\usepackage{CJK}
\usepackage{amsmath}
\usepackage{bm}%此包与上一个包配合使用, 可用 \bm 命令同时加粗行中公式与单列公式
\usepackage{amsfonts}%空心字母? 实测不可用
\usepackage{amssymb}
\usepackage{cite}
\usepackage[CJKbookmarks]{hyperref}
\usepackage{graphicx}%加载 jpg 图片
\usepackage{simplewick}%使用缩并的包
\linespread{1.3}%调整行间距
\usepackage{mathrsfs}%花体
\usepackage{mathtools,slashed}%费曼斜线
\newcommand\hatslashed[1]{{\hat{#1}\mathllap{\slashed{#1}}}}%使费曼斜线与算符帽更紧凑的一个定义
\usepackage{centernot}%斜线中央版本
%不用任何包, 可写\not 命令(这个线偏左. 所以要\centernot).  但是, \not, \centernot, 以及 \slashed 三个命令, 还是最后一个最漂亮


%\usepackage{fancy}%页眉页脚包
%\pagestyle{plain}%页眉页脚设置

%\usepackage{geometry}%页边距包
%\geometry{left=3.9cm,right=3.9cm,top=3.9cm,bottom=3.9cm}%页边距参数

%场变成了算符, 作用于粒子数上. 那么, 谁是场? 那个场算符可以是, 粒子数基, 也可以是. 另外, 粒子数基, 必满足薛定谔方程.
\begin{document}
\begin{CJK*}{GBK}{}
\title{量子场论}
\author{项海波}
\maketitle
\thispagestyle{empty}%去掉本页页码

\newpage
\pagenumbering{Roman}
\tableofcontents
\newpage
\listoffigures
\newpage
\listoftables
\newpage
\pagenumbering{arabic}


\input{part/part1}\newpage
\input{part/part2}\newpage

\section{ʵ������: �������������޺�, $\pi^0$ ����}

%���˵ʱ���е�һ�ֲַ�, ����һ�ֳ��Ļ�, ��ʲô��������, ������ѧ�еĸ��ʷ�Ҳ��. ���ӳ�����, �����ᵽ�ij�, һ��Ĭ�Ͼ��dz����. �������, �������Ͻ�, ��ľ�ֻ�dz��IJ����������. �����ϵij�, ��ʵ����Ϊ����������ÿռ�Ķ�����̬. ������������. (�����, ���������������ϵ����, �������������������ϵ����!!!!!!!!!!)

%������ʱ, �����ְѳ����ֱ�ӳ�Ϊij��; ��ʵ�ϳ���ij��������ÿռ��.

%Ϊʲô�������ڹ���DZ�Ҫ��? ��Ϊ, һ, ���Dz��������������Ƿ����; ��, ������������ݻ�, ����Ѧ����, ���ܿ����ó�, ��������ʹ���ǻ�ȡ������Ϣ; ��, ��������۶��ܹ�ϵ��, ǡ�������������ϵIJ����������
%��, ��ʲô???
%����ij����, ĩ�����, ���Ǿ��������۳�. ��, һ, ����û�����˼; ��, ���㿴�����ӵĸ��ʷ�, Ҳ��̫��ȷ;

 %������Ƿ���, ������˵��~$\phi$, �����, �����DZ�����������ӻ��ϵIJ����������, ��ʵ��������!!!!!
 %��ô, �����ӳ�����, ����ʲô, �ش�һ, �����; ��, ����������õľ�ϣ�����ؿռ��еĶ�����̬.

 %��Ի, ����������ÿռ���ʲô������������!|n_1,\cdots,n_i\rangle, ��Ҳ���Խг���һ��λ��, ��һ��λ������һ����������.

%��Ϊ������������\phi �����, ����, ���ӳ���������һ�㶼���ں�ɭ���澰, ���һ���, �໥���û澰�¹�����.


%ij������ϵ��, ����������һ�����������, ������ֱ��, ���, ����������������!!!!!!!!!!!!!!1
\subsection{�������ķ���: Klein-Gordon ����; ���и�����ij���, �����������ڹ��, ��λ��~(configuration)}
%���������ܶ�, ׼ȷ�Ľ�, �г��������ܶ����. ������, ׼ȷ�ؽ�, �г��������. ���������, �������ij�~(����λ��, ����������) ����, ��Ȼ, ����ͬ, ��, �ǵȼ۵�!

%\phi ��λ��, �����������ϵIJ����������, ���߿���˵���������ij�. ��, ǰ��, ��ǰ���������֮ǰ, ��Ӧ��, �dz�������!! ������������, ��Ӽ�������۹�ϵ��, ���ǰѳ����ֱ�ӽ�����!~~~$�������ӵ�·�����̷�Ωһ��, ��Ӧ�ڳ����̵ķ�Ωһ��!!!!!!!!!!1
%��Ҫ!!!!!!!!!!!!!!!!!����, �������ӵ���С������·�����̲�����̫������, ��˲�ͬ, ���Ķ�Ӧ����С�������ķ���, �Դ�������!

%Ѧ���̷�����, ��Ҫ��~$E=T+V$ ���Ͼ�����˶�·��!!!!!!!!

~~~~�������������������֪, ��������ӵ��ܶ���ϵΪ~$p^2-m^2=0$; ��������ѧ, ������֪�Ӿ��䵽���ӵ�һ�����ԭ��Ϊ~$p^\mu=i\partial^\mu,~p_\mu=i\partial_\mu$; �ݴ�, ���ǿɵ÷���
\begin{align}
(\partial^2+m^2)\phi(x)=0;
\end{align}
��Ϊ~Klein-Gordon ����. �˷���, �������������ȱ���, ʸ������������, ����Ȼ�����; �����, ʸ��������������ǿҪ���ʽ��; ֻ������ʽ���޸�ǿ��ϵ��--Ҳ���DZ��½��о��ij�, ��ֻ������Ϊ~0 �ı�����. �ڶ���, �������к���������, ���ʾ��Ӧ����������. ���, ���µĴ󲿷�ƪ����, ���ǽ�ȡ����Ϊʵ, ��~$\phi^\dag=\phi$. ��ǰ�ķ�����ѧһ��������֪, ����ζ�ű��������������޺�. ��֮, ��������������Ϊ������, ������, �޺�; ���Ӧ����Ȼ���е�����, ��~$\pi^0$ ����.

%�����о����̵Ľ�.
�������Ǿ����о������̵Ľ�. �ɷ��̵���ʽ�ɿ���, ���������ֽ�, �ֱ���~$\propto e^{-ip^\mu x_\mu}$ ��--��Ϊ���ܽ�, ��~$\propto e^{ip^\mu x_\mu}$ ��-- ��Ϊ���ܽ�. ��ǰ��ƽ�沨���ӳɵ�~K-G ���̵�һ��ʵ��Ϊ
\begin{align}\label{so of kg}
\phi(x)&=\int\frac{d^3\bm{p}}{(2\pi)^3\sqrt{2E_{\bm{p}}}}\left(a_{\bm{p}}e^{-ipx}+a^\dag_{\bm{p}}e^{ipx}\right)\nonumber\\
\bigg(:&=\phi^+(x)+\phi^-(x):=\int d^3\bm{p}\big[\varphi_{\bm{p}}(x) a_{\bm{p}}+\varphi^*_{\bm{p}}(x) a^\dag_{\bm{p}}\big]\bigg);
\end{align}
��������~$\frac{1}{(2\pi)^3}$ �����һ�������ֵ�, ��������֪��; ��~$\frac{1}{\sqrt{2E_{\bm{p}}}}$ ��������۲����Զ����ֵ�, ֤������: ��ѧ������֪\footnote{�ο�~ $\int dx\delta(cx)=\frac{1}{|c|},~\delta(g(x))=\sum_i\frac{\delta(x-x_i)}{|g'(x_i)|}$; ��~$\delta(x^2-c^2)=\frac{1}{2|c|}[\delta(x-c)+\delta(x+c)]$.}~$\int dx\delta(g(x))=\sum_i\frac{1}{|g'(x_i)|}$, ���~$\int dp^0\delta(p^2-m^2)\theta(p_0)=\frac{1}{|2p^0|_{p^0=+E_{\bm{p}}}|}=\frac{1}{2E_{\bm{p}}}$, ��������۲����
\begin{align}
\int \frac{d^4p}{(2\pi)^4}2\pi\delta(p^2-m^2)\theta(p_0)=\int\frac{d^3\bm{p}}{(2\pi)^32E_{\bm{p}}}.
\end{align}
ע��, ����ʽ����ߵ�~$\delta$ ����, ����ά��������~$\int d^4p$ Լ��Ϊ���������ɫɢ��ϵ�Ļ���~(����˲��ܵõ��ұߵĽ��), ���������Ժ�����������������ӻ���ά�����ռ��еĵ�����~$\delta$ �������еĶ���ά�����Ļ���~(�ѱ��������ص�������ƽ��) ����Ҫ��ɫɢ��ϵ. --�Ժ󽫼�, ����~$a_{\bm{p}}$ ������������Э���, $\sqrt{2E_{\bm{p}}}a_{\bm{p}}$ ����, ���������ȱ���; ��������ȷ������������һ����������ȱ���.


%��~$\frac{1}{\sqrt{2E_{\bm{p}}}}$ ���յ���~$a^\dag_{\bm{p}},~a_{\bm{p}}$ ֮��, ���Ǿ��������ڵĽ��.


%�����~K-G ���̼������һ������. ����,
�Ӹ��ܽ�ij���, ���ǾͿɷ���, ������~$\phi(x)$ ڹ��Ϊ���������Ӹ��ʷ��IJ�����, �򼴼�����~K-G ����ڹ��Ϊ�����ӵĸ��ʲ�������, ���Dz����ܵ���: ��������������������. ��ʵ��, �Ժ����Ǽ���ȷ֪, �����~$a^\dag_{\bm{p}},~a_{\bm{p}}$ Ӧ��ڹ��Ϊ����ȷ�����������ӵIJ����������\footnote{����������~$i$ ��г���ӵ�~$x,~p$ �ϳ�~$a,~a^\dag$, �Ӷ��Բ�������������ֶεõ���г�����ܼ�.}~(�Ӷ�~$\phi(x)$ ���Ǿ���ȷ���ռ�λ�õ����ӵIJ����������; K-G ���̾�����������㷽��).% �����������ʽ���Կ���, �����Ժ�ľ�����������Ǹ���ȷ��,
~�����, ������/�����������ܲ���, ��Ӧ�����ӵ�����; ���ܲ���, ��Ӧ�����ӵIJ���. �ݴ�, �������, ����һ���ֽ���������ֱ��Ϊ����Ƶ��. --�����Ӹ��ʲ�ڹ���������ĸ�������, ���˵����������.

%���, K-G ����, �������������ȱ���, ʸ����������, ����Ȼ�����; �����, ʸ��������������ǿҪ���ʽ��; ֻ������ʽ���޸�ǿ��ϵ��--Ҳ���DZ���, ��ֻ������Ϊ~0 �ı�����. ��˵������/�������������ӵ�����Ϊ~0. ����, �ɷ��̵���ʽ�ɼ�, ����ȷ��������Э���; ��������￴��, ��~$\phi(x)$ Ҳ��ȷ�������ȱ���. ����,




�����ڹ��֮ǰ, ��������Ⱥ��ʾ����, ���ǰ�~$\phi(x)$ ֱ�ӳ�Ϊ��; �ɵ�ų�~(���������һ�����ӻ�) ����������˹Τ���̿ɿ���, ~$\phi(x)$ ���г��Ƶĵ�λ. �����~(��˰�����ǰ������ѧ�����ó��ij��ĸ�����ѧ��, Ҳ�ͳ�����Ӧ����ѧ�����) �Ժ�, $\phi(x)$ �ɳ�֮Ϊ�����. ��Ȼ, �������ö��������������, �����ij���������ֲ�������̬:
\begin{align}
|\phi\rangle:=|n_1,\cdots,n_i\rangle;
\end{align}
һ���, ���dz���Ϊ����һ���ض���λ��~(configuration). �����Ժ�ͽ�����, ��λ�ε���ʱ�ݻ��������㷽��~$i\frac{\partial}{\partial t}|\phi\rangle_{S/I}=H_{\textrm{total}/I}|\phi\rangle_{S/I}$ ��, �������໥���ó��۵ĸ�������֮һ.





���, ���ѿ���, ������
\begin{gather}\label{rela.}
\int d^3\bm{x}\varphi_{\bm{p}}(x)i\overset{\leftrightarrow}{\partial}_0\varphi_{\bm{p}'}(x)=0,\\\label{rela.2}
\int d^3\bm{x}\varphi^*_{\bm{p}}(x)i\overset{\leftrightarrow}{\partial}_0\varphi_{\bm{p}'}(x)=\frac{1}{(2\pi)^3}\delta^3(\bm{p}-\bm{p}');
\end{gather}
����~$A\overset{\leftrightarrow}{\partial}_\mu B:=A\partial_\mu B-\partial_\mu A\cdot B$. ǡ����������ʽ, ��ʹ������̱�ô�Ϊ��������. ����ʽ��ɿ���
\begin{gather}\label{diejia}
a_{\bm{p}}=(2\pi)^3\int d^3\bm{x}\varphi^*_{\bm{p}}(x)i\overset{\leftrightarrow}{\partial}_0\phi(x),\\
a^\dag_{\bm{p}}=-(2\pi)^3\int d^3\bm{x}\varphi_{\bm{p}}(x)i\overset{\leftrightarrow}{\partial}_0\phi(x).
\end{gather}


\subsection{�������ӻ�: �����ռ���׹�ϵ�ĸ��������ӽ�Ļ��; ��λ�ε��˶�����}
~~~~
%������һС�ڵķ���, ����~$\psi(x)$ ����ij�ض�����ϵ�ϵIJ����������������,
%����, ���ǾͿɰ�~$\phi(x)$ ��������һ����������ϵ���������,
����һС�ڵĻ�����, ���������÷�����ѧ�ķ������ҳ������ѧ��~(���). ����, �ܵ��±������̵������ܶ�Ϊ
\begin{align}
\mathcal{L}=\frac{1}{2}(\partial_\mu\phi\partial^\mu\phi-m^2\phi^2);
\end{align}
��Ȼ��ʽΪ�����ȱ���, --�������г��������ܶȶ�Ӧ�����. ����, �ɼ�������Ĺ����Ϊ
\begin{align}
\pi(x)=\frac{\partial\mathcal{L}}{\partial\dot{\phi}}=\dot{\phi}=\int\frac{d^3\bm{p}}{(2\pi)^3}(-i)\sqrt{\frac{E_{\bm{p}}}{2}}\left(a_{\bm{p}}e^{-ipx}-a^\dag_{\bm{p}}e^{ipx}\right);
\end{align}
ע�ⳡ�Ĺ��������������Э����. ������, ��Ϊ�����������, �����ڶ����ռ��жԱ����������¶��׹�ϵ:
\begin{gather}
[a_{\bm{p}},a^\dag_{\bm{p}'}]=(2\pi)^3\delta^3(\bm{p}-\bm{p}'),\\
[a_{\bm{p}},a_{\bm{p}'}]=[a^\dag_{\bm{p}},a^\dag_{\bm{p}'}]=0;
\end{gather}
����Ϊʲô����˶��׹�ϵ�����Ƿ����׹�ϵ, �����Ժ���ѧ���Ƶ����̿��Կ���, ���ɺ��ĵ�����ɵ��Ը���˵��. ��������׼��, �Ϳ���ñ��������������ӵ���ѧ����, ������Ϊ\footnote{һ��С����: Ϊ��������, ���ǿ������Ƶ������������~$K=\int \frac{d^3\bm{p}}{(2\pi)^3\sqrt{2E_{\bm{p}}}}$.}
\begin{align}
H=&\int d^3\bm{x}(\mathcal{\pi\dot{\phi}-\mathcal{L}})=\int d^3\bm{x}\frac{1}{2}\left[\dot{\phi}^2+(\nabla\phi)^2+m^2\phi^2\right]\nonumber\\
=&\int d^3\bm{x}\frac{1}{2}(\partial_\mu\phi\partial_\mu\phi+m^2\phi^2)\nonumber\\
=&\frac{1}{2}\int \frac{d^3\bm{p}}{(2\pi)^32E_{\bm{p}}}
\left(p_\mu p_\mu a_{\bm{p}}a^\dag_{\bm{p}}+p_\mu p_\mu a^\dag_{\bm{p}} a_{\bm{p}}+m^2 a_{\bm{p}}a^\dag_{\bm{p}}+m^2 a^\dag_{\bm{p}}a_{\bm{p}}\right)\nonumber\\
=&\frac{1}{2}\int \frac{d^3\bm{p}}{(2\pi)^32E_{\bm{p}}}(p_\mu p_\mu+m^2)(a_{\bm{p}}a^\dag_{\bm{p}}+a^\dag_{\bm{p}}a_{\bm{p}})\nonumber\\
=&\int \frac{d^3\bm{p}}{(2\pi)^3}\frac{E_{\bm{p}}}{2}(a_{\bm{p}}a^\dag_{\bm{p}}+a^\dag_{\bm{p}}a_{\bm{p}})\nonumber\\
=&\int\frac{d^3\bm{p}}{(2\pi)^3}\frac{E_{\bm{p}}}{2}\left[2a^\dag_{\bm{p}}a_{\bm{p}}+(2\pi)^3\delta^3(0)\right];
\end{align}
�����õ��˹�һ��\footnote{��Ϊ�Ա�, ��������ά���, ������
\begin{align}
\delta^3(\bm{p}-\bm{p}')=\frac{1}{(2\pi\hbar)^3}\int d^3\bm{r}e^{i(\bm{p}'-\bm{p})\cdot\bm{r}/\hbar}.
\end{align}
--�����Ժ󻹽�����~$\delta^4(p-p')$ �ı���.}
\begin{align}
\delta^3(\bm{p}-\bm{p}')=\int \frac{d^3\bm{x}}{(2\pi)^3}e^{\pm i(p-p')x}.
\end{align}
����ȷ��, �����ò������������ڹ�ͺ�����, ��ȷ�õ���һ�����Խ��ܵ�����. ����, �����������������֪��, ��Ϊ��ά�������������������Ȼ������������Э����, �������������ܶ��г�����~$p_\mu p_\mu$ ������ϵ���ֵ��һ�µ�. ���Ƶ�, ���������ó�/���ӵ��ܶ�������
\begin{align}
P=&\int d^3\bm{x}\mathcal{P}=\int d^3\bm{x} (-\pi\nabla\phi)=\int d^3\bm{x} (-\dot{\phi}\nabla\phi)\nonumber\\
=&\int\frac{d^3\bm{p}}{(2\pi)^3}\frac{\bm{p}}{2}\left[2a^\dag_{\bm{p}}a_{\bm{p}}+(2\pi)^3\delta^3(0)\right].
\end{align}

����, ���ǽ���һ�¼��㶯���ռ�����ѧ������ʽ����һ�ַ���. ��ʵ��, �˷����ĺ���������dz�����ù�ϵʽ~(\ref{rela.}) ��~(\ref{rela.2}). ����ȫɢ����, �Լ�Ӧ�ó������~K-G ����, ���ǾͿɵ�
\begin{align}
H=&\int d^3\bm{x}\frac{1}{2}\left[(\partial_0\phi)^2+(\nabla\phi)^2+m^2\phi^2\right]\nonumber\\
=&\int d^3\bm{x}\frac{1}{2}\left[(\partial_0\phi)^2+\nabla\cdot(\phi\nabla\phi)-\phi\nabla^2\phi-\phi\partial^2\phi\right]\nonumber\\
=&\int d^3\bm{x}\frac{1}{2}\left[(\partial_0\phi)^2-\phi\partial_0^2\phi\right]\nonumber\\
=&\frac{1}{2}\int d^3\bm{x}i\phi i\overset{\leftrightarrow}{\partial}_0\partial_0\phi;
\end{align}
�ɴ�, ��Ϲ�ϵʽ~(\ref{rela.}) ��~(\ref{rela.2}), ���ǿ������������ֽ��. ����, �Զ���, ����Ҳ�ɼ����
\begin{align}
P^i=&\int d^3\bm{x}\partial_0\phi\partial^i\phi=\int d^3\bm{x}\left[\partial_0(\phi\partial^i\phi)-\phi\partial_0\partial^i\phi\right]\nonumber\\
=&\frac{1}{2}\int d^3\bm{x}\left[\partial_0\phi\partial^i\phi-\phi\partial_0\partial^i\phi\right]\nonumber\\
=&\frac{1}{2}\int d^3\bm{x}i\phi i\overset{\leftrightarrow}{\partial}_0\partial^i\phi;
\end{align}
�ɴ˶������ֽ��������Ȼ��. ����ʽ�����̿�֪���ǿɺ�д����ά�ܶ�ʸ��, ��һ����������ֽ��Ϊ
\begin{align}
P^\mu=&\frac{1}{2}\int d^3\bm{x}i\phi i\overset{\leftrightarrow}{\partial}_0\partial^\mu\phi\nonumber\\
=&\int \frac{d^3\bm{p}}{(2\pi)^3}\frac{p^\mu}{2}(a_{\bm{p}}a^\dag_{\bm{p}}+a^\dag_{\bm{p}}a_{\bm{p}}).
\end{align}









%���, ���ǽ�һ�º�ɭ�������볡���̵�һ����.
��ǰ�Ĺ������ǿ��Կ���, �ڴ�����������, �����ʱ, ��λ�β���ʱ; �������ǹ����ں�ɭ���澰�е�. ���ǿ������~$i\frac{\partial}{\partial t}\phi(x)=i\pi(x)=[\phi(x),H],~i\frac{\partial}{\partial t}\pi(x)=i(\nabla^2-m^2)\phi(x)=[\pi(x),H]$, �Ӷ���~$\frac{\partial^2}{\partial t^2}\phi=(\nabla^2-m^2)\phi$, ��~K-G ����. Ҳ����˵, �����ӳ�����, �������к�ɭ������
\begin{align}
i\frac{\partial}{\partial t}\mathcal{O}=[\mathcal{O},H]
\end{align}
�dz�����. �Ӷ�Ҳ����ζ��, �����Կɽ���дΪ~$\phi(x)=e^{iHt}\phi(\bm{x})e^{-iHt}$ ��������ʽ. ��һ��, $U(t,0):=e^{-iHt}$ �Ϳ��Կ����DZ���λ�εĴ�ʱ���ݻ����. �����Ժ��о������໥���õ����, ����~$\phi_I(x):=e^{iH_0t}\phi(\bm{x})e^{-iH_0t}$, �Լ�~$\phi_H=e^{iHt}e^{-iH_0t}\phi_Ie^{iH_0t}e^{-iHt}$. ǰ�����һʽ�м���~$U_I(t,0):=e^{iH_0t}e^{-iHt}$; ����~$i\hbar\frac{\partial}{\partial t}U_I=H'_IU_I$. ��Щ����, ��������ѧ�ж�����ȫһ�µ�. �Զ���֮, ���������Ϊ����, ת����Ѧ���̻��໥���û澰��, ���Ǿ���
\begin{align}
i\frac{\partial}{\partial t}|\phi\rangle_{S/I}=H_{\textrm{total}/I}|\phi\rangle_{S/I}
\end{align}
�dz�λ�ε��˶�����. ����ǰ���Ѿ�˵��, �Ժ󼴽�������, ��ʽ���໥���ó��۵���Ҫ����.


%�����, ����˵, ��Ѧ���̷��������оͿɷ��ֵĺ�ɭ������, ��ǰ�߸��߹㷺��.


%��������, ����������, ������

\subsection{��ɢ��̬��, �����׶�ЧӦ}


~~~~
���Կ���, ������Ϊ��, �ڻ�̬ʱ, ��ÿ������ģ������Ȼ����~$\frac{1}{2}E_{\bm{p}}$ ������. ��Ȼ�Էֲ���ȫ�ռ�/�����ռ�ij���˵, �ܻ�ֵ̬���Ƿ�ɢ��. %��ʵ��, ���Ǿ���һ���Ե�: ���ӳ����е���ѧ�������ٻ�ֵ̬��ɢ������.
��Ϊһ����˵, ���ǹ��ĵ�������������ϵ����, �����ǰѴ˷�ɢֱֵ����ȥ�Dz��������. �򵥵�ͨ��������滯~$N(a_{\bm{p}}a^\dag_{\bm{p}}+a^\dag_{\bm{p}}a_{\bm{p}})=2a^\dag_{\bm{p}}a_{\bm{p}}$ ����ʵ����һ��.
%
%����ܷ�ɢ, Ҳ�������ӳ��������������ĵ�һ����ɢ; �Ժ����ǻ������������ķ�ɢ. ��Щ��ɢ���������ӳ��۵Ļ�غ���������������. �����Щ���ѵ�ϵͳ���ķ����ѱ���չ����, �����������; ���˷����Ա�����֮���, ���Ǵ���ѧ�����ϵ�, �������������ϵ�. �������������ǽ����ѽ���, һ������, ֻҪ����ij�ֳ߶����ǿ���������, ��ô��������Ӧ�ij߶�������ȷ��, ��Ի��Ч��. ����˼��, ������Ч����.
%
%
%һ����˵, ; ��ʱ, ��һ�ڰ����������Dz��������. ����, ���ǿ��԰�������ɢ������ܵ�����һ����ѧ�ϵ�����. ����, �������֮��ѧ���, ���Ԥʾ��ij����������, ��ô���Ǿͱ�������Դ���. ��ʵ��, �����, ��ȷ����ijЩ״���»���ֳ���, ���翨���׶�ЧӦ. �������Ǿ��о�֮.
%
%
%
%
%
%
%
\begin{figure}[!h]
\begin{center}
\includegraphics[width=4 cm]{pic/Casimir.jpg}
\caption{�����׶�ЧӦ.}
\label{Casimir}
\end{center}
\end{figure}

����, ��̬�ܵij���, ��ζ��������ν�����Ҳ������������ô�տ���Ҳ��. ���������֮����������, ��Ϊ�����׶�ЧӦ; ��ЧӦ, �Ϳ����õ�ų����������˵��. ��������ƽ�н��������Ϊ~$d$, ��ƽ����~$x-y$ ƽ��; ����������, Ϊ~$A$. �������ǿɵ�~$d$ �����, �Ե�ų�, ��λ����ϵĻ�̬����Ϊ
\begin{align}
\frac{\langle E\rangle}{A}=2\int \frac{dk_xdk_y}{(2\pi)^2}\sum_{n=1}^\infty\frac{\hbar}{2}\omega_n=2\int \frac{dk_xdk_y}{(2\pi)^2}\sum_{n=1}^\infty\frac{\hbar}{2}c\sqrt{k^2_x+k^2_y+\left(\frac{n\pi}{d}\right)^2};
\end{align}
���ж�~$z$ �������Dz���������ѧ������, ��~$x, y$ �������Dz��������ڱ߽�����, ��~$k=\frac{2\pi}{L}$. ��������ֵ�Ƿ�ɢ��; ���Dz���~zeta ���滯, ��ת�Ƶ�������֮��, ����
\begin{align}
\frac{\langle E(s)\rangle}{A}=&\hbar\int \frac{dk_xdk_y}{(2\pi)^2}\sum_{n=1}^\infty\omega_n|\omega_n|^{-s}\nonumber\\
=&\frac{\hbar c^{1-s}}{4\pi^2}\sum_n\int_0^\infty 2\pi q dq\Big|q^2+\frac{n^2\pi^2}{d^2}\Big|^{(1-s)/2}\nonumber\\
=&-\frac{\hbar^{1-s}\pi^{2-s}}{2a^{3-s}}\frac{1}{3-s}\sum_n|n|^{3-s}.
\end{align}
������DZ�����
\begin{align}
\frac{\langle E\rangle}{A}=\lim_{s\rightarrow0}\frac{\langle E(s)\rangle}{A}=-\frac{\hbar c\pi^2}{6a^3}\zeta(-3)=-\frac{\hbar c\pi^2}{720a^3};
\end{align}
�����õ���~$\zeta(-3)=\frac{1}{120}$. ���ǿ����׶�������
\begin{align}
\frac{F_c}{A}=-\frac{\partial}{\partial d}\frac{\langle E\rangle}{A}=-\frac{\hbar c\pi^2}{240a^4}.
\end{align}



\subsection{������̬����������һ������, ����ռ䳡������׹�ϵ, Э�����ʽ}

~~~~
��������ѧ����������֪, ǰ�Ķ����ռ䳡����Ķ��׹�ϵ, ��ζ��~$a^\dag_{\bm{p}}$ �����ӵIJ������, ��~$a^\dag_{\bm{p}}|0\rangle=|\bm{p}\rangle$. ���ǿ�֪~$\langle0|[a_{\bm{p}},a^\dag_{\bm{p}'}]|0\rangle=\langle0|a_{\bm{p}}a^\dag_{\bm{p}'}|0\rangle=\langle\bm{p}|\bm{p}'\rangle=(2\pi)^3\delta^3(\bm{p}-\bm{p}')$. ��һ����,
\begin{align}
\phi(x)|0\rangle=&\phi^\dag(x)|0\rangle=\int\frac{d^3\bm{p}}{(2\pi)^3\sqrt{2E_{\bm{p}}}}e^{ipx}a^\dag_{\bm{p}}|0\rangle\nonumber\\
=&\int\frac{d^3\bm{p}}{(2\pi)^3\sqrt{2E_{\bm{p}}}}e^{ipx}|\bm{p}\rangle:=|x\rangle
\end{align}
�ͱ�ʾ��~$x$ �㴦����һ���ɷֽ�Ϊ���ģ���ӵ�����; ע����������̬�������ȱ���, �����κι���ϵ, ������ͬ��������. �ɴ˿��Եõ�\footnote{��Ϊ�Ա�, $\delta^3(\bm{r}-\bm{r}')=\frac{1}{(2\pi\hbar)^3}\int d^3\bm{p}e^{i\bm{p}\cdot(\bm{r}-\bm{r}')/\hbar}$.}
\begin{align}
\langle x|y\rangle:=\langle0|\phi(x)\phi(y)|0\rangle=\int\frac{d^3\bm{p}}{(2\pi)^32E_{\bm{p}}}e^{-ip(x-y)}:=D(x-y)
\end{align}
���������Ӵ�~$y$ ��~$x$ �Ĵ���. ע��˴�����ϵ���������ȱ���.

��ѧ������֪��, $\delta(f(x)-f(x_0))=\frac{1}{|f'(x_0)|}\delta(x-x_0)$, �ʿɼ����
\begin{align}
\delta^3(\bm{p}-\bm{p}')=&\delta^3(\bm{p}_b-\bm{p}'_b)\frac{dp^i_b}{dp^i}=\delta^3(\bm{p}_b-\bm{p}'_b)\gamma(1+\beta\frac{dE_{\bm{p}}}{dp^i})\nonumber\\
=&\delta^3(\bm{p}_b-\bm{p}'_b)\frac{\gamma}{E_{\bm{p}}}(E_{\bm{p}}+\beta p^i)=\delta^3(\bm{p}_b-\bm{p}'_b)\frac{E_{\bm{p}_b}}{E_{\bm{p}}};
\end{align}
�����õ�������������άʸ���������ȱ任ʽ~(\ref{titi}). ���������̿ɼ�, $E_{\bm{p}}\delta^3(\bm{p}-\bm{p}')$ �������Ȳ����. ������ƽ�ֵ��������������, �͵�~$\sqrt{2E_{\bm{p}}}a_{\bm{p}}$ �������ȱ���. �ڱ���~K-G �����̽�~(\ref{so of kg}) �������ȱ���ʱ, ǰ�����ᵽ���������, ���ڵ��Խ��\footnote{�����һ������������ʵ�ռ�ת���µı仯��Ϊ���������Ƶķ�ʽ����, ����~$U(\Lambda_p)\sqrt{2E_{\bm{p}}}a_{\bm{p}}=\sqrt{2E_{\Lambda^{-1}_p\bm{p}}}a_{\Lambda^{-1}_p\bm{p}}=\sqrt{2E_{\bm{p}}}a_{\bm{p}}$.}.~%; ������~$U(\Lambda_p)a_{\bm{p}}=\frac{\sqrt{2E_{\Lambda^{-1}_p\bm{p}}}}{\sqrt{2E_{\bm{p}}}}a_{\Lambda^{-1}_p\bm{p}}$.}
%
%\footnote{�����һ������������ʵ�ռ�ת���µı仯��Ϊ���������Ƶķ�ʽ����, ����~$\sqrt{2E_{\bm{p}}}a_{\bm{p}}\rightarrow\sqrt{2E'_{\bm{p}}}a'_{\bm{p}}=\sqrt{2E_{\bm{p}}}a_{\bm{p}}= U(\Lambda_p)\sqrt{2E_{\bm{p}}}a_{\bm{p}}=\sqrt{2E_{\Lambda^{-1}_p\bm{p}}}a_{\Lambda^{-1}_p\bm{p}}$~(������ʾ~$M=I$, �Ա���~$\phi'(x)=U(\Lambda)\phi(x)=\phi(\Lambda^{-1}x)$); ������~$a'_{\bm{p}}=a_{\bm{p}}=\frac{\sqrt{2E_{\Lambda^{-1}_p\bm{p}}}}{\sqrt{2E'_{\bm{p}}}}a_{\Lambda^{-1}_p\bm{p}}=\frac{\sqrt{2E_{\Lambda^{-1}_p\bm{p}}}}{\sqrt{2E_{\bm{p}}}}a_{\Lambda^{-1}_p\bm{p}}$, �Ӷ�~$a_{\bm{p}}\rightarrow a'_{\bm{p}}=a_{\bm{p}}=U(\Lambda_p)a_{\bm{p}}=\frac{\sqrt{2E_{\Lambda^{-1}_p\bm{p}}}}{\sqrt{2E'_{\bm{p}}}}a_{\Lambda^{-1}_p\bm{p}}=\frac{\sqrt{2E_{\Lambda^{-1}_p\bm{p}}}}{\sqrt{2E_{\bm{p}}}}a_{\Lambda^{-1}_p\bm{p}}$. ע���һ���ﲢ������ʲô��ʾ, ��ֻ��ʾһ�ֱ仯��ϵ; ��ͬ���ȵ���~boost �µ���Ϊ����; ���Բ�����~$A'^\mu(x)=U(\Lambda)A^\mu(x)={\Lambda^\mu}_\nu A^\nu(\Lambda^{-1}x)$ ��һ�Ƚ����.}
%
%��Ʋ�����º�����ij��, ���������Ϻ�����ij��
%$\sqrt{2E_{\bm{p}}}a_{\bm{p}}\rightarrow U(\Lambda_p)\sqrt{2E_{\bm{p}}}a_{\bm{p}}=\sqrt{2E_{\Lambda_p\bm{p}}}a_{\Lambda_p\bm{p}}=\sqrt{2E_{\bm{p}}}a_{\bm{p}}$, �Ӷ�~$a_{\bm{p}}=\frac{\sqrt{2E_{\Lambda_p\bm{p}}}}{\sqrt{2E_{\bm{p}}}}a_{\Lambda_p\bm{p}}$, $a_{\bm{p}}\rightarrow a_{\Lambda_p\bm{p}}=U(\Lambda_p)a_{\bm{p}}=\frac{\sqrt{2E_{\bm{p}}}}{\sqrt{2E_{\Lambda_p\bm{p}}}}a_{\bm{p}}$
%
%����DZ���, �����ڱ�������.
%
%������Ϊʲô���б���, ʸ����? ��ΪȺ��ƽӹ��ʾ, ������ʾ.
%
����, ��������̬
\begin{align}
|p^\rangle:=|p^\mu\rangle:=\sqrt{2E_{\bm{p}}}a^\dag_{\bm{p}}|0\rangle=\sqrt{2E_{\bm{p}}}|\bm{p}\rangle
\end{align}
���������Ȳ����; ���ɵ����һ��Ϊ~$\langle p|p'\rangle=2E_{\bm{p}}(2\pi)^3\delta^3(\bm{p}-\bm{p}')$. ��������֪ʶ, ���ǾͿɼ����
\begin{gather}
\langle p|x\rangle=e^{ipx},~\langle x|p\rangle=e^{-ipx};
\end{gather}
���ǾͿɵõ�
\begin{gather}
I=\int\frac{d^3\bm{p}}{(2\pi)^32E_{\bm{p}}}|p\rangle\langle p|,~I=2E_{\bm{p}}\int d^3\bm{x}|x\rangle\langle x|.
\end{gather}
%����������������Dz�����.
����ѡ����ʱ��~$\langle \bm{p}|\bm{x}\rangle=e^{-i\bm{p}\cdot\bm{x}},~\langle \bm{x}|\bm{p}\rangle=e^{i\bm{p}\cdot\bm{x}}$, �Լ�~$\langle\bm{x}|\bm{y}\rangle=\delta^3(\bm{x}-\bm{y})$, �򻹿ɼ����
\begin{gather}
I=\int\frac{d^3\bm{p}}{(2\pi)^3}|\bm{p}\rangle\langle\bm{p}|,~I=\int d^3\bm{x}|\bm{x}\rangle\langle\bm{x}|.
\end{gather}
���ƸղŸ�����~$|p\rangle$ ��~$|\bm{p}\rangle$ �Ĺ�ϵ, ����Ҳ�ɸ���~$\sqrt{2E_{\bm{p}}}|x\rangle\leftrightarrow|\bm{x}\rangle$.


Ӧ��ǰһ�����Ǹ����Ķ����ռ䳡����Ķ��׹�ϵ, ���Ѽ��������ռ�:
\begin{align}
[\phi(x),\phi(y)]&=\int\frac{d^3\bm{p}}{(2\pi)^32E_{\bm{p}}}\left[e^{-ip(x-y)}-e^{-ip(y-x)}\right]\nonumber\\
&=D(x-y)-D(y-x),\\
[\pi(x),\pi(y)]&=\int\frac{d^3\bm{p}}{(2\pi)^3}\frac{E_{\bm{p}}}{2}\left[e^{-ip(x-y)}-e^{-ip(y-x)}\right],\\
[\phi(x),\pi(y)]&=\frac{i}{2}\int\frac{d^3\bm{p}}{(2\pi)^3}\left[e^{-ip(x-y)}+e^{-ip(y-x)}\right];
\end{align}
������һ�����׹�ϵ, ���ΪЭ����׹�ϵ, �������ȱ���; ����������. ��Ӧ�ñ��ڸ�������ά����ά���˱�����, ���Ǽ���������ʽ�Ӷ�����ʱ���׹�ϵ
\begin{gather}
[\phi(\bm{x}),\phi(\bm{y})]=0,~[\pi(\bm{x}),\pi(\bm{y})]=0,\\
[\phi(\bm{x}),\pi(\bm{y})]=i\delta^3(\bm{x}-\bm{y}).
\end{gather}

���, ��ʽ~(\ref{diejia}) ���ֳ����⺭����д����
\begin{align}
a_{\bm{p}}&=\sqrt{2E_{\bm{p}}}\int d^3\bm{x}e^{ipx}\frac{1}{2}\left[\phi(x)+\frac{i\pi(x)}{E_{\bm{p}}}\right],\\
a^\dag_{\bm{p}}&=\sqrt{2E_{\bm{p}}}\int d^3\bm{x}e^{-ipx}\frac{1}{2}\left[\phi(x)-\frac{i\pi(x)}{E_{\bm{p}}}\right];
\end{align}
�����������ռ䳡���ĺ�ʱ���׹�ϵ, ����ʽ���ǿ����Ƴ������ռ�Ķ���ʽ��.


\subsection{���������΢������ɶ����ӳ��۵�Ҫ��: ���������������/�Գ���/ͳ�ƵĹ�ϵ}

~~~~
��������۱���, ��ʱ��������������¼�, ���Խ����ź�����; ����ռ�����¼��Dz����Ե�, ���߽�ʧȥȷ�����Ⱥ����. ���Ϊ������������綨��΢������������. ��ӳ�����ӳ�����, �����Ҫ��, ��ռ�������ϵ���ѧ�����, �����ǻ�����׵�, ��
\begin{align}
[\mathcal{O}(x),\mathcal{O}(y)]=0,~\textrm{when}~(x-y)^2<0.
\end{align}
���׿���, ��������ӳ�������ѧ������ij������, ����Ҫ��, �������ȡ�Ķ��׹�ϵ����, Ҫʹ������ռ����, ���IJ�ȡͬ�����׹�ϵ��Э������ʽ���Ϊ��. ����һС�ڵļ���������֪~$[\phi(x),\phi(y)]=D(x-y)-D(y-x)$; ����ռ����, �˶���ʽ��ֵ��ȷ�ǵ������. ͬʱҲ�ɿ���, ��������ȡΪ������ʽ, ����ռ�������ǵķ����׽����Ȼ��Ϊ��. �ǹ�����Ϊ��ı�������ȡ���׹�ϵ. ͬ��, ����һ�µ����˳�������Ҳ������, �Ծ���~$1/2$ ����������, ����ռ����, ���ȡ������ʽ�ij���Э������ʽ���Ϊ��. �ǹ�����Ϊ~$1/2$ �����ӱ�ȡ�����׹�ϵ.

��һ��, ���ǽ�����, ���������������ɵ�Ҫ��, ������������������, ���뱻������׹�ϵ; ���Ӳ����������ӽ����ǶԳƵ�, ����ϵ���㲣ɫ-����˹̹ͳ��; ���������ӳ�Ϊ��ɫ��. �����а���������������, ���뱻���跴���׹�ϵ; ���Ӳ����������ӽ����Ƿ��ԳƵ�, ����ϵ�������-������ͳ��; ���������ӳ�Ϊ������.




\subsection{�����Ӵ������ʷ�: ����������; ��ά������~$\delta$ ����}

~~~~ǰ��������֪��~$\langle x|y\rangle=D(x-y)$ ��ʾ���Ӵ�λ�ÿռ��~$y$ �㵽~$x$ ��Ĵ���; ���������Dz�û�мƼ�ʱ����Ⱥ�. ���ǽ�ʱ���Ⱥ�Ĵ�����ϵ~$D_F(x-y):=\langle0|T\phi(x)\phi(y)|0\rangle$, ��Ϊ����������. ����, $D_R(x-y):=\theta(x^0-y^0)\langle0|[\phi(x),\phi(y)]|0\rangle$ ��Ϊ�Ƴٸ��ֺ���, �ȵ�. ����, �����Է���������Ϊ��, �����н�һ������.
\begin{align}
&D_F(x-y)=\langle0|T\phi(x)\phi(y)|0\rangle\nonumber\\
=&\theta(x^0-y^0)\langle0|\phi(x)\phi(y)|0\rangle+\theta(y^0-x^0)\langle0|\phi(y)\phi(x)|0\rangle\nonumber\\
=&\theta(x^0-y^0)\int\frac{d^3\bm{p}}{(2\pi)^32E_{\bm{p}}}e^{-ip(x-y)}+\theta(y^0-x^0)\int\frac{d^3\bm{p}}{(2\pi)^32E_{\bm{p}}}e^{-ip(y-x)}\nonumber\\
=&\theta(x^0-y^0)\int\frac{d^3\bm{p}}{(2\pi)^32E_{\bm{p}}}e^{-ip(x-y)}\Big|_{p^0=E_{\bm{p}}}\nonumber\\
&~~~~~~~~~~~~~~~~~~~~~~~~~~~~~~~~-\theta(y^0-x^0)\int\frac{d^3\bm{p}}{(2\pi)^3(-2E_{\bm{p}})}e^{-ip(x-y)}\Big|_{p^0=-E_{\bm{p}}}\nonumber\\
=&\int\frac{d^3\bm{p}}{(2\pi)^3}\int\frac{dp^0}{2\pi i}\frac{-1}{p^2-m^2}e^{-ip(x-y)}~\textrm{with~Feynman~prescription}\nonumber\\
=&\int\frac{d^4p}{(2\pi)^4}\frac{i}{p^2-m^2}e^{-ip(x-y)}~\textrm{with~Feynman~prescription};
\end{align}
%����������ɫɢ��ϵ��, ���ַ���, û�а���ɫɢ��ϵ��; ��Ȼ, �������ֳ��������ս��, �õ������Ӵ�����ϵ, ������ɫɢ��ϵ��.
%�������ص�������ƽ��, ����ʵ�����, �ͳ����������.
\begin{figure}[!h]
\begin{center}
\includegraphics[width=7 cm]{pic/feynman.jpg}
\caption{��������: ���������ӻ�Χ����ѡȡ.}
\label{feynman}
\end{center}
\end{figure}
�����õ�����������~$\oint_l f(z)dz=2\pi i\sum_k f(a_k)$, ������Χ����ѡ���ʾ��ͼ~\ref{feynman} ��, ��Ϊ��������. ע�����������еĽ���������ζ�Ŷ���ά�����Ļ���, ����Ҫ����ά��������ɫɢ��ϵ, ������Ϊ���ֵ�������. ��Ȼ, ��~$\int\frac{d^4p}{(2\pi)^4}\frac{i}{p^2-m^2}e^{-ip(x-y)}$ ����Χ���IJ�ͬѡ��, ��������ͬ�Ĵ�����; ��ͼ~\ref{ret} �������Ƴٸ��ֺ���.
\begin{figure}[!h]
\begin{center}
\includegraphics[width=7 cm]{pic/ret.jpg}
\caption{�Ƴ��ƻ���Χ����ѡȡ.}
\label{ret}
\end{center}
\end{figure}

\noindent ���ڷ���������, һ�������ֻ�������¼Ƿ�:
\begin{align}
D_F(x-y)=\int\frac{d^4p}{(2\pi)^4}\frac{i}{p^2-m^2+i\epsilon}e^{-ip(x-y)},
\end{align}
���Ϊ~$\epsilon$ ����.

ע��, �����ӵı���ʽ, �Է���������Ϊ��, ��������~$D_F(p):=\frac{i}{p^2-m^2+i\epsilon}$ ����ά����Ҷ�任; ���к��߼��������ڶ����ռ�ı���ʽ. �Ѵ����Ӵ���~K-G ����, �ɵ�
\begin{align}
(\partial^2+m^2)D_F(x-y)=-i\int\frac{d^4p}{(2\pi)^4}e^{-ip(x-y)}:=-i\delta^4(x-y),
\end{align}
���������dz����̵ĸ��ֺ���. ����������, ͨ����Ӽ����~$D_F(x-y)=\int\frac{d^4p}{(2\pi)^4}D_F(p)e^{-ip(x-y)}$ ������֮, �������ֱ�ӵõ�~$D_F(p)$ ����~$D_F(x-y)$ �ı���ʽ. --����ʽ��, ���Ƕ�����λ�ÿռ���ά������~$\delta$ ����
\begin{align}
\delta^4(x-y)=&\delta(x^0-y^0)\delta^3(\bm{x}-\bm{y})=\delta(x^0-y^0)\frac{1}{(2\pi)^3}\langle\bm{x}|\bm{y}\rangle\nonumber\\
=&\int\frac{d^4p}{(2\pi)^4}e^{-ip(x-y)}.
\end{align}
ע��, ����ǰ���Ѿ�˵��������, ��������ά�����Ļ���, ��Ҫ��ɫɢ��ϵ, ������Ȼ�ֹ��ɻش�������. ��Ϊ�Ա�, �����Ƕ���~$\langle p|p'\rangle=2E_{\bm{p}}\langle\bm{p}|\bm{p}'\rangle=2E_{\bm{p}}(2\pi)^3\delta^3(\bm{p}-\bm{p}'):=(2\pi)^4\delta^4(p-p')$, ��ȡ~$I=\int d^4x|x\rangle\langle x|$, �Ϳɵõ���ά�����ռ�ĵ�����~$\delta$ ��������:
\begin{align}
\delta^4(p-p')=\int\frac{d^4x}{(2\pi)^4}e^{-i(p-p')x}.
\end{align}


%���ֺ����Ļ���, ��������ɫɢ��ϵ, ����, ����˴�������. Ҳ����˵�������������.

���Ժ���໥���ó�����, ��ij�����̵�ij�׽���, ����ͨ���Ὣ����Ӧ�Ľ�������ʽ��ͼ�εķ�������. ��Щͼ��, ��Ϊ����ͼ; ͼ���������ʽ֮��Ķ�Ӧ����, ��Ϊ��������. ����, ��򵥵ķ���ͼ, ���������������һ���߶α�ʾ����������:
\begin{align}\label{green f}
D_F(x-y)=\begin{aligned}\includegraphics[width=1.3 cm]{pic/pp.jpg}\end{aligned}~.
\end{align}


%���ڱ�������, ǡǡ��Ϊ���Ǵ˴���~$p^2=m^2$, ���������ڿǵ�, ���Dz�֪�����Ĵ������������ǿɱ�̽������ʵ������. ��Ϊ�Ա�, ���໥���ó�����, �������Ӳ��ڿ�, �������, �Dz���̽���.

%���������������ɳ����еó�������Ҫ�Ľ��֮һ, �������պ���໥���ó����е�ɢ������ļ������ֱ�ӵĹ�ϵ. ����, �����ɳ����������Ӿ���������ʽ��~$\langle\Omega|T\phi_H(x)\phi_H(y)|\Omega\rangle$--��Ϊ������������ֺ���--Ҳ����������ĵ�������֮һ. ��ʱ����Ҫ����, ������ΰѹ������������ɳ����еķ�����������ϵ����������.


%�������Ƿ��漰�������?



\subsection{��������: �������������к�, $\pi^\pm$ ����}

~~~~������֪, ������ǰ���ֽ�����ʵ������, ��������~$\pi^0$ ����. ����һ���������ɵĽ���, ��Ϊ~$\pi^\pm$; ��ǰ������֪, �������ǵļ�Ϊ��������. ���������������ܶ�, �Լ���֮���µij����˶�����, �������Ľ�Ϊ:
\begin{gather}
\mathcal{L}=\partial_\mu\phi^\dag\partial^\mu\phi-m^2\phi^\dag\phi,\\
(\partial^2+m^2)\phi=0,~(\partial^2+m^2)\phi^\dag=0;\\
\phi(x)=\int\frac{d^3\bm{p}}{(2\pi)^3\sqrt{2E_{\bm{p}}}}\left(a_{\bm{p}}e^{-ipx}+b^\dag_{\bm{p}}e^{ipx}\right),\\
\phi^\dag(x)=\int\frac{d^3\bm{p}}{(2\pi)^3\sqrt{2E_{\bm{p}}}}\left(a^\dag_{\bm{p}}e^{ipx}+b_{\bm{p}}e^{-ipx}\right);
\end{gather}
����~$a^\dag_{\bm{p}}$ ��~$b^\dag_{\bm{p}}$ ���������ֱ��������ɵ���������. �������������Լ������ܶȾ���
\begin{gather}
\pi=\frac{\partial\mathcal{L}}{\partial\dot{\phi}}=\dot{\phi}^\dag,~\pi^\dag=\frac{\partial\mathcal{L}}{\partial\dot{\phi}^\dag}=\dot{\phi};\\
\mathcal{H}=\pi\dot{\phi}+\pi^\dag\dot{\phi}^\dag-\mathcal{L}.
\end{gather}
���豾�����¶��׹�ϵ, ����ʵ����ϵ�����ӻ�:
\begin{gather}
[a_{\bm{p}},a^\dag_{\bm{p}'}]=(2\pi)^3\delta^3(\bm{p}-\bm{p}'),~[b_{\bm{p}},b^\dag_{\bm{p}'}]=(2\pi)^3\delta^3(\bm{p}-\bm{p}');\\
\Leftrightarrow~[\phi(\bm{x}),\pi(\bm{y})]=i\delta^3(\bm{x}-\bm{y}),~[\phi^\dag(\bm{x}),\pi^\dag(\bm{y})]=i\delta^3(\bm{x}-\bm{y});
\end{gather}
�����Ķ��׽�Ϊ��. ����Э����׹�ϵ�����׶���Ϊ~
\begin{gather}
[\phi(x),\phi(y)]=0,~[\phi^\dag(x),\phi^\dag(y)]=0,\\
[\phi(x),\phi^\dag(y)]=D_a(x-y)-D_b(y-x).
\end{gather}
���������Ĵ󲿷���ѧ������ʵ���������Ƕ�Ӧһ�µ�; ������Ҫ�о��������������Ĺ淶��. ����������Ӧ������淶������غ����ܶ�Ϊ
\begin{align}
j^\mu_{(q)}=&\frac{q}{i}\left(\frac{\partial\mathcal{L}}{\partial\partial_\mu\phi}\phi-\phi^\dag\frac{\partial\mathcal{L}}{\partial\partial_\mu\phi^\dag}\right)\nonumber\\
=&iq(\phi^\dag\partial^\mu\phi-\phi\partial_\mu\phi^\dag);
\end{align}
���������ڷ�����ѧ�����Ǿ����������. �������ǿ���������������غ��Ϊ
\begin{align}
Q=&\int d^3\bm{x}j^0_{(q)}=iq\int d^3\bm{x}(\phi^\dag\partial^0\phi-\phi\partial_0\phi^\dag)\nonumber\\
=&q\int\frac{d^3\bm{p}}{(2\pi)^3}(a^\dag_{\bm{p}}a_{\bm{p}}-b^\dag_{\bm{p}}b_{\bm{p}})
\end{align}
�ɴ�ȷ��, $a^\dag_{\bm{p}}$ ��~$b^\dag_{\bm{p}}$ ���������Ӻ��Ե�ȷ�෴.





\newpage

\section{��������: 1/2-�����������к�, �����˷�����}
\subsection{�����˷��̼��乲���, ����-�����˱���, ������̵������ܶ�, ��Լ���ɷ�����, ��ʸ��}
~~~~ǰ��Ⱥ��һ����, �������ϸ񵼳���������ķ���~$(\gamma^\mu p_\mu-m)\psi=0$. ������~$p_\mu=i\partial_\mu$, ��ɵõ����ӷ���
\begin{align}
(i\gamma^\mu\partial_\mu-m)\psi=0;
\end{align}
��Ϊ�����˷���. �ܵ��µ����˷��̵������ܶ�Ϊ
\begin{align}
\mathcal{L}=\bar{\psi}(i\gamma^\mu\partial_\mu-m)\psi;
\end{align}
%ǡ��ǰ�ķ���, �������ܶ���ֵΪ�㴦, ȡ���˶�����; ��Ի
�ɼ���ʵ�˶�ʹ���������ܶ�Ϊ��. �����~$\bar{\psi}$ Ӧ���������շ��̼��ɵ�ǰ�������˷���; �����~$\psi$ Ӧ�����Ϸ���, �ɵ�
\begin{align}
i\partial_\mu\bar{\psi}\gamma^\mu+m\bar{\psi}=0;
\end{align}
�˼������˷��̵Ĺ����. ��Ȼ, ��ʽҲ���ɵ����˷������ݻ����. ��������˵��, �κγ�����Ҫ����~Klein-Gordon ���̵�. �Ե����˷���������֤����~$-(i\gamma^\nu\partial_\nu+m)(i\gamma^\mu\partial_\mu-m)\psi=(\gamma^\mu\gamma^\nu\partial_\mu\partial_\nu+m^2)\psi=(\partial^2+m^2)\psi=0$.

�����Ѿ�֤���˵����˷��̶���ɢ�任�ǶԳƵ�; ����, ������֤�������˷��������������ȱ任�µIJ�����. ��ʵ��, ԭ���, һ����ֻҪ������������ʸ������, ����Ƕ������ȱ任����ı���. �����������ǻ��Ǿ�����һ��. ����ϵ������ϵ����һ��������ת��, ����
\begin{align}
(i\gamma^\mu\partial_\mu-m)\psi=0&\rightarrow\left(i\Lambda_{\frac{1}{2}}\gamma^\mu\Lambda^{-1}_{\frac{1}{2}}{\Lambda_\mu}^\nu\partial_\nu-m\right)\Lambda_{\frac{1}{2}}\psi(\Lambda^{-1}x)\nonumber\\
&=\left[i\Lambda_{\frac{1}{2}}\gamma^\mu\Lambda^{-1}_{\frac{1}{2}}\partial_\mu-m\right]\Lambda_{\frac{1}{2}}\psi(\Lambda^{-1}x)\nonumber\\
&=\Lambda_{\frac{1}{2}}(i\gamma^\mu\partial_\mu-m)\psi(\Lambda^{-1}x)=0;
\end{align}%���ԭ���ķ��̳���, �����ǹ۲�ϵ��������תһ�º�, ��ϵ���������ķ���.
�ɴ˿ɼ�, �����������ȱ任��, �����˷��̵�ȷ�Dz����.

����, ��Ⱥ��һ�����Ƶ�ʱ, $\psi$ �����·����Ǿ���ȷ����������������; ��Ӧ�õ���~$\gamma^\mu$ �򷽺͵ľ�����ʽ, ���������������������µ�. ��ѧ������֪��, ÿһ������~$\gamma^\mu$ �Ĵ����ľ���, ���ǵ����˷��̵�һ����ʾ. ��ȡ~$\gamma^0=\left[\begin{array}{cc}1&0\\0&-1\end{array}\right]$, ��ȡ~$\gamma^i$ ����ʽ����������е���ͬ, ���ѷ���, ������һ�����, ��������~$\gamma^\mu$ �Ĵ�����. �����˱�ʾ�ı���, ���dz�Ϊ����-�����˱���. ��ʱ, �����˷��̵ľ�����ʽ����
\begin{align}
\left[\begin{array}{cc}
E-m&-\bm{\sigma}\cdot\bm{p}\\
\bm{\sigma}\cdot\bm{p}&-E-m
\end{array}\right]
\left[\begin{array}{c}\varphi\\\chi\end{array}\right]=0.
\end{align}
������֪, ����������Ϊ��ʱ, �����˷����˻�Ϊ�����������, �����������˻�Ϊ�����ֱ�����������̬���������. Ҳ����˵, �����ӵĶ���Զ���ھ�����ʱ, ������������Ƿ����. ͬ���ɿ���, �����Ӷ��ܽ�С, ����ԶС�ھ�����ʱ, ���õ����˱����Ƿ����.

�����������������������ʽ����������˳������ܶ�\footnote{���������൱����~$\psi=\left[\begin{array}{c}\psi_L\\\psi_R\end{array}\right]$. ��������~$\psi=\left[\begin{array}{c}L\\R\end{array}\right]$, ��~$\psi_L=\left[\begin{array}{c}L\\0\end{array}\right],~\psi_R=\left[\begin{array}{c}0\\R\end{array}\right]$, ��~$\psi=\psi_L+\psi_R,~\bar{\psi}\psi=R^\dag L+L^\dag R=\bar{\psi}_R\psi_L+\bar{\psi}_L\psi_R$; ��~$\bar{\psi}\gamma^\mu\psi=L^\dag\bar{\sigma}^\mu L+R^\dag\sigma^\mu R=\bar{\psi}_L\gamma^\mu\psi_L+\bar{\psi}_R\gamma^\mu\psi_R$.}:
\begin{align}
\mathcal{L}=\psi^\dag_L i\bar{\sigma}^\mu\partial_\mu\psi_L+\psi^\dag_R i\sigma^\mu\partial_\mu\psi_R-m(\psi_R^\dag\psi_L+\psi^\dag_L\psi_R).
\end{align}
��Ȼ, �������������������������. ������Ϊ��ʱ, �����˳������ܶ��˻�Ϊ�������������, ���߼������ֱ����������������˶�����������.

���ǿ��Խ������˷���дΪ~$i\gamma^0\partial_0\psi=(-i\gamma^i\partial_i+m)\psi$, �Ӷ�~$i\partial_0\psi=(-i\gamma^0\gamma^i\partial_i+\gamma^0m)\psi$; ����~$\gamma^0=\beta,~\alpha^i=\gamma^0\gamma^i$, ��͵�
\begin{align}
i\partial_0\psi=(-i\alpha^i\partial_i+\beta m)\psi;
\end{align}
�˼�����������д�µ���ʽ.







������̸һ����Լ���ɷ�����. %�����������Ǹ���, �����˱�ʾ��������Ⱥ��һ������ʾ; ��Դ��������Ϊ~$\gamma$ ����������~Clifford ������ѡ������Ӧ����.
������Ϊ~Clifford ����ѡȡ������ʾ����: $\gamma^0=
\left[\begin{array}{cc}
0&i\sigma^2\\
i\sigma^2&0
\end{array}\right],~
\gamma^1=
\left[\begin{array}{cc}
i\sigma^3&0\\
0&i\sigma^3
\end{array}\right],~
\gamma^2=
\left[\begin{array}{cc}
0&-\sigma^2\\
\sigma^2&0
\end{array}\right],~
\gamma^3=
\left[\begin{array}{cc}
-i\sigma^1&0\\
0&-i\sigma^1
\end{array}\right],$
��Ϊ��Լ���ɻ�, ��ɷ������ǵõ���������Ⱥ��һ��ʵ������ʾ; ��Ӧ�ı�ʾ�������ͳ�Ϊ��ά��������. ��Ȼ, ��ά���������ĺɹ���, ��~$C$ �任�����������~$\psi=\left[\begin{array}{c}\psi_L\\i\sigma^2\psi^*_L\end{array}\right]$. ���Ƿ�������������������, Ϊ��Լ���ɷ�����.

��֮, ������������������������, 1/2-�����������к�, �����, ��˵�, ��Ϊ�����˷�����; ������������������������, 1/2-�����������к�, ��Ϊ���������; ������ʵ����������������, 1/2-�����������޺�, ��Ϊ��Լ���ɷ�����.

���, �����ó�������һЩ��ѧ��. �����Ĺ���������������Ϊ
\begin{gather}
\pi=\frac{\partial\mathcal{L}}{\partial\dot{\psi}}=i\bar{\psi}\gamma^0=i\psi^\dag,\\
\mathcal{H}=\pi\dot{\psi}-\mathcal{L}=i\psi^\dag\dot{\psi}-\bar{\psi}(i\gamma^\mu\partial_\mu-m)\psi=\bar{\psi}(-i\gamma^i\partial_i+m)\psi.
\end{gather}
�ڷ�����ѧ��, ������֪�����Ķ�Ӧ����Ϊ�ڲ��Գ��ԵĹ淶�任���غ���Ϊ
\begin{align}
j^\mu_{(q)}=\frac{q}{i}\left(\frac{\partial\mathcal{L}}{\partial\partial_\mu\phi}\phi-\phi^\dag\frac{\partial\mathcal{L}}{\partial\partial_\mu\phi^\dag}\right)=q\bar{\psi}\gamma^\mu\psi;
\end{align}
�Ӷ����������غ�ɼ�~$Q=q\int d^3\bm{x}\bar{\psi}\gamma^0\psi=q\int d^3\bm{x}\psi^\dag\psi$. ������������������������ְ��෴�ķ������淶�任, ��~$\psi\rightarrow e^{iq_H\alpha\gamma^5}\psi,~\bar{\psi}\rightarrow\bar{\psi}e^{iq_H\alpha\gamma^5}$, --���ֳ�Ϊ�����任, --�����Ƶ�, ���ǿ����ô˶�Ӧ�ڴ˵���Ϊ
\begin{align}
j^\mu_{(q)A}=q_H\bar{\psi}\gamma^\mu\gamma^5\psi;
\end{align}
��֮Ϊ��ʸ��. ����ע��, ��Ϊ~$\partial_\mu(\bar{\psi}\gamma^\mu\gamma^5\psi)=2im\bar{\psi}\gamma^5\psi$, ��Ȼ����ֻ������Ϊ��ʱ�ų�Ϊ�غ���. ��ʵ��, �˼���������ӵ�����. �����Ҳ���, ��Ȼ��
\begin{align}
j^\mu_{(q)AL}=q_H\bar{\psi}\gamma^\mu P_L\psi,~j^\mu_{(q)AR}=q_H\bar{\psi}\gamma^\mu P_R\psi.
\end{align}





\subsection{�����˷��̵Ľ�, $u^s(p),~v^s(p)$, Jordan-wigner ���ӻ�: �����׹�ϵ�ĸ���, ����������}

~~~~���ǿ���ֱ�Ӷ���, ��Ϊ�ķ��������������ĵ����˷��̵Ľ���
\begin{align}
\psi(x)&=\int\frac{d^3\bm{p}}{(2\pi)^3\sqrt{E_{\bm{p}}}}\sum_s\left(a^s_{\bm{p}}u^s(p)e^{-ipx}+b^{s\dag}_{\bm{p}}v^s(p)e^{ipx}\right);\\
\bar{\psi}(x)&=\int\frac{d^3\bm{p}}{(2\pi)^3\sqrt{E_{\bm{p}}}}\sum_s\left(a^{s\dag}_{\bm{p}}\bar{u}^s(p)e^{ipx}+b^{s}_{\bm{p}}\bar{v}^s(p)e^{-ipx}\right).
\end{align}
��Ҫ֮��, ��Ҫ��������~$u^s(p),~v^s(p)$ ������. ����Ƶ����Ϊ��, ���Ƿ��̵���Ƶƽ�沨��~$u(p)e^{-ipx}:=(u_1,u_2)^Te^{-ipx}$ ���뷽��, ������������, ����
\begin{align}
\left[\begin{array}{cc}
-m&\sigma^\mu p_\mu\\
\bar{\sigma}^\mu p_\mu&-m
\end{array}\right]\left[\begin{array}{c}u_1\\u_2\end{array}\right]=0;
\end{align}
�ɴ�֪~$\frac{u_1}{u_2}=\frac{\sigma^\mu p_\mu}{m}=\frac{m}{\bar{\sigma}^\mu p_\mu}$. �ɴ�, Ϊ����ʽ�Գ�, ���Ǽ��ɽ�~$u(p)$ �Ľ��������; �����Ƶİ취Ҳ�ɽ�~$v(p)$ ������; ���ǵĽ��Ϊ
\begin{align}
u(p)=\left[\begin{array}{c}\sqrt{\sigma^\mu p_\mu}\xi\\\sqrt{\bar{\sigma}^\mu p_\mu}\xi\end{array}\right],~v(p)=\left[\begin{array}{c}\sqrt{\sigma^\mu p_\mu}\eta\\-\sqrt{\bar{\sigma}^\mu p_\mu}\eta\end{array}\right];
\end{align}
����~$\xi^{s\dag}\xi^r=\delta^{sr},~\eta^{s\dag}\eta^r=\delta^{sr};~s,r=1,2$. ע�⵱���Ӿ�ֹʱ, ������~$u(\bm{p}=0)=\sqrt{m}(\xi,\xi)^T,~v(\bm{p}=0)=\sqrt{m}(\eta,-\eta)^T$. ���ǽ�ȷ��, $\xi^s$ �����˵��ӵ�����״̬; ����˵, $\xi^1=(1,0)^T,~\xi^2=(0,1)^T$. ����, ���Ƿdz����׼����:
\begin{gather}
u^{s\dag}(p)u^r(p)=2E_{\bm{p}}\delta^{sr},~\bar{u}^{s}(p)u^r(p)=2m\delta^{sr};\\
v^{s\dag}(p)v^r(p)=2E_{\bm{p}}\delta^{sr},~\bar{v}^{s}(p)v^r(p)=-2m\delta^{sr};\\
\bar{u}^{s}(p)v^r(p)=\bar{v}^{s}(p)u^r(p)=0;\\
u^{s\dag}(\bm{p})v^r(-\bm{p})=v^{s\dag}(\bm{p})u^r(-\bm{p})=0;\\
\sum_{s=1}^2u^s(p)\bar{u}^s(p)=\gamma^\mu p_\mu+m,~\sum_{s=1}^2v^s(p)\bar{v}^s(p)=\gamma^\mu p_\mu-m.
\end{gather}
��Ȼ~$\bar{u}^{s}(p)u^r(p)$ ��~$\bar{v}^{s}(p)v^r(p)$~�������Ȳ����.



ǰ������½�ͨ�����ϵ������������, ������ɵ�����, �����ӱ��뱻���跴���׹�ϵ, --����Ϊ~Jordan-Wigner ���ӻ�; �ʱ�����
\begin{gather}
\{a^{s}_{\bm{p}},a^{r\dag}_{\bm{p}'}\}=(2\pi)^3\delta^3(\bm{p}-\bm{p}')\delta_{sr},\\
\Leftrightarrow~\{\psi_\alpha(\bm{s}),\psi^\dag_{\beta}(\bm{y})\}=\delta^3(\bm{x}-\bm{y})\delta_{\alpha\beta}.
\end{gather}
��ʵ��, ����Ҳ������, ֻ�б����跴���׹�ϵ, �����ӵ���ѧ�����ܵõ��������. ������ϵ��������Ϊ
\begin{align}
H=&\int d^3\bm{x}\bar{\psi}(-i\gamma^i\partial_i+m)\psi=\int d^3\bm{x}i\psi^\dag\dot{\psi}\nonumber\\
=&\int\frac{d^3\bm{p}}{(2\pi)^3}\sum_sE_{\bm{p}}(a^{s\dag}_{\bm{p}}a^{s}_{\bm{p}}+b^{s\dag}_{\bm{p}}b^{s\dag}_{\bm{p}}).
\end{align}
�����糡���ܵ��
\begin{align}
Q=q\int d^3\bm{x}\bar{\psi}\gamma^0\psi=q\int d^3\bm{x}\psi^\dag\psi=\int\frac{d^3\bm{p}}{(2\pi)^3}\sum_sq(a^{s\dag}_{\bm{p}}a^{s}_{\bm{p}}-b^{s\dag}_{\bm{p}}b^{s\dag}_{\bm{p}}).
\end{align}
������������~$S^k=\int d^3\bm{x}\pi\frac{-i}{2}\Sigma^k\psi=\int d^3\bm{x}\psi^\dag\frac{1}{2}\Sigma^k\psi$; �����ڶ���������չ��, ��ɷ���~$\xi=(1,0)^T,~(0,1)^T$ �ֱ��������~$1/2,~-1/2$; ��~$\eta=(1,0)^T,~(0,1)^T$ �ֱ��������~$-1/2,~1/2$. %�����, ����, ���Dz������������, ��ͨ����~$S^k$ ������

���ѽ���������´�����ϵ
\begin{gather}
\langle0|\psi_\alpha(x)\bar{\psi}_\beta(y)|0\rangle=(i\gamma_\mu\partial^\mu_x+m)_{\alpha\beta}D(x-y),\\
\langle0|\bar{\psi}_\beta(y)\psi_\alpha(x)|0\rangle=-(i\gamma_\mu\partial^\mu_x+m)_{\alpha\beta}D(y-x).
\end{gather}
���DZ�����Э�䷴���׹�ϵ����
\begin{align}
\{\psi_\alpha(x),\bar{\psi}_\beta(y)\}=(i\gamma_\mu\partial^\mu_x+m)_{\alpha\beta}\left[D(x-y)-D(y-x)\right].
\end{align}
�����ķ��������Ӿ���
\begin{align}
S_F:=&\langle0|T\psi(x)\bar{\psi}(y)|0\rangle=\left\{\begin{array}{cl}\langle0|\psi(x)\bar{\psi}(y)|0\rangle,&x^0>y^0\\
-\langle0|\bar{\psi}(y)\psi(x)|0\rangle,&x^0<y^0\end{array}\right.\\
=&\int\frac{d^4p}{(2\pi)^4}\frac{i(\gamma^\mu p_\mu+m)}{p^2-m^2+i\epsilon}e^{-ip(x-y)}.
\end{align}
���л���Χ����ѡȡ, ��ʵ������ʱ���������ͬ��. %ע��Է��������, ����λ�ý�����һ������.













\newpage


\section{ʵʸ����: 1-�����޺�������, ����}

~~~~��ų������˹Τ��������, ��ų�������, ������. �����˹Τ����~(��ʵ����ܽ�) ����, ��ų������, ��û�о�������.

����, �����˹Τ������ʽ���ɿ���, ��ų���ʸ����. ��ǰ��Ⱥ���еķ���, ����֪��ʸ����������Ϊ~1; ����ų����ӻ���, ���Ǹ���ȷ����һ��.

���, ����ɳ������, ����Ҳ�ɻ�֪��ų�������ʸ����, ������������. ��ʵ��, �淶��--��Ի�м䲣ɫ��, ����ʸ����. ���ǿ�����ǰָ��, �����õ��м䲣ɫ������������; �⽫���������м䲣ɫ�Ӿ����Գ�����ȱ����ʵ��.

���������ص㽲��ų������ӻ�.

%��ų�, �൱���Զ������һ�����ӻ��ij�; ��Ա�������ʵ������, ��Ѧ���̳�, K-G ��, ������˳�, �Ϳɷ���!


\subsection{��ų������ӻ�}

~~~~������֪���ɵ�ų��ķ���Ϊ
\begin{gather}
\partial_\mu F^{\mu\nu}=0,\\
\partial_\mu F_{\nu\rho}+\partial_\nu F_{\rho\mu}+\partial_\rho F_{\mu\nu}=0;
\end{gather}
���е�һʽ�Ƕ�����, �ڶ�ʽ��~Bianchi ��ʽ��. �������������ȱ���
\begin{align}%ע��!!!!!!!L ��, һֱ������ͬ���ڳ�, Ȼ�����.!!!!!!!!!!!!!!!!!!!!!!!!!!111
\mathcal{L}=&-\frac{1}{4}F_{\mu\nu}F^{\mu\nu}=-\frac{1}{4}(\partial_\mu A_\nu-\partial_\nu A_\mu)(\partial^\mu A^\nu-\partial^\nu A^\mu)\nonumber\\
=&-\frac{1}{2}(\partial_\mu A_\nu\partial^\mu A^\nu-\partial_\mu A_\nu\partial^\nu A^\mu)=-\frac{1}{2}\partial_\mu A_\nu F^{\mu\nu}.\label{laofma}
\end{align}
��~$A_\nu$ ����ŷ������, ���ɵ�~$\frac{\partial\mathcal{L}}{\partial A_\nu}-\partial_\mu\frac{\partial\mathcal{L}}{\partial\partial_\mu A_\nu}=0+\partial_\mu(\partial^\mu A^\nu-\partial^\nu A^\mu)=\partial_\mu F^{\mu\nu}=0$.
�ǹ�~(\ref{laofma}) ʽ���ܵ������˹Τ���̵ĵ�ų��������ܶ�. --������~$F_{\mu\nu}$ ����д�������ǻ��ɷ���~$\mathcal{L}=\frac{1}{2}(\frac{\bm{E}^2}{c^2}-\bm{B}^2)$. -- ��~$A_\mu$ ������������
\begin{align}
\pi^\mu=\frac{\partial\mathcal{L}}{\partial\dot{A}_\mu}=F^{\mu 0}=\partial^\mu A^0-\partial^0 A^\mu;
\end{align}
��~$\pi^0=0,~\pi^i=E^i$. ���Կ���, �����ڳ���ʱ�����~$A_0$ ��������Ϊ��. ��һ���ɵó��Ĺ��ܶ��ܶȾ���
\begin{align}
\mathcal{H}=&\pi^\mu \dot{A}_\mu-\mathcal{L}=\pi^i \dot{A}_i-\mathcal{L}=-\bm{E}\cdot\frac{\partial\bm{A}}{\partial t}-\mathcal{L}=\bm{E}\cdot(\bm{E}+\nabla\varphi)-\mathcal{L}\nonumber\\
=&\frac{1}{2}(\bm{E}^2+\bm{B}^2)+\bm{E}\cdot\nabla\varphi=\frac{1}{2}(\bm{E}^2+\bm{B}^2)-\varphi\nabla\cdot\bm{E};
\end{align}
�����Ѳ�����Ȼ��λ. �������һ������������ȥ����һ��ȫɢ����.

\subsubsection{����淶���ӻ�, ���׹�ϵ���޸�, ��ɢ~$\delta$ ����}
%��, ���׹�ϵ, ����Ҫ��������.

~~~~��ų�/�ƵĹ淶����, ��ζ�����ǿ��Ը���������������, ��Ϊ�淶����. ���ع淶��~$\nabla\cdot\bm{A}=0$, --���൱���ų��˵�ų������ռ����--��Ϊ��շ���--�е��������, �϶���ų��Ǻ᳡. �������ɳ�, $\rho=0$ , ���Ǿ͵�����һ����, ��~$\nabla\cdot\bm{E}=0$ ��ȼ۵�~$\varphi=A^0=0$; ���߿���ΪΪ��ų�����ʱ����/��������. ����������׹淶�����ϳ�Ϊ����淶.


��ʵ������������֪��, ��ų�ֻ�к�����������ɶ�, �������Ϊ��. ����淶��������, ǡ����ų����ɶ�����Ϊ���˶�����ֱ����������; ������ʵ��һ�µ�. ����, ��Ȼ����淶�����ǽ����ٱ������۵�������Э����. �������, �������ڴ˹淶�½�����ϵ�����ӻ�.

��Ӧ�ڵ�ų������������ɶ�, ����������ƫ����, ����ȡ��Ӧ��������λʸ��, ��Ϊ��λ����ʸ��~$\bm{\epsilon}_r(\bm{p}),~r=1,2$�� �������ȵ�Ȼ����~$\bm{\epsilon}_r(\bm{p})\cdot\bm{p}=0$; ����, ���ǵ�������һ����ϵ���걸�Թ�ϵ�ֱ�Ϊ
\begin{gather}
\bm{\epsilon}_r(\bm{p})\cdot\bm{\epsilon}_s(\bm{p})=\delta_{ij}\epsilon^i_r(\bm{p})\epsilon^j_s(\bm{p})=\delta_{rs},\\
\sum_{r=1}^2\epsilon^i_r(\bm{p})\epsilon^j_r(\bm{p})=\delta_{ij}-\frac{p_ip_j}{\bm{p}^2}.
\end{gather}
����~$i,j$ Ϊ����ʸ������������ָ��. �걸�Թ�ϵ���޸����ԭ��, �ڽ��������ῴ��.

����淶�µij�����Ϊ~$\partial_\mu F^{\mu\nu}=\partial_\mu(\partial^\mu A^\nu-\partial^\nu A^\mu)=\partial_\mu\partial^\mu A^\nu-\partial_\mu\partial^\nu A^\mu=\partial^2\bm{A}=0$, �ʿ�ȡ�ñ����̵Ľ⼰������Ϊ
\begin{align}
&\bm{A}(x)=\int\frac{d^3\bm{p}}{(2\pi)^3\sqrt{2E_{\bm{p}}}}\sum_{r=1}^2\bm{\epsilon}_r(\bm{p})\left(a^r_{\bm{p}}e^{-ipx}+a^{r\dag}_{\bm{p}}e^{ipx}\right),\\
&\bm{E}(x)=\int\frac{d^3\bm{p}}{(2\pi)^3}(-i)\sqrt{\frac{E_{\bm{p}}}{2}}\sum_{r=1}^2\bm{\epsilon}_r(\bm{p})\left(a^r_{\bm{p}}e^{-ipx}-a^{r\dag}_{\bm{p}}e^{ipx}\right).
\end{align}


һ���, ������֪������ʱ�����Ĺ������Ϊ��; ������淶��, ����ǡ�г�����ʱ������Ϊ��. ����, ����淶������ֻ�ÿ��ǿռ䲿��--$A^i,~\pi^i=-\partial_0 A^i$--�Ķ��׹�ϵ. ������һ��, ��Ϊ�����, ���Ǹ�����Ϊ����Ϊ~1 ��ʸ�����ı��������µ�ʱ���׹�ϵ, ��ǡ����:
\begin{align}
[A_i(\bm{x}),A_j(\bm{y})]=0,~[\pi^i(\bm{x}),\pi^j(\bm{y})]=0.
\end{align}
����~$[A_i(\bm{x}),\pi^j(\bm{y})]=ig^j_i\delta^3(\bm{x}-\bm{y})$ ȴ�Dz�������Ҫ���, ��Ϊ��ʽ����ɢ�ȹ���, ���~$\partial_i[A^i(\bm{x}),\pi_j(\bm{y})]=[\partial_iA^i(\bm{x}),\pi_j(\bm{y})]=0$, ���ұ�~$\partial_i ig^j_i\delta^3(\bm{x}-\bm{y})=\partial_iig^j_i\int\frac{d^3\bm{p}}{(2\pi)^3}e^{i\bm{p}\cdot(\bm{x}-\bm{y})}=-\int\frac{d^3\bm{p}}{(2\pi)^3}p_je^{i\bm{p}\cdot(\bm{x}-\bm{y})}\neq0$.
�������, ҽ������. ��ȡ
\begin{align}
[A_i(\bm{x}),\pi^j(\bm{y})]=&i\int\frac{d^3\bm{p}}{(2\pi)^3}\left(g^j_i-\frac{p_ip^j}{p_k p^k}\right)e^{i\bm{p}\cdot(\bm{x}-\bm{y})}\nonumber\\
=&i\left(g^j_i-\frac{\partial_i\partial^j}{\nabla^2}\right)\delta^3(\bm{x}-\bm{y}):=i\bar{\delta}^3(\bm{x}-\bm{y}),
\end{align}
�������������ɢ��ͬʱΪ���Ҫ����. ������ʱ�ְ�~$\bar{\delta}^3(\bm{x}-\bm{y})$ ��Ϊ��ɢ~$\delta$ ����. ͨ�����Ľ�, ���ѻ�֪����ʵ�ռ�Ķ��׹�ϵ�����¶����ռ��еĶ��׹�ϵ
\begin{align}
[a^r_{\bm{p}},a^{s\dag}_{\bm{p}'}]=(2\pi)^3\delta^{rs}\delta^3(\bm{p}-\bm{p}'),~[a^r_{\bm{p}},a^s_{\bm{p}'}]=[a^{r\dag}_{\bm{p}},a^{s\dag}_{\bm{p}'}]=0;
\end{align}
�ǻ��µ�. ��ʵ������ʱ�������ͬ, �����ռ䳡����ʵ�ռ䳡�����T
\begin{align}
a_{\bm{p}}^r&=(2\pi)^3\int d^3\bm{x}\varphi^*_{\bm{p}}(x)i\overset{\leftrightarrow}{\partial}_0\big[\bm{\epsilon}_r(\bm{p})\cdot\bm{A}(x)\big],\\
a^{r\dag}_{\bm{p}}&=-(2\pi)^3\int d^3\bm{x}\varphi_{\bm{p}}(x)i\overset{\leftrightarrow}{\partial}_0\big[\bm{\epsilon}_r(\bm{p})\cdot\bm{A}(x)\big].
\end{align}

����, ���Ǽ��ɵõ����������ӻ�����ѧ��, ������Ϊ\footnote{
���Ƕ�~$\int d^3\bm{x}(\nabla\times\bm{A})^2=-\int d^3\bm{x}\bm{A}\cdot\nabla^2\bm{A}$ �ļ�����һչʾ:
\begin{align}
&(\nabla\times\bm{A})^2=\sum_{j,k=1,2,3;even}(\partial_jA_k-\partial_kA_j)(\partial_jA_k-\partial_kA_j)\nonumber\\
=&\sum_{j,k=1,2,3;any}(\partial_jA_k)(\partial_jA_k-\partial_kA_j):=\partial_jA_k\partial_jA_k-\partial_jA_k\partial_kA_j\nonumber\\
=&\sum_{k=1,2,3}\sum_{j=1,2,3}\bigg\{\Big[\partial_j(A_k\partial_jA_k)-A_k\partial_j\partial_jA_k\Big]-\Big[\partial_j(A_k\partial_kA_j)-A_k\partial_j\partial_kA_j\Big]\bigg\}\nonumber\\
=&\sum_{j,k=1,2,3}\bigg\{\Big[\partial_j(A_k\partial_jA_k)-A_k\nabla^2A_k\Big]-\Big[\partial_j(A_k\partial_kA_j)-A_k\partial_k\nabla\cdot\bm{A}\Big]\bigg\}
\end{align}
ȫɢ�ȵĻ���Ϊ��, ���ǻ��ֺ���������������������ڵĵ�һ�����ʧ; ������淶��������~$\nabla\cdot\bm{A}=0$, �������յ�~$\int d^3\bm{x}(\nabla\times\bm{A})^2=-\int d^3\bm{x}\bm{A}\cdot\nabla^2\bm{A}$.
}
\begin{align}
H=&\int d^3\bm{x}\frac{1}{2}(\bm{E}^2+\bm{B}^2)=\int d^3\bm{x}\frac{1}{2}\left[\dot{\bm{A}}^2+(\nabla\times\bm{A})^2\right]\nonumber\\
=&\int d^3\bm{x}\frac{1}{2}\left[\dot{\bm{A}}^2-\bm{A}\cdot\nabla^2\bm{A}\right]=\int d^3\bm{x}\frac{1}{2}\left[\dot{\bm{A}}^2-\bm{A}\cdot\partial^2_0\bm{A}\right]\nonumber\\
=&\frac{1}{2}\int d^3\bm{x}i\bm{A}\cdot i\overset{\leftrightarrow}{\partial}_0\partial_0\bm{A}=\int\frac{d^3\bm{p}}{(2\pi)^3}E_{\bm{p}}\sum_{r=1}^2a^{r\dag}_{\bm{p}}a^r_{\bm{p}}.
\end{align}
��������������ȫ΢�����Լ�Ӧ���˳�����ķ���~$\partial^2\bm{A}=0$; ����õ���~$\bm{\epsilon}_r(\bm{p})\cdot\bm{\epsilon}_s(\bm{p})=\delta_{rs}$. ͬ��, ������������������Ӧ�ı���, ��д�����������ĺϳ�
\begin{align}
P^i&=\frac{1}{2}\int d^3\bm{x}i\bm{A}\cdot i\overset{\leftrightarrow}{\partial}_0\partial^i\bm{A},\\
P^\mu&=\frac{1}{2}\int d^3\bm{x}i\bm{A}\cdot i\overset{\leftrightarrow}{\partial}_0\partial^\mu\bm{A}.
\end{align}



����, ��������չʾһ�������Ļ��. ����, $\mathcal{S}^k=\varepsilon_{ijk}A^i\pi^j=-A^i\partial_0A^j\varepsilon_{ijk}=-A^i\partial_0A^j\varepsilon_{ijk}+A^j\partial_0 A^i\varepsilon_{ijk}-A^j\partial_0 A^i\varepsilon_{ijk}=A^j\overset{\leftrightarrow}{\partial}_0 A^i\varepsilon_{ijk}-\mathcal{S}^k$, ����
\begin{align}
S^k=&\int d^3\bm{x}\frac{1}{2}iA^i i\overset{\leftrightarrow}{\partial}_0 A^j\varepsilon_{ijk}\nonumber\\
=&\frac{i}{2}\int \frac{d^3\bm{p}}{(2\pi)^3}\epsilon_r^i(\bm{p})\epsilon_s^j(\bm{p})\varepsilon_{ijk}\left(a_{\bm{p}}^sa_{\bm{p}}^{r\dag}-a_{\bm{p}}^{s\dag}a_{\bm{p}}^r\right)\nonumber\\
=&\frac{i}{2}\int \frac{d^3\bm{p}}{(2\pi)^3}\left[(a_{\bm{p}}^{i\dag}a_{\bm{p}}^j+a_{\bm{p}}^ja_{\bm{p}}^{i\dag})-(a_{\bm{p}}^ia_{\bm{p}}^{j\dag}+a_{\bm{p}}^{j\dag}a_{\bm{p}}^i)\right]\nonumber\\
=&\frac{1}{2}\int \frac{d^3\bm{p}}{(2\pi)^3}\left[(a_{\bm{p}}^{+\dag}a_{\bm{p}}^+ +a_{\bm{p}}^+a_{\bm{p}}^{+\dag})-(a_{\bm{p}}^- a_{\bm{p}}^{-\dag}+a_{\bm{p}}^{-\dag}a_{\bm{p}}^-)\right];
\end{align}
����
\begin{align}
a_{\bm{p}}^\pm:=\frac{1}{\sqrt{2}}(a_{\bm{p}}^i\pm i a_{\bm{p}}^j),~a_{\bm{p}}^{\pm\dag}:=\frac{1}{\sqrt{2}}(a^{i\dag}_{\bm{p}}\mp i a_{\bm{p}}^{j\dag}),
\end{align}
�ֱ������������������ӵ����������. ��ǰ�����֪, �������Ӿ�������~1, �������Ӿ�������~$-1$.



%�������̹���, ƫ���򲻻��, ���д�.

���ڱ��淶�µ�Э����׹�ϵΪ
\begin{align}
[A_i(x),A_j(y)]=&\int\frac{d^3\bm{p}}{(2\pi)^32E_{\bm{p}}} \sum_{r=1}^2\epsilon^i_r(\bm{p})\epsilon^j_r(\bm{p}) \left[e^{-ip(x-y)}-e^{-ip(y-x)}\right]\nonumber\\
=&\int\frac{d^3\bm{p}}{(2\pi)^32E_{\bm{p}}}\left(\delta_{ij}-\frac{p_ip_j}{\bm{p}^2}\right)\left[e^{-ip(x-y)}-e^{-ip(y-x)}\right]\nonumber\\
=&\left(\delta_{ij}+\frac{\partial_i\partial_j}{\nabla^2}\right)\int\frac{d^3\bm{p}}{(2\pi)^32E_{\bm{p}}}\left[e^{-ip(x-y)}-e^{-ip(y-x)}\right]\nonumber\\
=&\left(\delta_{ij}+\frac{\partial_i\partial_j}{\nabla^2}\right)[\phi(x),\phi(y)];
\end{align}
���dz��ڱ��淶�µķ���������Ϊ
\begin{align}
D_{ij}(x-y)=\langle0|TA_i(x)A_j(y)|0\rangle=\int\frac{d^4p}{(2\pi)^4}\frac{i}{p^2+i\epsilon}\left(\delta_{ij}-\frac{p_ip_j}{\bm{p}^2}\right)e^{-ip(x-y)}.
\end{align}







\subsubsection{Э��~(���״Ĺ淶) ���ӻ�, �����ܶȵ��޸�, �����淶, Gupta-Bleuler ����, ����������������������, ��̬}

~~~~�������淶, �����״Ĺ淶~$\partial_\mu A^\mu=0$ ��, ���۽�����������Э����. �˹淶��, ���ǵõ������ɳ����˶�������~$\partial_\mu\partial^\mu A^\nu-\partial_\mu\partial^\nu A^\mu=0\rightarrow\partial_\mu\partial^\mu A^\nu-\partial^\nu\partial_\mu A^\mu=0\xrightarrow{\partial_\mu A^\mu=0}\partial^2 A^\mu=0$. ����, ���������ı�; ����ͨ���޸������ܶȵİ취, ��ֱ�ӵõ��������. --��Ȼ��Ҳ����ζ�Ŵ˰취�������״Ĺ淶�����ȼ۵�, ��Ի�˰취�������Ѳ���Ҫ������������~$\partial_\mu A^\mu=0$. --�����������ܶ�
\begin{align}
\mathcal{L}=-\frac{1}{4}F_{\mu\nu}F^{\mu\nu}-\frac{1}{2}(\partial_\mu A^\mu)^2=-\frac{1}{4}F_{\mu\nu}F^{\mu\nu}-\frac{1}{2}(g^{\nu\mu}\partial_\mu A_\nu)^2
\end{align}
��~$A_\nu$ ����ŷ������, �͵�~$\partial_\mu\partial^\mu A^\nu-\partial_\mu\partial^\nu A^\mu+\partial_\mu g^{\mu\nu}\partial_\rho A^\rho=\partial_\mu\partial^\mu A^\nu=0$. �ɴ˿�ȷ֪���������ܶȵ�ȷ�������״Ĺ淶�����ȼ۵ı���.

����, ���������������ø�һ�㻯һЩ, ���������ܶ�д��
\begin{align}
\mathcal{L}=-\frac{1}{4}F_{\mu\nu}F^{\mu\nu}-\frac{1}{2\xi}(\partial_\mu A^\mu)^2;
\end{align}
������һ��~$-\frac{1}{2\xi}(\partial_\mu A^\mu)^2$ �����淶�̶���; $\xi$ ��Ϊ�淶�̶�����; $\xi=1$ ��������������淶, $\xi=0$ �������Ϊ�ʵ��淶. ����ʽ�õ��ij����˶����̾���
\begin{align}
\partial_\mu F^{\mu\nu}+\frac{1}{\xi}\partial^\nu\partial_\rho A^\rho=\partial_\mu\partial^\mu A^\nu-(1-\frac{1}{\xi})\partial^\nu\partial_\rho A^\rho=0.
\end{align}
��Ȼ, ��ȡ�����淶ʱ, ���ǻص����״Ĺ淶�µij�����.

���������ܶȳ����ڻ�����, �����󳡵�����ѧ��ʱ, �����Ҳ��������, �����ǰ������ܶ��е���άɢ����ȥ��, �ǿ��Ե�. �����Ժ�������ܶȿɽ�һ��дΪ
\begin{align}
\mathcal{L}=&-\frac{1}{4}F_{\mu\nu}F^{\mu\nu}-\frac{1}{2}(\partial_\mu A^\mu)^2\nonumber\\%=-\frac{1}{2}(\partial_\mu A_\nu\partial^\mu A^\nu-\partial_\mu A_\nu\partial^\nu A^\mu)-\frac{1}{2}(\partial_\mu A^\mu)^2\nonumber\\
%=&-\frac{1}{2}\partial_\mu A_\nu\partial^\mu A^\nu+\frac{1}{2}\partial_\mu (A_\nu\partial^\nu A^\mu)-\frac{1}{2}A_\nu\partial_\mu\partial^\nu A^\mu-\frac{1}{2}\partial_\mu(A^\mu\partial_\nu A^\nu)+\frac{1}{2}A^\mu\partial_\mu\partial_\nu A^\nu\nonumber\\
=&-\frac{1}{2}\partial_\mu A_\nu\partial^\mu A^\nu.
\end{align}
������֤����ʽ��ȷ�ǿ����ó����״Ĺ淶�µij����̵�.

�ɴ�, ���ǿ���д���ڷ����淶��, �볡�������������Ϊ
\begin{align}
\pi^0=&\frac{\partial\mathcal{L}}{\partial\dot{A}_0}=F^{00}-\partial_\mu A^\mu=-\partial_\mu A^\mu=\partial^iA^i-\partial^0A^0,\\
\pi^i=&\frac{\partial\mathcal{L}}{\partial\dot{A}_i}=F^{i0}-0=\partial^i A^0-\partial^0 A^i=E^i;
\end{align}
�����������ܶ���ȥ����άɢ�����, ��������������
\begin{align}
\pi^\mu=\frac{\partial\mathcal{L}}{\partial\dot{A}_\mu}=-\dot{A}^\mu.
\end{align}
�Ƚ���ʽ����δȥ����άɢ����������ܶȵó��Ķ����ɵ�, �������ܶ���ȥ����άɢ����, �൱���ڹ������ȥ���ռ���άɢ����.

��Ϊ���ڳ���~$A^\mu$ ��~4 ������, ��������Ҫ����~4 ����������ά����ʸ��~$\epsilon^\lambda_\mu(\bm{p}),~\lambda=0,1,2,3$. ���������������һ����ϵ���걸��ϵ����
\begin{gather}
g^{\mu\nu}\epsilon^\lambda_\nu(\bm{p})\epsilon^{\lambda'}_\mu(\bm{p})=g^{\lambda\lambda'},\\
g_{\lambda\lambda'}\epsilon^\lambda_\mu(\bm{p})\epsilon^{\lambda'}_\nu(\bm{p})=g_{\mu\nu}.
\end{gather}
����ȡ��ʸ�ص�����ʱ, $p^\mu=k^\mu=(\omega,0,0,k)$; ��Ϊ��ӳ��Ų�������������������ʵ, ����Ӧ��~$g^{\mu\nu}p_\mu\epsilon^1_\nu=0,~g^{\mu\nu}p_\mu\epsilon^2_\nu=0$; ���Ǵ�ʱ~4 ����ά����ʸ��ѡΪ����~$\epsilon^0=(1,0,0,0)^T,~\epsilon^1=(0,1,0,0)^T,~\epsilon^2=(0,0,1,0)^T,~\epsilon^3=(0,0,0,1)^T$.

��������д�������Ľ�
\begin{align}
&A_\mu(x)=\int\frac{d^3\bm{p}}{(2\pi)^3\sqrt{2E_{\bm{p}}}}\sum_{\lambda=0}^3\epsilon^\lambda_\mu(\bm{p})\left(a^\lambda_{\bm{p}}e^{-ipx}+a^{\lambda\dag}_{\bm{p}}e^{ipx}\right),\\
&\pi^\mu(x)=\int\frac{d^3\bm{p}}{(2\pi)^3}i\sqrt{\frac{E_{\bm{p}}}{2}}\sum_{\lambda=0}^3(\epsilon^\mu)^\lambda(\bm{p})\left(a^\lambda_{\bm{p}}e^{-ipx}-a^{\lambda\dag}_{\bm{p}}e^{ipx}\right).
\end{align}
�������������µ�ʱ���׹�ϵ��ǡ����:
\begin{gather}
[A_\mu(\bm{x}),A_\nu(\bm{y})]=[\pi^\mu(\bm{x}),\pi^\nu(\bm{y})]=0,\\
[A_\mu(\bm{x}),\pi_\nu(\bm{y})]=ig_{\mu\nu}\delta^3(\bm{x}-\bm{y});
\end{gather}
��Ӧ�Ķ����ռ��еĶ��׹�ϵ����:
\begin{gather}
[a_{\bm{p}}^\lambda,a_{\bm{p}'}^{\lambda'}]=[a_{\bm{p}}^{\lambda\dag},a_{\bm{p}'}^{\lambda'\dag}]=0,\\
[a_{\bm{p}}^\lambda,a_{\bm{p}'}^{\lambda'\dag}]=-g_{\lambda\lambda'}(2\pi)^3\delta^3(\bm{p}-\bm{p}').
\end{gather}
���������һʽ���ѿ���, �������ӵĹ�һ���Ǹ���: $\langle0|a^{0\dag}_{\bm{p}}a^0_{\bm{p}'}|0\rangle=-(2\pi)^3\delta^3(\bm{p}-\bm{p}')$, �ǹ�̬. --���и���һ����̬, �Ƿ�������, ���dz�Ϊ��̬.

���ھͿ��ó����ĸ�����ѧ����; �����ڴ�֮ǰ, ��������ԭ�ȵ�������������һ����, �����������̬������



�����Ѿ�˵��, �ڷ����淶��, �����Ѳ���Ҫ������������~$\partial_\mu A^\mu$. ��ȷ, ����Ҳ���Կ���, �������������ʵ�ռ��г����乲����Ķ��׹�ϵ��ì�ܵ�. ��������һ��Ҫ��, ԭ��������������, �ڴ�ʱ, ���ݵ�ȴ���Ǻ��ֽ�ɫ��? ���Ժ�ķ�������ȷ֤, ��ʱ������
\begin{align}
\langle\psi|\partial_\mu A^\mu|\psi\rangle=0.
\end{align}
��ʽ��Ϊ������������, ���Ϊ~Gupta-Bleuler ����. ��һ����, ����������дΪ~$\partial^\mu A^+_\mu(x)|\phi\rangle=0$. ���ѽ��, ǰʽ�ڶ����ռ��м�
\begin{align}
\left(a_{\bm{p}}^3-a_{\bm{p}}^0\right)|\phi\rangle=0;
\end{align}
�˼������ռ��е�~Gupta-Bleuler ����. ��ʽ����չʾ��, ���κ������, �����������ݹ���ͬʱ����, �һ������. ����, ����������, ��ʽ�͸���~$\langle\psi|a^{3\dag}_{\bm{p}}a^3_{\bm{p}'}-a^{0\dag}_{\bm{p}}a^0_{\bm{p}'}|\psi\rangle=0$.


����������ѧ��, ������, �����Ϊ
\begin{align}
H=&\int d^3\bm{x}\left(-\dot{A}^\mu \dot{A}_\mu+\frac{1}{2}\partial_\mu A_\nu\partial^\mu A^\nu\right)=\frac{1}{2}\int d^3\bm{x}\left(A_\mu\partial_0\partial^0A^\mu-\dot{A}^\mu \dot{A}_\mu\right)\nonumber\\
=&-\frac{1}{2}\int d^3\bm{x}iA_\nu i\overset{\leftrightarrow}{\partial}_0\partial^0A^\nu=\int\frac{d^3\bm{p}}{(2\pi)^3}E_{\bm{p}}\left(-g_{\lambda\lambda'}\right)a^{\lambda'\dag}_{\bm{p}}a^{\lambda}_{\bm{p}}\nonumber\\
=&\int\frac{d^3\bm{p}}{(2\pi)^3}E_{\bm{p}}(a^{i\dag}_{\bm{p}}a^i_{\bm{p}}-a^{0\dag}_{\bm{p}}a^0_{\bm{p}'})=\int\frac{d^3\bm{p}}{(2\pi)^3}E_{\bm{p}}\sum_{i=1}^2a^{i\dag}_{\bm{p}}a^i_{\bm{p}}.
\end{align}
��Ȼ, ��������~$P^\mu=-\frac{1}{2}\int d^3\bm{x} iA_\nu i\overset{\leftrightarrow}{\partial}_0\partial^\mu A^\nu$. ��Ȼ�ɼ�, �ڱ�Э�����ӻ���, ������~Gupta-Bleuler ����, ������Ȼʹ���ۻ������ʵ��.




���ڱ��淶�µ�Э����׹�ϵΪ
\begin{align}
[A_\mu(x),A_\nu(y)]=-g_{\mu\nu}[\phi(x),\phi(y)];
\end{align}
���ڱ��淶�µķ���������Ϊ
\begin{align}
D_{F\mu\nu}(x-y)=\langle0|TA_\mu(x)A_\nu(y)|0\rangle=\int\frac{d^4p}{(2\pi)^4}\frac{-ig_{\mu\nu}}{p^2+i\epsilon}e^{-ip(x-y)}.
\end{align}
����ǰ���о�������~$\xi$ ֵ�������ܶ�, ��Ӧ�ij��ķ��������Ӿ���
\begin{align}
D_{F\mu\nu}(x-y)=\int\frac{d^4p}{(2\pi)^4}\frac{-i}{p^2+i\epsilon}\left(g_{\mu\nu}+(\xi-1)\frac{p_\mu p_\nu}{p^2}\right)e^{-ip(x-y)}.
\end{align}






\subsection{������ʸ����, �����ƻ��淶�任}



~~~~�����ɵ�ų����������м���һ������:
\begin{align}
\mathcal{L}=-\frac{1}{4}F_{\mu\nu}F^{\mu\nu}+\frac{1}{2}m^2A_\mu A^\mu,
\end{align}
���Ǿ͵õ���������������ʸ�����������ܶ�. ��ʽ���µ��˶����̼�
\begin{align}
\partial_\mu F^{\mu\nu}+m^2A^\nu=0.
\end{align}
����ʽ����ȡɢ��, �ɵ�~$m^2\partial_\nu A^\nu=0$; ����֪~$\partial_\nu A^\nu=0$~(��������ʹ�������̻�Ϊ~$(\partial^2+m^2)A^\mu=0$). �˽���ǶԳ��ı�Ȼ����, ������������ʸ�������ٿ��Ա�ʩ�й淶�任��.

��������˵��, �����м䲣ɫ�ӵ�������淶�任��������, �������Է��Գ���ȱ��ϣ��˹���Ƶij���. ��������, �Ȳ�׸��.




\newpage



\section{���ӳ��۵ķ�������: ·���������ӻ�}

~~~~��������, һ���, �Ϳ���ֱ�ӽ����໥����������. ����, �������������ɳ��ۻ������໥���ó�����, ·�����ֶ�������Ҫ�һ������һ��, �����ڽ����໥��������֮ǰ, �������о���Ϊ���ӳ���/��ѧ�ķ���������·����������.


%�������ӵ���С������ԭ��, ���������˶���ţ�ٷ���, �������²�����̫������; ������С������ԭ��, ���������˶�����, �༴����һ�����ӻ����~(����) ����, ��������������Ҫ��λ!

%��Ի, ��������������С����ţ�ٷ���, ���������ش��λ; ������~(����, ��λ��) ��������С, ����������, ��������������!!!!!!Ϊʲô��? ԭ����, ��Ȼ������, ����dz��Ĺ۵��������ѧ�۵��ཻţ����ѧ����ʤ֮��! Ҳ��, ���ǻ���, �����ӳ�Ӧ���˻���ţ�����, �������������, �����Եõ�������! ����������?û��! ��Ϊ���۷�����, ���Ǵ����˳���\phi, ������, �����������!

%������ѧ��·������, �ھ���ʱ, ������λ����, ��������С��һ������, ���Ǿ���·��.


֮ǰ�����ij������ӽ��--�����ӻ�, �������ó������˶��������, ��������ʵ�ռ�����ռ��еij�����ǡ���Ķ��׹�ϵ�����е�. �������ӻ��ķ���, ������~(����ϵ�Ƕ��׵�) ��~Jordan-Wigner (����ϵ�Ƿ����׵�) ���ӻ�. ·�����ֵı�����ʽ, �������������ӻ�, ����ijЩ����Ⱥ��߸�Ϊ��Ч. �������Ǿ��о�֮.

\subsection{����΢���ֻ���, ���ӳ������ĵ�ʽ, ����˹����}
\subsubsection{��˹����, ��˹�ػ���, ��˹��}
~~~~
ͨ��ƽ��ת����������ϵ�ٿ���, �ɵû�����˹����~$\int dx e^{-\frac{1}{2}x^2}=\sqrt{2\pi}$; ��һ���ɵó����±�ȸ�˹����
\begin{align}
\int_{-\infty}^{+\infty} dx e^{-\frac{1}{2}ax^2}=\sqrt{\frac{2\pi}{a}}.
\end{align}
����ʽ, ���ǻ��ɵ���������
\begin{align}
\int_{-\infty}^{+\infty} dx e^{-\frac{1}{2}ax^2\pm Jx}=\sqrt{\frac{2\pi}{a}}e^{J^2/2a},
\end{align}
��Ϊ��Դ��˹����; ����~$a,~J$ ��ȡ����.

��������~$\det S$ �Ǵ�~$\bm{x}:=(x^1,x^2,\cdots,x^n)^T$ ��~$\bm{y}=S\bm{x}$ ���ſɱ�����ʽ, ʹ~$\bm{x}^TK\bm{x}=\bm{y}^T\bm{y}$; �Ӷ����ǵõ�~$K=S^TS$ �ǶԳƾ���, ��~$\det S=\sqrt{\det K}$. ����, ���ǿ����ʵ�ռ�~$n$ ά��˹�ػ�������
\begin{align}
&\int_{-\infty}^{+\infty} d^n x e^{-\frac{1}{2}\bm{x}^TK\bm{x}}=\int_{-\infty}^{+\infty} \frac{d^n y}{\det S} e^{-\frac{1}{2}\bm{y}^T\bm{y}}\nonumber\\
=&\frac{1}{\sqrt{\det K}}\prod_{i=1}^n\int_{-\infty}^{+\infty}dy_ie^{-\frac{1}{2}y^2_i}=\sqrt{\frac{(2\pi)^n}{\det K}}.
\end{align}
ͬ��, ��Դ��˹�ػ��־���
\begin{align}
\int_{-\infty}^{+\infty} d^n x e^{-\frac{1}{2}\bm{x}^TK\bm{x}+\bm{J}^T\bm{x}}:=\sqrt{\frac{(2\pi)^n}{\det K}}e^{-\frac{1}{2}JK^{-1}J}.
\end{align}

����, �Ա�ȸ�˹�����е�~$a$ ����~$n$ ����, ���ǻ��ɵ�
\begin{align}
\int_{-\infty}^{+\infty} dx x^{2n}e^{-\frac{1}{2}ax^2}=\sqrt{\frac{2\pi}{a}}\frac{1}{a^n}(2n-1)(2n-3)\cdots5\cdot3\cdot1=\sqrt{\frac{2\pi}{a}}\frac{1}{a^n}(2n-1)!!,
\end{align}
��Ϊ��˹��. �Ӷ�, ���ǿɵ�����ƽ��
\begin{align}
\langle x^{2n}\rangle=\frac{\int_{-\infty}^{+\infty} dx x^{2n}e^{-\frac{1}{2}ax^2}}{\int_{-\infty}^{+\infty} dx e^{-\frac{1}{2}ax^2}}=\frac{1}{a^n}(2n-1)!!.
\end{align}



\subsubsection{��������, ��˹��������}


~~~~���ǰѷ������ֶ���Ϊ��������ά�ػ���:
\begin{align}
\int\mathcal{D}\xi F[\xi]:=\lim_{\Delta x\rightarrow0}\int\left(\prod_x\frac{d\xi_x}{C}\right)F(\xi_x);
\end{align}
�ɴ˶���ɿ���, �������־�������ƽ�Ʋ�����:
\begin{align}
\int\mathcal{D}\xi F[\xi]=\int\mathcal{D}\xi F[\xi+\eta].
\end{align}
����������, ���ǿ���÷�����˹����Ϊ
\begin{align}
\int\mathcal{D}\xi e^{-\frac{1}{2}\xi^2}:=\int\mathcal{D}\xi e^{-\frac{1}{2}\int dx\xi^2(x)}=1;
\end{align}
���Ľ��������ѡ����Ӧ�ij���~$C$ ����; �����Dz����˼��~$\xi:=\int dx\xi$. �������ǽ�һ���ɵ�
\begin{gather}
\int\mathcal{D}\xi e^{-\frac{1}{2}\xi^2\pm J\xi}=e^{\frac{1}{2}J^2},\\
\int\mathcal{D}\xi e^{-\frac{1}{2}\xi K\xi+J\xi}=\frac{1}{\sqrt{\det K}}e^{\frac{1}{2}JK^{-1}J},\label{fanhan}\\
\int\mathcal{D}\xi^*\mathcal{D}\xi e^{-\xi^* K\xi+J^*\xi+J\xi^*}=\frac{1}{\det K}e^{J^*K^{-1}J}.
\end{gather}
����������˼��~$\xi^*K\xi:=\int dxdy\xi^*(x)K(x,y)\xi(y)$ ��.


\subsubsection{����΢��, ���ӳ������ĵ�ʽ}

~~~~
�躯��~$\xi(x)$ ��~$y$ ���иı�~$\varepsilon\delta(x-y)$, ��ɶ��巺��������΢��:
\begin{align}
\frac{\delta F[\xi(x)]}{\delta\xi(y)}:=\lim_{\varepsilon\rightarrow0}\frac{F[\xi(x)+\varepsilon\delta(x-y)]-F[\xi(x)]}{\varepsilon}.
\end{align}
�ɴ˷���~$\frac{\delta}{\delta \xi(y)}\xi(x)=\delta(x-y)$. ���Dz���������������·���΢��~(ע�����п�ʼ���ü��):
\begin{gather}
\frac{\delta}{\delta \xi(x)}\xi=\frac{\delta}{\delta \xi(x)}\int dy \xi(y)=\int dy\delta(x-y)=1,~\frac{\delta}{\delta\xi(x)}\xi G\eta=G\eta,\\
\frac{\delta}{\delta \xi(x)}e^{-\frac{1}{2}\xi^2}=-\xi e^{-\frac{1}{2}\xi^2},~\frac{\delta}{\delta \xi(x)}e^{-\xi\eta}=-\eta e^{-\xi\eta}.
\end{gather}
�ݴ�, ���ǿ����������������
\begin{align}
\int\mathcal{D}\xi F(\xi)e^{-\frac{1}{2}\xi K\xi+J\xi}=&\int\mathcal{D}\xi\sum_nf_n\xi^n e^{-\frac{1}{2}\xi K\xi+J\xi}=F(\delta/\delta J)e^{\frac{1}{2}JK^{-1}J};
\end{align}
��������ȥ����~$1/\sqrt{\det K}$. �ɴ�����������ʽ
\begin{align}\label{center}
\int\mathcal{D}\xi e^{-\frac{1}{2}\xi K\xi-V(\xi)+J\xi}=e^{-V(\delta/\delta J)}e^{\frac{1}{2}JK^{-1}J},
\end{align}
��Ϊ���ӳ��۵����ĵ�ʽ.


\subsubsection{����˹����~(�����) ����΢����}
~~~~�ڷ����ӻ򳬶ԳƵ�������, ������Ҫ�õ�����˹����, �������. һ��~$n$ ά����˹��������~$n$ ������Ԫ~$\xi_i,~i=1,2,\cdots,n$, ����~$\{\xi_i,\xi_j\}=0$. �Ӷ�������֪, ���κθ���˹��������
\begin{align}
\xi^2=0.
\end{align}
�����, ����˹������չ��ʽֻ����������, ��~$f(\xi)=p+q\xi,~g(\xi,\eta)=p+q\xi+r\eta+s\xi\eta$ ��; ����~$p,q,r,s$ Ϊ~$c$ ��. ����, ����˹�����ڸ�������������~$(\xi^*)^*=\xi,~(\xi\eta)^*=-\eta^*\xi^*$.

����Ҫ�����˹�����ĺ���������Ȼ����ƽ�Ʋ�����, ��~$\int d\xi f(\xi+\eta)=\int d\xi f(\xi)$; ��~$f(\xi)=p+q\xi$ ����, �͵õ�~$\int d\xi q\eta=0$. Ҫ����ʽ���κ�~$c$ ��~$q$ ������, �ͱ�Ȼ�ó�
\begin{align}
\int d\xi=0.
\end{align}
ע��΢��~$d\xi_i$ Ҳ��һ������˹������, ������~$\{\xi_i,d\xi_j\}=0,~\{d\xi_i,d\xi_j\}=0$. ����, ��Ϊһ��~$c$ ��, ���Ƕ���
\begin{align}
\int d\xi\xi=1.
\end{align}
���������Ǹ���˹�����Ļ�����������, ���������ǾͿ��Խ��о���������. ����~$\int d\xi f(\xi)=q,~\int d\xi g(\xi,\eta)=q+s\eta,~\int d\eta g(\xi,\eta)=r-s\xi$.

�������ڿ��Ƕ�ά���, ��~$\xi:=(\xi_1,\xi_2,\cdots,\xi_n)$. �ڻ���Ԫ�任��~$\xi=S\eta$ ��, ������
\begin{align}
d^n\xi=\frac{1}{\det S}d^n\eta;
\end{align}
��Ȼ����~$c$ ���������ǡ���෴��.

���������о�����˹������΢������. ΢���������������~$\frac{1}{\Delta \xi_i}$ ��˵�����, ���Խ��������й�; ���Ƕ������������. ���ǿɵ�
\begin{align}
\frac{\partial}{\partial\xi}g(\xi,\eta)=q+s\eta,~\frac{\partial}{\partial\eta}g(\xi,\eta)=r-s\xi.
\end{align}
�ɴ˿ɼ�, ����˹������΢������ָ�����ͬ�Ľ��. ����, ��������֤��
\begin{align}
\left\{\xi_i,\frac{\partial}{\partial\xi_j}\right\}=\delta_{ij},~\left\{\frac{\partial}{\partial\xi_i},\frac{\partial}{\partial\xi_j}\right\}=0.
\end{align}


���, ���������Ǹ���˹�����ĸ�˹����. �������������ĸ���˹������, ��Ȼ��֪~$e^{-\bar{\xi}\xi}=1-\bar{\xi}\xi$; �������ǿɵ�
\begin{align}
\int d\bar{\xi}d\xi e^{-\bar{\xi}\xi}=1,~\int d\bar{\xi}d\xi e^{-\bar{\xi}a\xi}=a=e^{\ln a}.
\end{align}
�ƹ㵽��ά���, ����������
\begin{align}
\int d^n\bar{\xi}d^n\xi e^{-\bar{\xi}\cdot\xi}=1.
\end{align}
�����»�Ԫ~$\eta=S\xi,~\bar{\eta}=\bar{S}\bar{\xi}$ �������, ���ǿ����
\begin{align}
\int d^n\bar{\eta}d^n\eta e^{-\bar{\eta}\cdot\eta}=\frac{1}{\det(\bar{S}^{(T)}S)}\int d^n\bar{\xi}d^n\xi e^{-\bar{\xi}^T\bar{S}^TS\xi}=1,
\end{align}
������
\begin{align}
\int d^n\bar{\xi}d^n\xi e^{-\bar{\xi} K\xi}=\det K;
\end{align}
����~$K:=\bar{S}^TS$. ͬ��, ���ǿɵ����·�������:
\begin{align}
\int \mathcal{D}\bar{\xi}\mathcal{D}\xi e^{-\bar{\xi} K\xi}=\det K;
\end{align}
�����м��~$\bar{\xi}K\xi:=\int dxdy\bar{\xi}(x)K(x,y)\xi(y)$; ����~$\det K$ �ɱ�������ֲ����.


\subsection{������ѧ��·�����ֱ���, �����ʱ��, ���ɷ���, Bethe-Salpeter ����}
%\subsection{������ѧ��·��������ʽ: Ѧ���̷��̴����ӵĶԸ�����·��~(��δ�ؾ���~$S=0$ ������) �ķ�������ʽ����; Bethe-Salpeter ����}
%���ø�Ѧ�Ϸ���, ������ѧ��������, ���Ұ���
~~~~��~$K(q_n,t;q_0,t_0)=\langle q|U(t,t_0)|q_0\rangle=\langle q|e^{-iH(t-t_0)}|q_0\rangle=\langle q_n,t|q_0,t_0\rangle$, ��Ѧ���̷���/���Ĵ�����/ԾǨ���, ���ǿ��Խ�����������~(�������ǽ�~$e^{-iH(t-t_0)/n}$ ��~$e^{A+B}=e^Ae^Be^{-[A,B]/2}$ ���ֽ�, �������˸߽���, Ȼ���ֽ����˺ϲ�):
\begin{align}
&K(q_n,t;q_0,t_0)=\langle q_n|e^{-iH(t-t_0)}|q_0\rangle\nonumber\\
=&\prod_{i=0}^{n-1}\int dq_i\langle q_{i+1}|e^{-iH(t-t_0)/n}|q_i\rangle=\prod_{i=0}^{n-1}\int dq_i  dp_i  \langle q_{i+1}|p_i\rangle\langle p_i|e^{-iH(t-t_0)/n}|q_i\rangle\nonumber\\
=&\prod_{i=0}^{n-1}\int \frac{dq_i  dp_i}{2\pi} e^{ip_i(q_{i+1}-q_i)}e^{-iH(t-t_0)/n}=\prod_{i=0}^{n-1}\int \frac{dq_i  dp_i}{2\pi} e^{ip\dot{q}dt}e^{-iHdt}\nonumber\\
=&\prod_{i=0}^{n-1}\int \frac{dq_i  dp_i}{2\pi} e^{i(p\dot{q}-H)dt}:=\int\mathcal{D}q\mathcal{D}pe^{i\int dt(p\dot{q}-H)};
\end{align}%��ϵ���ж���������, ����һ��Ҫ���������ݵݵĸ�ֵ�����������ļ��ٶ�!!!!!!!!!!!
ע������������п�ʼ, $H$ �Ѳ������. ����, ��Ϊ��ij��������������, ���Ƕ��ῼ�ǹ�һ��, ���������Լ���������������, ���Ƕ�ʡ����һ����������. �����ܶ�������~$H(p,q)=\frac{p^2}{2m}+V(q)$ ����ʽ, �����ǽ�һ���ɵ�
\begin{align}
K=&\int\mathcal{D}qe^{-i\int dt V(q)}\int \mathcal{D}pe^{\int dt(-\frac{i}{2m}p^2+i\dot{q}p)}
=\int\mathcal{D}qe^{-i\int dt V(q)}e^{i\int dt\frac{i}{2}m\dot{q}^2}\nonumber\\
=&\int\mathcal{D}q e^{i\int dt\left[\frac{1}{2}m\dot{q}^2-V(q)\right]}=\int\mathcal{D}q e^{i\int dtL}=\int\mathcal{D}q e^{iS};
\end{align}
�����õ��˷�������ʽ~(\ref{fanhan}).

���������ð������������п��ܵľ���·��~(��������~$S=0$ ������) ���з������ֵķ�ʽ��ʾ����, ��Ϊ������ѧ��·��������ʽ; �˷��������������������������������ѧ��ƽ����Ѧ���̷���~(��Ի������ѧ��������ʽ) ����һ����. ǰ��, ������·������~$K=\int\mathcal{D}q e^{iS}$ ����ΪѦ���̷��̵���Ȼ���۳��ֵ�; ��Ȼ, ����Ҳ�ɰѴ˽�������͵�����һ��ԭ��~(��Ȼ, ��֮Ϊ�������ǿ��Ե���Ѧ���̷��̵�), ��Ϊ����������ԭ��.

·������, ������ͼ���Ͻ�, �����൱����ֱ�۵�. �����Ѿ�֪��, �Ա��ھ�����ѧ��, �����������ѧ����ʹ�������Ӱ�������С�������ķ�ʽ�����˶�, ��Ȼ, ��΢��������, ���ӵ�����Ѧ���̷��̵�·��, �Ǿ��и��־���������ֵ��·���ĵ���; ���ɴ����������~(�������걸��ϵ���ɼ���) �ݸ�˹ԭ������ɿ���. ��ô���������/˼�뱾������һ��ԭ��~(������ΪѦ���̷��̵�����), Ҳ���Ƿdz���Ȼ����. %����, ���ǾͿ����԰����и�����������·�������з������ֵİ취�����촫����. �˰취��Ϊ����������ԭ���ľ�����ѧʵ��, �ͳ�Ϊ·������. ��Ȼ, ·�����ַ�����Ѧ���̷����ǵȼ۵�. ���ڳ�����, ��������·�����ֿ���ƽ���ڸ������˵ij����̶���������������. �������Ǿ;����о�֮.


��·������������ѧ��, �ռ�����~$q(t)$ �ı�ʱ����Ȼ����
%���--��Ϊ~$q(t)$--�ı�ʱ���Ϳɱ�Ϊ
\begin{align}
\langle q,t|Tq(t_1)\cdots q(t_n)|q_0,t_0\rangle=\int\mathcal{D}qq(t_1)\cdots q(t_n)e^{i\int dtL(q,\dot{q})};
\end{align}
��ϵ������Դ~$L\rightarrow L+Jq$ ���ԾǨ����Ϳɱ�Ϊ
\begin{align}
\langle q,t|q_0,t_0\rangle_J=\int\mathcal{D}q e^{i\int dt Jq}e^{i\int dtL(q,\dot{q})}=\langle q,t|Te^{i\int dt Jq}|q_0,t_0\rangle.
\end{align}
����Դ��������ϵ�Ĵӳ���̬��ĩ��̬��ԾǨ���
\begin{align}
Z[J]:=\langle 0,\infty|0,-\infty\rangle_J=\langle 0,\infty|Te^{i\int dt Jq}|0,-\infty\rangle%=N\int\mathcal{D}qe^{i\int^{\infty}_{-\infty}dt\left[L+Jq+\frac{1}{2}i\varepsilon q^2\right]}
\end{align}
��Ϊ���ɷ���; �������ǿ�֪�����ʱ���Ļ�̬ƽ�������ɷ�����΢��֮�������¹�ϵ:
\begin{align}
\langle 0,\infty|Tq(t_1)\cdots q(t_n)|0,-\infty\rangle=\frac{\delta^nZ[J]}{i^n\delta J(t_1)\cdots\delta J(t_n)}\Big|_{J=0}.
\end{align}
��ʵ��, ������ʽ�������ӳ����и�������.

%ע����Щ�м�ŵĶ������, ����ת�����ӳ�����, �м���ϳ���������ǿ��Ե�.





%������ѧ�е��е�·������, �Ǹ��־���������·����ȡ; ���ӳ����е�·������, �Ǹ�����������������·��ȡ!
%������ѧ������, ������Ƕ������ӻ�, ·���������ж���·�����ֻ�. ����, һЦ.



%������ѧ·������: take ����·�� to Ѧ���̷���·��/�����·��
%���ӳ���·������: take ����·�� to ��Ѧ���̷��̾�/���Ķ�·��


���������ɷֽ�Ϊ����, ����һ��~$L_0$ ʹ���������ϸ��, ��һ��~$-V$ ��΢��, �����ǿɵ�
\begin{align}
&K(q,t;q_0,t_0)=\int\mathcal{D}qe^{i\int dt(L_0-V)}=\int\mathcal{D}qe^{i\int dtL_0}e^{-i\int dtV}\nonumber\\
=&\int\mathcal{D}qe^{i\int dtL_0}\sum_{n=0}^{\infty}\frac{1}{n!}\left(-i\int dtV\right)^n=K_0+\int\mathcal{D}qe^{i\int dtL_0}\sum_{n=1}^\infty\frac{1}{n!}\left(-i\int dtV\right)^n\nonumber\\
:=&K_0+\sum_{n=1}^{\infty}(-i)^nK_n:=K_0+K_0(-iV)K_0+K_0(-iV)K_0(-iV)K_0+\cdots\nonumber\\
=&K_0-iK_0VK=K_0-iKVK_0;
\end{align}
ע��������ز����в�ȡ�˻��ֵļ�д; ����д�����ʵ��Ҳ�������ռ��еĹ�ʽ. ��ʽ��Ϊ~Bethe-Salpeter ����; ��ɽ�һ���任Ϊ~$K=\frac{K_0}{1+iVK_0}$.







\subsection{���ӳ��۵�·��������ʽ}
%���ø㳡����, ��������������, �ֱ��Ұ���

%Ҳ���Զ���һ��Ѧ���̳���·������, ע��˼������Ѧ���̷��̵�·�����ֵ���������ϵ.
%������ѧ�ǵ�����, ������·��������, ������ָ���ϵ����������ǵ����ӵ�, ������ʱ���ݻ�����ϵ��������ǵ����ӵ�; ���ӳ�����, ���ǿ��ǵ��Ƕ����ӵ�, ����ָ���ϵ�, ����ȫ������, ����ϵ���ܵ�������������, Ҳ������Ȼ֮����. ������, Ҫ��·������ǰ�е�$\langle\phi|$ ��һ˼��, �Լ��Գ�����ʱ���ݻ����, ���������ڳ����������ȷ������, ������������������ѧ�еIJ�����̬ʸ��, ��ʲô����. ��������ʵ����ͬһ��; Ҳ����˵, ������ʲô����??????


\subsubsection{���ӳ����е�·�����ּ�����������ѧ�е���������ϵ, ��λ�μ���~``��Ѧ���̷���'', ���ɷ���, ���ֺ���}
%\subsubsection{���ӳ����е�·��������������ѧ�е���������ϵ: δ��Ҫ������~$S=0$, ���Թ������ӵĶ��dz�λ�εĴ�����; �໥���ó����е����ɷ���, ���ֺ���}
~~~~�����������ѧ�еĹ�ʽ, ���ӳ����е�·�����ֹ�ʽ������ֱ��ȡ��Ϊ
%����ϸ����������ѧ, ����dz�1����2 (�м�������K-G����) �Ĵ�����; ��, ���ǻ��ǰ���ʽ��Ϊ, �ض��ij�, ���Ӵ�һ�ص���һ�صĴ���; ����, ����ʽ������ֵ����������Ҫ��Ĵ�����~(��Ϊ���dz�һ��������), ����λ�IJ���!!!!!!!!!!!!!!!!!!!!!!!!!!!!!!!!!!!!!!!!!!ʣ�µ�, ���Ǿ���İ취��֯��������!!!!!!!!!!!!!
%���ѷ���, ��λ�ν�����·��; �Գ�λ��\phi ����~D\phi, ���൱�ڿ����������ӵ�~Dq
\begin{align}
&K(\phi,t;\phi_0,t_0)={}_S\langle\phi,t|\phi_0,t_0\rangle_S={}_S\langle\phi|e^{-iH(t-t_0)}|\phi_0\rangle_S={}_I\langle\phi|e^{-iH^I_{\textrm{int}}(t-t_0)}|\phi_0\rangle_I\nonumber\\
=&\int\mathcal{D}\phi e^{i\int dx\mathcal{L}(\phi,\partial_\mu\phi)}=\int\mathcal{D}\phi e^{i\int dt L(\phi,\partial_\mu\phi)};
\end{align}%������������, ���ú�ɭ���澰, ��ʸ����, ����ֵΪ��; �໥���û澰��, ��ʸ�ݻ����໥���õĹ��ܶȶ�������.
%Ѧ���̷���, ӵ�в�ͬ�ľ���������, ��ӵ��һ����С������������.
����~$H=\int d^3\bm{x}\mathcal{H}$ Ϊ��/��ϵ���ܹ��ܶ���, ��~$|\phi\rangle:=|n_1,\cdots n_i\rangle$ �dz�λ��. %����������ÿռ�Ķ�����̬, ��Իһ�ֳ�λ��.
%�м�����~(���и�������������) ��λ��, ���й���!
%�����ݶȵĸ�ֵ, �������Ǹ��ܵ�����ν����!

%�����ɳ�����, �����ں�ɭ���澰�¹���, ��ʱ��λ�β��Ժ�ʱ��, ������ʱ~$\langle\phi|e^{-iH(t-t_0)}|\phi_0\rangle=0$; ������С������ԭ��~$\delta S=0$

%��Ҫ����: �����໥����ʱ, ������������ʱ, ����·�����ֽ��ӦΪ��; ����ָ���Ͽ�, ָ��������ϵ����������, ��ô���ֽ��Ӧ��Ϊ��Ŷ�.

%�ڵ�����������ѧ, ��Ѧ���̷�����, ��λ�ε��ݻ�����, �������Ӹ��ʷ��ķ���, �볡�������, ��һ����; �ڶ��������ӳ�����,

�������������ѧ��, Ѧ���̷��̻�������ѧ·�����ְ������˶��ľ���·����Ωһ��~(��~``����·��'' ��Ωһ��), ����ʽ�Ӱ�����·��~(������; �����Ϸ��̻�~K-G ���̵�) Ҳ��Ωһ����. Ҳ����˵, ��ʽ��ζ�ų����δ��Ҫ����ǰ�ĵ���Щӵ����С���������������ӳ�����; ��ͬ������ѧ��ָ���ľ�������δ��Ҫ����ӵ����С����������������·��һ��.
%ע��~$\phi$ ��������ij����, ��~$K-G$ ���̵��Ǹ���, ��~$|\phi,t\rangle$ ������ȡ
%ע��������д����~$t$ ��������~$\phi(x)$ ��������ʱ�������IJ�ͬ, �Լ�ע��~$|\phi,t\rangle$ ��������������ѧ��~$|\psi(t)\rangle$ ������IJ�ͬ. --

%�Ӷ�, Ҳ����˵, $K(\phi,t;\phi_0,t_0)$ ��ζ�ų���ӵ��ij�����ӳ�������Ϊ���ɵ�λ��, ��ӵ����һ�����ӳ��������ù��ɵ�λ�εĴ���.
�Ӷ�, Ҳ����˵, $K(\phi,t;\phi_0,t_0)$ ��ζ�ų���ij��λ�ε���һ��λ�εĴ������ʷ�. ����, ��ȴ������������Ҫ�Ĵ�����; ������Ҫ��, ����ǰ���ᵽ�������������Ӵ����ķ���������. ���Ǻܿ콫����, ���Ҫ��, �����Dz���ֵ�, ������·������������ѧ�е���ع�ʽǡ����һ�µ�.



%��������ͬһ�����ӳ����̵ij�, �������Ӵ�һ�㵽��һ��Ĵ���.

%���, ͬ�������������ѧ, ���ǿ��Եó���Ωһ���˵ij��������·��, ����Ωһ�˶�����--���Dz�����֮Ϊ��Ѧ���̷���--Ϊ
%\begin{align}
%|\phi,t\rangle=e^{-iH(t-t_0)}|\phi_0,t_0\rangle;
%\end{align}
%$\phi$ ���������ض�λ�ε�һ����, $|\phi,t\rangle$ ������ȡ��ͬλ��~(����) �ĸ��ʷ�, �������̼������˸��ʷ���Ωһ����.


%�Dz����ݵ�, �������Ǿͽ���������ʵ��.



%���ǽ�����, �Ա���������ѧ��, ·������ͬʱ������Ѧ���̷�������ѧ���Ķ��׻�, �����ӳ�����, (��·��������ʽ��ͬʱ�����˳������볡���Ķ��׻�, ��������һ��, �����Ķ��׻�����û��������). �����ڴ�, ����ʽ���ɿ���, �Ա���������ѧ��·��������ʽ��, ���ӵľ���������Ψһ��, �������������Ѧ���̷�����Ψһ��, �����ӳ�����, ����������������˶�����Ҳ����Ψһ����. ���������ֱ仯, �����Ժ󽫷���, ���ӳ����еĴ����Ӳ���~$K(\phi,\phi_0)$, ���Ա�����ݳ���.




���ӳ�����, ���໥���������, ��ϵ�����ɷ�����
\begin{align}
Z_0[J]:=\langle 0,\infty|0,-\infty\rangle_J=\int\mathcal{D}\phi e^{i\int^{\infty}_{-\infty}dx\left(\mathcal{L}_0+J\phi+\frac{1}{2}i\varepsilon \phi^2\right)}=\langle 0|Te^{i\int dx J\phi}|0\rangle;
\end{align}
���Ƕ���������Ϊ��ϵ��~$n$ �������������ֺ���
\begin{align}
G(x_1,\cdots,x_n):=&\langle0|T\phi(x_1)\cdots \phi(x_n)|0\rangle=\int\mathcal{D}\phi \phi(x_1)\cdots \phi(x_n) e^{i\int dx\mathcal{L}_0}\nonumber\\
=&\frac{\delta^nZ_0[J]}{i^n\delta J(x_1)\cdots\delta J(x_n)}\Big|_{J=0}.
\end{align}
�������໥���õ����, ���ɷ�����Ϊ
\begin{align}
Z[J]:=&\int\mathcal{D}\phi e^{i\int^{\infty}_{-\infty}dx\left(\mathcal{L}_0+\mathcal{L}_{\textrm{int}}+J\phi+\frac{1}{2}i\varepsilon \phi^2\right)}\nonumber\\
=&e^{i\mathcal{L}_{\textrm{int}}\left(\delta/i\delta J\right)}Z_0[J];
\end{align}
���еڶ������Ǽ�����~$\mathcal{L}_{\textrm{int}}=\mathcal{L}_{\textrm{int}}(\phi)$, ���õ��˷�������ʽ~(\ref{center}). ����~$n$ �������������
\begin{align}
G(x_1,\cdots,x_n):=&\langle\Omega|T\phi_{H}(x_1)\cdots \phi_{H}(x_n)|\Omega\rangle=\int\mathcal{D}\phi \phi_H(x_1)\cdots \phi_H(x_n) e^{i\int dx\mathcal{L}}\nonumber\\
=&\frac{\delta^nZ[J]}{i^n\delta J(x_1)\cdots\delta J(x_n)}\Big|_{J=0}.
\end{align}
����, ��Ϊ���ǽ�����һ��֤����ʽ��~(\ref{omega}), ��������
\begin{align}
\langle\Omega|T\phi_{H}(x_1)\cdots \phi_{H}(x_n)|\Omega\rangle=\frac{\int\mathcal{D}\phi \phi_I(x_1)\cdots \phi_I(x_n) e^{i\int dx\mathcal{L}}}{\int\mathcal{D}\phi e^{i\int dx\mathcal{L}}}.
\end{align}







\subsubsection{���ɱ�����/��������ʸ�����Ĵ�����, ���ɷ����������������}
%ע��˷������ϸ�ƽ�����������ӻ���; ����˵, ��ʱ���ǿ��Լ�װ��ȫ��֪��K-G ���̼������ӻ����κ�����!!!!!!!!!!!!!!!!!!!!!!!!!!!!!!
\paragraph{���ɱ�����}
~

���ǵ��������������άɢ����, ���ɱ������������ܶȿ���дΪ~$\mathcal{L}_0=\frac{1}{2}(\partial_\mu\phi\partial^\mu\phi-m^2\phi^2)=-\frac{1}{2}\phi(\partial^2+m^2)\phi$; ��������Դ��С�ĸ�������, ���ǾͿɵõ�
\begin{align}
\mathcal{L}&=\mathcal{L}_0+J\phi+\frac{1}{2}i\varepsilon \phi^2=-\frac{1}{2}\phi(\partial^2+m^2-i\varepsilon)\phi+J\phi\nonumber\\
:&=\frac{i}{2}\phi D_F^{-1}\phi+J\phi;
\end{align}
����~$D_F:=-i(\partial^2+m^2-i\varepsilon)^{-1}$. �Ա�������ʽ�ij���֪ʶ, ����֪���˼������ķ���������. Ҳ����˵, �ڶ����͵������ܶ���, ������֮�����������֮��, �������ɳ��Ĵ�����. ��������Ǿ���һ���Ե�. ���ɱ����������ɷ���Ϊ
\begin{align}
Z_0[J]=&\int\mathcal{D}\phi e^{-i\int dx\left[\frac{1}{2}\phi(\partial^2+m^2-i\varepsilon)\phi-J\phi\right]}\nonumber\\
=&e^{-\frac{1}{2}J\left[i(\partial^2+m^2-i\varepsilon)\right]^{-1}J}\nonumber\\
:=&e^{-\frac{1}{2}JD_FJ}.
\end{align}
ע������������, ���Dz�ȡ�˼��, ��~$e^{-\frac{1}{2}JD_FJ}:=e^{-\frac{1}{2}\int dxdyJ(x)D_F(x-y)J(y)}$.

ͨ�����ֺ��������ɷ����Ĺ�ϵʽ, ���Dz���������ĵ�����ֺ���Ϊ
\begin{align}
G(x)=\langle0|\phi(x)|0\rangle=\frac{\delta}{i\delta J(x)}e^{-\frac{1}{2}JD_FJ}=(iD_{Fxy}J_y)e^{-\frac{1}{2}JD_FJ}\Big|_{J=0}=0;
\end{align}
���Ƶ�, ���ǿ���ó���������ֺ���Ϊ
\begin{align}
G(x,y)=\langle0|T\phi(x)\phi(y)|0\rangle=D_F(x-y);
\end{align}
�����ķ���������. һ���, ���Dz������, ���ɳ�����������ֺ���Ϊ~0, ż������ֺ���������������������, ������������ֺ����˻�֮��.

\paragraph{����������}
~

�������������������ܶ�~$\mathcal{L}_0=\bar{\psi}(i\gamma^\mu\partial_\mu-m)\psi$ �е����������ֱ�������Դ, ������һ��С�ĸ�������, �͵�
\begin{align}
\mathcal{L}=&\bar{\psi}(i\gamma^\mu\partial_\mu-m)\psi+\bar{\eta}\psi+\bar{\psi}\eta+i\varepsilon\bar{\psi}\psi\nonumber\\
:=&-i\bar{\psi}S_F^{-1}\psi+\bar{\eta}\psi+\bar{\psi}\eta;
\end{align}
����������֪~$S_F:=i(i\gamma^\mu\partial_\mu-m+i\varepsilon)^{-1}$ �������ķ���������. ����, ���ǿ���ñ��������ɷ���Ϊ
\begin{align}
Z_0[\bar{\eta},\eta]=&\int\mathcal{D}\bar{\psi}\mathcal{D}\psi e^{i\int dx (-i\bar{\psi}S_F^{-1}\psi+\bar{\eta}\psi+\bar{\psi}\eta)}\nonumber\\
=&\int\mathcal{D}\bar{\psi}\mathcal{D}\psi e^{i\int dx \left[-i(\bar{\psi}-i\bar{\eta}S_F)S_F^{-1}(\psi-iS_F\eta)+i\bar{\eta}S_F\eta    \right]}\nonumber\\
=&e^{-\bar{\eta}S_F\eta}.
\end{align}
�ɴ˿ɵñ�����������ֺ���Ϊ
\begin{align}
G(x,y)=\langle0|T\psi(x)\bar{\psi}(y)|0\rangle=-\frac{\delta}{i^2\delta\bar{\eta}(x)\delta\eta(y)}\Big|_{\bar{\eta}=\eta=0}=S_F.
\end{align}



\paragraph{����ʸ����}
~

ǰ��, ������һ���֪��, �ڶ����͵������ܶ���, ������֮�����������֮��, �������ɳ��Ĵ�����. �������ǰѵ�ų������������ɶ�����, ���õ��䴫����.
\begin{align}
\mathcal{L}=&-\frac{1}{4}F_{\mu\nu}F^{\mu\nu}=-\frac12\partial_\mu A_\nu F^{\mu\nu}=\frac{1}{2}A_\nu\partial_\mu F^{\mu\nu}=\frac{1}{2}A_\nu\partial_\mu(\partial^\mu A^\nu-\partial^\nu A^\mu)\nonumber\\
=&\frac12 A_\nu(\partial^2g^{\mu\nu}-\partial^\mu\partial^\nu)A_\mu.
\end{align}
����, $i(\partial^2g^{\mu\nu}-\partial^\mu\partial^\nu)^{-1}$ ���κ���������������, ��������Ϊ�����Ĵ�����. ԭ������Ȼ��: ��û��δ���ǹ淶����. ����ȡ���״Ĺ淶, ���Ǵ�ʱ~$\mathcal{L}=\frac12 A_\nu\partial^2g^{\mu\nu}A_\mu$, ��Ȼ
\begin{align}
D_{F\mu\nu}:=\frac{i}{g^{\mu\nu}\partial^2}=\frac{ig^{\mu\nu}}{\partial^2-i\varepsilon}
\end{align}
�͵�ȷ�DZ����Ĵ�������.


��ʸ�������������ӻ���, �������ѱ��������ܶ�д���˸�һ�����ʽ; ���ǿ�ͬ�������һ����ʽ�µĴ�����.
\begin{align}
\mathcal{L}=&-\frac{1}{4}F_{\mu\nu}F^{\mu\nu}-\frac{1}{2\xi}\partial_\mu A^\mu\partial_\nu A^\nu\nonumber\\
=&\frac12 A_\nu(\partial^2g^{\mu\nu}-\partial^\mu\partial^\nu)A_\mu+\frac{1}{2\xi}A_\nu\partial^\nu\partial^\mu A_\mu\nonumber\\
=&\frac12 A_\nu\left[\partial^2g^{\mu\nu}-(1-\frac{1}{\xi})\partial^\mu\partial^\nu\right] A_\mu;
\end{align}
�ɴ˿ɵô�ʱ�ij��Ĵ����Ӽ�
\begin{align}
D_{F\mu\nu}=\frac{-i}{p^2+i\varepsilon}\left[g^{\mu\nu}-(1-\xi)\frac{p_\mu p_\nu}{p^2}\right].
\end{align}


















\newpage


% !Mode:: "TeX:UTF-8"


\section{相互作用场论}

































\newpage
\section{Ȧͼ��ɢ��~(��Ȧ) ������: �����ӵ綯��ѧΪ��}

~~~~���ڵ��ɢ������ (���Ժ�Ĵž�����), ��ͼ���Ƽ��ܸ����ȽϺõĽ����. �������, ��һ�����Ǹ߽׼�Ȧͼ����~(��Ϊ��������), ���������õĽ��. Ȼ��д����������ʽ���ѷ���, Ȧͼ�Ƿ�ɢ��. ����������, ����Ҫ������������һ����. �������Ǿ���һ����̽��֮.

ǰһ����, �����Ѽ�ʶ��~$\phi^4$ ���۵�����; ���ߵ���������, �����������ɴ����ӻ�����ͼ~(�����ߴ�����) ��~(�����Ӳ���Լ) ����. �ɱ˴��������, ���Dz��ѵ�֪���ӵĶ�������Ϊ
\begin{align}
-i\Sigma^{(2)}(p)=\begin{aligned}\includegraphics[width=2.5 cm]{pic/eself.jpg}\end{aligned}=(-ie)^2\int\frac{d^4k}{(2\pi)^4}\frac{-ig_{\mu\nu}}{k^2+i\epsilon}\gamma^\mu\frac{i}{\slashed p-\slashed k-m+i\epsilon}\gamma^\nu.
%\not p\not k\slashed{p}\hatslashed{D}\centernot{p} k\!\!\!/
\end{align}
����, ���ӵĶ�������~(�ֿɽж�����ռ���) Ϊ
\begin{align}
i\Pi^{(2)}_{\mu\nu}(k)=\begin{aligned}\includegraphics[width=2.7 cm]{pic/pself.jpg}\end{aligned}=(-ie)^2(-1)\int\frac{d^4p}{(2\pi)^4}\operatorname{tr}\left[\gamma_\mu\frac{i}{\slashed p-m}\gamma_\nu\frac{i}{\slashed p-\slashed k-m}\right];
\end{align}
�����Ӳ���Լ����/���涥��/��ȫ����/���Ǻ���~(����ڵ��ӻ���ӵ�����)~$\Gamma_{\mu}$ �Ķ�������~(��������Ŷ���) дΪ
\begin{align}
\Lambda^{(2)}(p)=\begin{aligned}\includegraphics[width=2.5 cm]{pic/vertex2.jpg}\end{aligned}=-ie^2\int\frac{d^4k'}{(4\pi)^4}\frac{1}{k'^2}\gamma_\nu\frac{1}{\slashed{p}'-\slashed{k}'-m}\gamma_\mu\frac{1}{\slashed{p}-\slashed{k}'-m}\gamma^\nu.
\end{align}
��Ȼ, �����������, ���Ƿ�ɢ��. һ���, ���ӳ����еĸ߽�����, ��Ȧͼ, ��������һ����. ������, ��ϵͳ��������Щ��ɢ��һ������.

�������ĵ�һ��, �ǰ�Ȧͼ����еķ�ɢ�������, ��һ���̳�Ϊ���滯. ���滯�ķ���, �нض����滯, ������滯�Լ�ά�����滯~(�������ƹ㵽~$d$ ά, Ȼ������~$d\rightarrow4$) ��. ����, ���Dz��ú���. �ھ���������滯����֮ǰ, ������Ҫ׼��һЩ���ڵ��ά������, ������������ʽ�Լ�~$\Gamma$ �����ȷ����֪ʶ. �����������������ЩԤ������.

����, ���ٷ�����֪, �ƹ㵽~$d$ ά�Ժ�, �Ե��������~$[q]=L^{d/2-2}=\Lambda^{2-d/2}$; �������ǿ���~$q=\mu^{2-d/2}e$, ����~$\mu$ ��Ϊ��������. ���, ���������¹�ʽ
\begin{align}
\frac{1}{ab}=\int^1_0\frac{dx}{\left[ax+b(1-x)\right]^2},
\end{align}
��Ϊ������������ʽ. ����, ����~$\gamma$ ����, ��~$d$ άʱ������~$\{\gamma^\mu,\gamma^\nu\}=2g^{\mu\nu}$, ���������Ȳ�׸��. ���, ����������~$d$ ά���Ͽռ�Ķ������ֹ�ʽ
\begin{align}
\int\frac{d^dp}{(2\pi)^d}\frac{1}{(p^2-m^2)^\alpha}=\frac{(-1)^\alpha i}{(4\pi)^{d/2}}\frac{\Gamma(\alpha-d/2)}{\Gamma(\alpha)}\frac{1}{(m^2)^{\alpha-d/2}}.
\end{align}
��������ƽ�Ʋ�����~$\int d^dpF(p)=\int d^dpF(p+q)$, �Լ��ٽ�ǰ�������~$q$ ������, ���ǻ��ɷֱ�õ�������һʽ�����ʽ
\begin{gather}
\int\frac{d^dp}{(2\pi)^d}\frac{1}{(p^2+2pq-m^2)^\alpha}=\frac{(-1)^\alpha i}{(4\pi)^{d/2}}\frac{\Gamma(\alpha-d/2)}{\Gamma(\alpha)}\frac{1}{(q^2+m^2)^{\alpha-d/2}},\\
\int\frac{d^dp}{(2\pi)^d}\frac{p^\mu}{(p^2+2pq-m^2)^\alpha}=\frac{(-1)^{\alpha-1} i}{(4\pi)^{d/2}}\frac{\Gamma(\alpha-d/2)}{\Gamma(\alpha)}\frac{1}{(q^2+m^2)^{\alpha-d/2}},\\
\int\frac{d^dp}{(2\pi)^d}\frac{p^\mu p^\nu}{(p^2-m^2)^\alpha}=\frac{(-1)^{\alpha-1} i}{(4\pi)^{d/2}}\frac{g^{\mu\nu}\Gamma(\alpha-1-d/2)}{2\Gamma(\alpha)}\frac{1}{(m^2)^{\alpha-1-d/2}}.
\end{gather}
����,


����~QED �����������ܶ�Ϊ
\begin{align}
\mathcal{L}=\bar{\psi}(i\gamma^\mu \partial_\mu-m)\psi-q\bar{\psi}\gamma^\mu\psi A_\mu-\frac{1}{4}F_{\mu\nu}F^{\mu\nu}-\frac{\lambda}{2}(\partial_\mu A^\mu)^2
\end{align}













\subsection{Ward-Takahashi ���ʽ, ��ѧ����}

\newpage
\section{����ɫ����ѧ, ��-Mills ����}

~~~~��һ˼���Ϳɷ���, ʹ��������ɺ˵��໥����, ����ǿ��~(��������, ����) ֮����໥����, ��һ�ֲ�ͬ�������������������µ��໥����; ���dz�Ϊǿ�໥����; �����ʼ�����ֻ����໥����֮һ. ������������ʾ������, ǿ�໥���������ֻ����໥��������ǿ��. �Ա��ڵ���໥���õĽṹ����~$\alpha=\frac{e^2}{4\pi\hbar c}\approx137$, ǿ�໥�����¿���~$\frac{g^2}{4\pi\hbar c}\approx1\sim10$. ����, ǿ�໥������һ�ֶ̳���~(���������Գ�), �����þ���ԼΪԭ�Ӻ��к���֮��ľ���. ���, ǿ�໥���ñ�����������, �и���ĶԳ���, ���и�����غ���. ����ɫ����ѧ����, ��ǿ�໥���ù��Ϊ�ɿ��֮�佻�����Ӷ�����. ���¼����о�֮.

\subsection{���: ����ζ�����ֺ�, ɫ�ռ��~SU(3) �淶�任, ���ֽ���, ��-Mills �淶�������ܶ�}
~~~~
�����~6 ζ: u,~d,~s,~c,~t,~b; �����±�~$f=1,\cdots,6$ ���. ÿζ�ַ�~3 ɫ~(�������Ϸ�����, ��˹���~36 ��): R,~G,~B; �����±�~$n=1,2,3$ ���. ÿζ��˵�������̬, ���ų�һ����ά�ڲ��ռ�, ��Ϊɫ�ռ�, ���Ա�ʾΪ
\begin{align}
\psi=\psi_f=(\psi_1,\psi_2,\psi_3)^T.
\end{align}
ע����������Ϊ 1/2 �ķ�����, ��ÿ��ɫ̬~$\psi_n$ ���ǵ���������. ����~(����) ��˵������ܶȾͿ�дΪ
~$\mathcal{L}_q=\bar{\psi}(i\gamma^\mu\partial_\mu-m)\psi$; ����~$m$ ��Ȼ�ͳ��˾���, ��Ϊ��������. ע����������ɫ�޹�, �������������dz������ĵ�λ��. ���ڿ����������������, һ�ǿ�˵�ɫ����, ��һ�ǽ�������.

�ڿ��ɫ����ά���ռ��ڵı���ʸ�����Ȳ����ת������~$U$~(��~$3\times3$ �ĸ�����, ��~18 ������) �γ�~SU(3) Ⱥ, ������ģ~($\det U=1$) ����~($U^\dag U=1$) ��. �ɴ�, $U$
���������������ֻ��~8 ��. �������DZ�ɽ�~$U$ �����ð˸������Ķ��׾���--��Ϊ�Ƕ�������\footnote{SU(3) �иǶ�������ĵ�λ, ��������~SU(2) �������������ĵ�λ. �������Ǿ���д���Ƕ����������ʽ
\begin{gather}
\lambda_1=\left[\begin{array}{ccc}0&1&0\\1&0&0\\0&0&0\end{array}\right],~\lambda_2=\left[\begin{array}{ccc}0&-i&0\\i&0&0\\0&0&0\end{array}\right],
~\lambda_3=\left[\begin{array}{ccc}1&0&0\\0&-1&0\\0&0&0\end{array}\right],\nonumber\\
\lambda_4=\left[\begin{array}{ccc}0&0&1\\0&0&0\\1&0&0\end{array}\right],~\lambda_5=\left[\begin{array}{ccc}0&0&-i\\0&0&0\\i&0&0\end{array}\right],\nonumber\\
~~~~~~~\lambda_6=\left[\begin{array}{ccc}0&0&0\\0&0&1\\0&1&0\end{array}\right],~\lambda_7=\left[\begin{array}{ccc}0&0&0\\0&0&-i\\0&i&0\end{array}\right],~\lambda_8=\frac{1}{\sqrt{3}}\left[\begin{array}{ccc}1&0&0\\0&1&0\\0&0&-2\end{array}\right].
\end{gather}}--$\lambda^a$ ��ʾΪ
\begin{align}
U=e^{i\theta_a\lambda^a/2}:=e^{i\theta_a T^a},~a=1,\cdots,8.
\end{align}
�Ƕ�������������~$\textrm{tr}\lambda^a=0,~\textrm{tr}\lambda^a\lambda^b=2\delta^{ab}$. SU(3) Ⱥ���������
\begin{align}
[T_a,T_b]=if_{abc}T_c;
\end{align}
���нṹ����~$f_{abc}$ ����������ָ�궼�Ƿ��ԳƵ�. ���ſɱȺ��ʽ~$[A,[B,C]]+[B,[C,A]]+[C,[A,B]]=0$, ���Ƴ���Ⱥ�ṹ��������~$f_{ade}f_{bcd}+f_{bde}f_{cad}+f_{cde}f_{abd}=0$. ��Ⱥ�ṹ������Ϊ��ķ�����
\begin{gather}
f_{123}=1,~f_{458}=f_{678}=\frac{\sqrt{3}}{2},\\
f_{147}=-f_{156}=f_{246}=f_{257}=f_{345}=-f_{367}=\frac{1}{2}.
\end{gather}



��˳���ɫ���ǵĹ淶�任Ϊ~$\psi\rightarrow\psi'=U\psi=e^{i\theta_aT^a}\psi$. ���ɿ�˳���ɫ�ռ�����ϸ������淶������. �����������淶~(��һ������Գ�������Ϊ����
�Գ���, ����������֮Ϊ�淶����Գ���; һ������Գ���ֻ�����ϸ��, ���ܱ��淶) ����Գ���. ��������Ȼ��, ��������Ȼ����һ��ƫ΢�ֵ�Э��΢�̵ı任:
$\partial_\mu\rightarrow D_\mu=\partial_\mu+iqA_\mu$; ��Ҫ��~$D_\mu\rightarrow D'_\mu=\partial_\mu+iq A'_\mu$ ʱ,
\begin{align}
A_\mu\rightarrow A'_\mu=UA_\mu U^\dag+\frac{i}{q}(\partial_\mu U)U^\dag=UA_\mu U^\dag-\frac{i}{q}U\partial_\mu U^\dag,
\end{align}
���ǾͿɵ�~$D_\mu\psi\rightarrow D'_\mu\psi'=UD_\mu\psi=UD_\mu U^\dag\psi'$, ��
\begin{align}
D_\mu\rightarrow D'_\mu=UD_\mu U^\dag.
\end{align}
Ҳ����˵, ����, ���Ǿ͵õ���һ������淶�����������: $\mathcal{L}_{q+qg}=\bar{\psi}(i\gamma^\mu D_\mu-m)\psi=\bar{\psi}(i\gamma^\mu \partial_\mu-m)\psi-q\bar{\psi}\gamma^\mu\psi A_\mu$.


SU(3) Ⱥ����Ԫ���ڲ��ռ�ľ���, һ�㻥������, ������Ⱥ��Ϊ�ǰ�����Ⱥ. �����dz���~SU(3) ��ת�������Զ�����ij�, Ϊ�ǰ�������, ����-Mills �淶��.

������������~$A_\mu$ ��һ������. ��ǰ��~$A_\mu$ �Ĺ淶�任ȡ����, ����~$A^\dag_\mu\rightarrow A'^\dag_\mu=UA^\dag_\mu U^\dag-\frac{i}{q}U\partial_\mu U^\dag$; ���ǿɵ�~$A'^\dag_\mu-A'_\mu=U(A^\dag_\mu-A_\mu)U^\dag$. Ҳ����˵, $A^\dag_\mu-A_\mu=0$ �淶����. �����ζ��, ��������ij�淶��ȡ����~$A_\mu$ �Ƕ��׵�, �������κι淶�¶��Ƕ��׵�.


��Ϊ~$\textrm{tr}T^a=0$, ���Կɵ�~$\textrm{tr}A'_\mu=\textrm{tr}A_\mu$, ���淶���ļ��ǹ淶�����. �������ǿ�ȡ��~$\textrm{tr}A_\mu=0$, ���൱��ȡ��
\begin{align}
A_\mu=A^a_\mu T_a
\end{align}
������ͨ��ɫ��ϵ�~8 ��~��-Mills �淶��~$A^a_\mu$, ��~8 ������. �������ɫ�������໥���õ�����, ����Ϊ~QCD: ����ɫ����ѧ.



������С�淶�任��, $U=e^{i\theta_a T^a}=1+i\theta_a T^a$, ����������~$A_\mu\rightarrow A'_\mu=A_\mu+i\theta^a[T^a,A_\mu]-\frac{1}{q}\partial_\mu\theta_a\cdot T^a$; ��һ���ɵ�~
\begin{align}
A^a_\mu\rightarrow A'^a_\mu=A^a_\mu-f_{abc}\theta^bA^c_\mu-\frac{1}{q}\partial_\mu\theta^a:=A^a_\mu-\frac{1}{q}D_\mu^{ab}\theta^b,
\end{align}
����~$D_\mu^{ab}=\delta^{ab}\partial_\mu+qf_{abc}A^c_\mu$.



���������ҳ���-Mills �淶���������ܶ�. �����˹Τ�����, ��~U(1) �淶������, ���ǾͿɷ���~$[D_\mu,D_\nu]=iqF_{\mu\nu}$; ��~��-Mills ����, ���Ǽ��Ѵ˼���Ϊ�淶��~$F_{\mu\nu}$ �Ķ���:
\begin{align}
F_{\mu\nu}=\frac{1}{iq}[D_\nu,D_\nu]=\partial_\mu A_\nu-\partial_\nu A_\mu+iq[A_\mu,A_\nu].
\end{align}
������ȡ~$\textrm{tr}A_\mu=0$, ������~$\textrm{tr}F_{\mu\nu}=0$, ����������ɽ�~$F_{\mu\nu}$ ��Ϊ~$F_{\mu\nu}=F^a_{\mu\nu}T_a$. ���ǶԴ˷���������ǾͿɵ�~$F^a_{\mu\nu}=\partial_\mu A^a_\nu-\partial_\nu A^a_\mu-qf^{abc}A^b_\mu A^c_\nu$.

���ǿ�����������С�淶�任
\begin{align}
F_{\mu\nu}\rightarrow F'_{\mu\nu}=&UF_{\mu\nu}U^\dag=F^a_{\mu\nu}UT_aU^\dag=F^a_{\mu\nu}T_a+i\theta^bF^a_{\mu\nu}[T^b,T^a]\nonumber\\
=&(F^a_{\mu\nu}-f^{abc}\theta^bF^c_{\mu\nu})T^a,
\end{align}
Ҳ����˵~$F^a_{\mu\nu}\rightarrow F'^a_{\mu\nu}=F^a_{\mu\nu}-f^{abc}\theta^bF^c_{\mu\nu}$. �����, $F_{\mu\nu}$, ����~$-\frac{1}{4}F_{\mu\nu}F^{\mu\nu}$, �Dz����й淶�����Ե�; ͬʱ��ɼ����й淶�����Ե����伣, ��
\begin{align}
\mathcal{L}=-\frac{1}{2}\textrm{tr}F_{\mu\nu}F^{\mu\nu}=-\frac{1}{4}F^a_{\mu\nu}F^{\mu\nu}_a.
\end{align}
����ȡ��ʽΪ��-Mills �淶���������ܶ�.





\subsection{��-Mills ����·���������ӻ�, Faddeev-Popov ����}
~~~~
���п��ܵ�~$A_\mu$, ����������~(�����ù淶�任����ϵ) ���ɹ淶�任��ϵ�ŵ�, �ų�һ�����ռ�; �������ӿռ���Ϊ�˿ռ�ij��������. ��ν�淶����, ����~$A_\mu$ ��ȡֵ�޶���ij�������ϵķ���:
\begin{align}
G^a[A_\mu(x)]=0.
\end{align}
���ڲ�ͬ�淶������������������ǵȼ۵�, ���Դ˳���������ж��ֲ�ͬ��ѡ��, �˴����������ȼ۵�.

ѡ���˳������, ����һ��~(������), ͨ���淶�任�͵õ��ڴ˵��ϵ������������һ������. �����ϲ�ͬ��Ĺ淶���߻����ཻ. ��֮, ����������ͨ���淶�任���ɸ�ά�򲢿ռ�, ��ǰ���ᵽ�ij��ռ�.

Ϊ�˼�������˼��, Faddeev ��~Popov ����İ취, �������淶���ijɷ����в������е�ʽ:
\begin{align}
1=\Delta[A_\mu]\int\mathcal{D}U\delta[G[A^U_\mu]].
\end{align}
����, $A^U_\mu$ ����~$A_\mu$ ͨ���淶�任~$U$ �õ���~$A'_\mu$, ����~$U$ �������Ĺ淶�����ϵ�����ij��; ��~$\delta[G[A^U_\mu]]=\prod_{a,x}\delta\left(G^a[A^U_\mu(x)]\right)$ ��~$\delta$ ����. ����~Faddeev-Popov ��ʽ, ��ʵ�ϼ�~$\Delta[A_\mu]$ �Ķ���. ��Ϊ�������ֽ����һ������ʽ, ���������ֳ�~$\Delta[A_\mu]$ Ϊ~Faddeev-Popov ����ʽ. ��������ι淶�任���ѷ���, Faddeev-Popov ����ʽ�ǹ淶�����, ����
\begin{align}
\Delta[A_\mu]=~\Delta[A^U_\mu].
\end{align}

��������д������~Faddeev-Popov ��ʽ�����ɷ���:
\begin{align}
Z[J]=\int\mathcal{D}A\mathcal{D}U\Delta[A_\mu]\delta[G[A^U_\mu]]e^{iS}=\int\mathcal{D}A\Delta[A_\mu]\delta[G[A_\mu]]e^{iS}.
\end{align}
���ǽ��淶����д��
\begin{align}
G^a[A_\mu(x)]=\Omega^a[A_\mu(x)]-\omega^a(x)=0,
\end{align}
��ɵ����ɷ���Ϊ
\begin{align}
Z[J]=\int\mathcal{D}A\Delta[A_\mu]e^{i(S-\Omega^2/2\xi)}.
\end{align}
����~$-\Omega^2/2\xi$ ���ǹ淶�̶���. ������ȡ~$\Omega^a[A_\mu(x)]=\partial^\mu A_\mu^a$, �͵õ����������˹Τ������ȡ��һ��淶.

Ϊ���ܴ����ɷ�������������, ���ǻ��뽫~Faddeev-Popov ����ʽ��д��ָ����ʽ. ����������Ҫ��, Faddeev-Popov ����ʽ���뱻дΪ~����˹�����ķ�������:
\begin{align}
\Delta[A_\mu]=\int\mathcal{D}\bar{\eta}\mathcal{D}\eta e^{-i\bar{\eta}M\eta}.
\end{align}
���൱��������һ����淶����ϵ���һ�ֳ�~(��ȥ�������ɶȵĴ����ǵ�������������); ���˳���Ȼ�����˸���˹������, ȴ����������, ���Dz�ɫ��. �������ַ�����ϵ, ���ֳ�����������, ���dz�֮Ϊ~Faddeev-Popov ����. ��ֻ������Ȧ��, �������Ϊ����. ��֮, ��-Mills �淶�����������ɷ�������
\begin{align}
Z[J]=\int\mathcal{D}A\Delta[A_\mu]e^{i(S-\Omega^2/2\xi-\bar{\eta}M\eta)};
\end{align}
����
\begin{align}
\mathcal{L}_{\textrm{eff}}=\mathcal{L}_A-\frac{1}{2\xi}\Omega^2-\bar{\eta}M\eta,
\end{align}
�����Ĺ淶�̶��������, ��Ϊ~Faddeev-Popov �����ܶ�.




\subsection{BRST �Գ���}
~~~~
��Э��淶��, �����ʳ�����~Faddeev-Popov �����ܶ�Ϊ
\begin{align}
\mathcal{L}=\bar{\psi}(i\gamma^\mu D_\mu-m)\psi-\frac{1}{4}(F^a_{\mu\nu})^2-\frac{1}{2\xi}(\partial^\mu A^a_\mu)^2-\bar{\eta}^a\partial^\mu D^{ab}_\mu\eta^b.
\end{align}
���ǿ��Խ�����ڹ��Ϊ���ɹ淶�任�������һ����, �����任������Ϊ
\begin{align}
\theta^a=-g\epsilon\eta^a;
\end{align}
����~$\epsilon$ �Ǹ���˹����. ���ǹ淶�����˳��Ĺ淶�任~(����С) ����
\begin{align}
\delta A^a_\mu=\epsilon D_\mu^{ab}\eta^b,~\delta\psi=-ig\epsilon \eta^a T^a\psi.
\end{align}
Ϊ�˱����������ܶ��������任�±��ֲ���, ������Ӧ�����±任��Ϊ
\begin{align}
\delta\eta^a=\frac{1}{2}g\epsilon f^{abc}\eta^b\eta^c,~\delta\bar{\eta}^a=-\frac{\epsilon}{\xi}\partial^\mu A^a_\mu.
\end{align}
����ʽ��ǰ��ʽ, �ϳ�Ϊ~BRST �任. ������֤, �����������ܶ�, ��~BRST �任�µ�ȷ�Dz����.

��~BRST �任�µIJ�����, �ֳ�Ϊ~BRST �Գ���. ��С�������ܶȵ�~BRST �Գ���, ��һ�ֱ任����Ϊ~$\epsilon$ ������Գ���; ���а��������ʳ���淶���Ķ��򲻱���. ��֮�ɼ�, BRST �Գ����Ƕ���淶�Գ��Ե�һ����չ.
%\begin{align}
%\mathcal{L}=\overline{\psi}(i\gamma ^\mu D_\mu -m)\psi+\sum _{k=1}^{\dim (G)}\left[ \frac{-1}{4}F_{\mu\nu}^kF^{\mu \nu k}+\frac{1}{2\xi}(\partial _\mu A^{\mu
%k})^2+\sum _{i=1}^{\dim (G)}\overline{c}^k(-\partial _\mu D^{\mu ki})c^i\right] .
%\end{align}

һ���, ���ǿ��԰�����һ������~BRST �任��Ϊ~$\delta \phi=\epsilon Q\phi$; ����~$Q$ �������ڳ��ϵ����, ��Ϊ~BRST ���. �������һ���������
\begin{align}
Q^2=0;
\end{align}
���ǽ����³�Ϊ~BRST �任����������ݵ�.






\newpage
\section{����ͳһ: GWS ����; ��׼ģ��}

%CKM ���, ���ζ��, ����; �������ƺ�����.

\subsection{�����໥���õ������ܶ�: ������Ϊ��; ��ʼ����������; ��ͬλ��, ������; �²���ת��, ����Ͻ�; Gell-Mann-Nishijima ��ʽ; ��ͬλ���Ĵ�������������}
%�����ѵ�, ��~QED ������, ����໥����Ӱ�����д���ɵ�����, ~QCD ��ǿ�໥����Ӱ�����д�ɫ�ɵ�����; �Ժ󽫷���, ��~GWS ������, ���ǰ����Ӽ�����໥���ù��Ϊ������������ͬλ�������м䲣ɫ�ӵ�����.
~~~~����~$\beta$ ˥���ԭ�Ӻ�/��ԭ�����ӵķ�����˥��, ���б�����һ���ܸı����ӵ�ζ�������Ļ����໥���õIJ���. ���dz������໥����Ϊ���໥����. �����������ֻ����໥���������þ�����̵�, ����ʱ���Ƴ�Ϊ�ķ����ӵ�ֱ�ӵĶ����໥����, �м䲣ɫ���кܴ������. ����, �����ñ���Ҫ�д�����м䲣ɫ��. �ڻ���������, ���໥����Ӱ�����з�����~(����, ���), �Լ�ϣ��˹��ɫ��. ����, ���ǽ�������Ϊ��. ����֪��, 6 ��~(��Ϊ~6 ��ζ; ���Ϸ���������~12��) ���ӹ�������, ���е���~$\mu$ ����~$\tau$ �Ӵ���, ��Ӧ����΢�Ӳ�����, ��������С.

1956 ��, ��������������ָ�������ù��������~$P$ ���غ�. �Ժ����ǽ�һ����ʶ��, �������ù�����~(�Լ����κ�ʱ��), ��΢������������. ��Ϊ���ǽ�һ�����������õ�����, ָ���˷���: ��Ϊ���������������������, �������Ϸ��־���ζ��, ���໥���������ܶ������Ӳ�����������. Ҳ����˵, ���Ǽ����ʼ��������������. (��������������������; ������λ��? �⽫����һ�ڵ�ϣ��˹���Ƹ���.)  �������ǿɽ����ӵĵ��������ܶ�дΪ
\begin{align}
\mathcal{L}_{EW}=\sum_{e:=e,\mu,\tau}i(\bar{e}\gamma^\mu\partial_\mu e+\bar{\nu_e}\gamma^\mu\partial_\mu\nu_e);
\end{align}
���������Ѳ�ȡ���������е�ϰ��, ����ij�����ӵij����ñ������ӵķ��ű�ʾ. ���Ǽ���ÿ�����ӵ���������, ����һ����ά�ڲ��ռ�--��Ϊ��ͬλ���ռ�--��ʸ��, ��Ϊ
\begin{align}
L_e:=\left[\begin{array}{c}\nu_e\\e_L\end{array}\right];
\end{align}
��Ϊ��ͬλ������̬. ��Ȼ, ��ͬλ���ռ����~SU(2) Ⱥ�ĶԳ���, ��ת������/�淶�任Ϊ~$e^{i\alpha^iT^i}=e^{i\alpha^i\sigma^i/2}$; ����~$\sigma^i$ �Ƕ�ά���ռ�ת���ij���Ԫ, ����������. �����ζ��, $L$ ����ͬλ��~$T=1/2$, ��~$TL=\frac{1}{2}\sigma L$; ������ͬλ���ĵ�������, ��΢������Ӧ�ĺɵ��ӷ��������~$T^3=1/2$ ��~$-1/2$. ����, ���Ƕ���~$R_e=e_R$~(��ʵ��, ����˫��̬�뵥̬������ʱ, ����Ӧȡ~$R_e=\left[\begin{array}{c}0\\e_R\end{array}\right]$), ���������ͬλ����ֵΪ~0, ����Ϊ��ͬλ����̬. ����, ���ǿɽ����������ܶȽ�һ����Ϊ
\begin{align}
\mathcal{L}_{EW}=i\bar{L}_e\gamma^\mu\partial_\mu L_e+i\bar{R}_e\gamma^\mu\partial_\mu R_e;
\end{align}
������ȥ����ͷ���, ��ʱ����Ҳ�ɽ��ű�~$e$ ��ȥ. ��ע���ʱ~$\bar{L}=L^\dag\left[\begin{array}{cc}\gamma^0&0\\0&\gamma^0\end{array}\right]$; Ҳ����˵, ����ͬλ������̬����������������������ɵĿռ���, Ҫô��~$\gamma^\mu$ ���ɾ���ֱ���ƶ��Ĺ���, Ҫô��������~$\left[\begin{array}{cc}\gamma^\mu&0\\0&\gamma^\mu\end{array}\right]$.

�������Ǽ����������ù淶�任���о�. ���������ᵽ��~SU(2) �Գ���, ��������������~SU(1) �淶�任~$e^{i\beta Y_W/2}$ ���Dz����; ����~$Y_W$ ��Ϊ���������, ���ڶ���̬�뵥̬�ϵı���ֵ��Ϊ~$Y_W L=Y_L L,~Y_W R=Y_R R$; $Y_{L/R}$ �ľ�����ֵ, ���ǽ����Ժ�ȷ��. ���, ��Ȼ�ɼ�ͬλ���ռ��~$T_W$ �Ǽ򲢵�, ����~$[T^i,Y_W]=0$; �������ǿɵõ����໥���õ����ֹ淶�任�ɺ�дΪ
\begin{align}
U=e^{i\beta Y_W/2+i\alpha^iT^i}.
\end{align}

%����Ƕ�Ӧ����һ�ĺ�, ���ǵ�һ�ĵ���໥�����е����; �ڵ�����, ��һ��Ӧ������, �϶���Ӧ��ͬλ��. ����!

����, �������淶�����໥���õ����ֹ淶�Գ���, �༴��������淶���Ĺ���. ��Ȼ, ���Դ�д��Э��΢�̿�ʼ:
\begin{align}
D_\mu=\partial_\mu+igT^iW_\mu^i+i\frac{1}{2}g'Y_WB_\mu.
\end{align}
�����, Э��΢����������ͬλ����̬�����̬�Ϸֱ�Ϊ
\begin{align}
D_\mu R_e=&\left(\partial_\mu+i\frac{1}{2}g'Y_R B_\mu\right)R_e,\\
D_\mu L_e=&\left(\partial_\mu+ig\frac{1}{2}\sigma^iW_\mu^i+i\frac{1}{2}g'Y_L B_\mu\right)L_e\nonumber\\
=&\partial_\mu L_e+\frac{i}{2}\left[\begin{array}{cc}gW_\mu^3+g'Y_LB_\mu&gW_\mu^1-igW_\mu^2\\gW_\mu^1+igW_\mu^2&-gW_\mu^3+g'Y_LB_\mu\end{array}\right]\left[\begin{array}{c}\nu_e\\e_L\end{array}\right]\nonumber\\
:=&\partial_\mu L_e+\frac{i}{2}\left[\begin{array}{cc}gW_\mu^0+g'Y_LB_\mu&\sqrt{2}gW^+_\mu\\\sqrt{2}gW^-_\mu&-gW_\mu^0+g'Y_LB_\mu\end{array}\right]\left[\begin{array}{c}\nu_e\\e_L\end{array}\right].
\label{weinberg}
\end{align}
���Կ���, ��ʽ��
\begin{align}
W^+_\mu:=\frac{1}{\sqrt{2}}(W_\mu^1-iW_\mu^2),~
W^-_\mu:=\frac{1}{\sqrt{2}}(W_\mu^1+iW_\mu^2)
\end{align}
�Ǿ�����ȷ�����������, ��Ϊ�������໥���õ�����~(�����) �м䲣ɫ��. ͬʱ��ɿ���, $W^0_\mu:=W^3_\mu$ ��~$B_\mu$ ��Ϊͬʱ����΢��������������, ���Բ����ܱ�ڹ��Ϊ�����û������ø��ԵĹ淶��. ���������Ӳ������, ����, �ص���~$W^0_\mu$ ��~$B_\mu$ �ĵ���; ���Ǽ���˵���Ϊһ��ת��:
\begin{align}
\left[\begin{array}{c}A_\mu\\Z^0_\mu\end{array}\right]=\left[\begin{array}{cc}\cos\theta_W&\sin\theta_W\\-\sin\theta_W&\cos\theta_W\end{array}\right]
\left[\begin{array}{c}B_\mu\\W^0_\mu\end{array}\right].
\end{align}
��ʽ��Ϊ�²���ת��, ��~$\theta_W$ ��Ϊ����Ͻ�, ���²����. ���������, ����Ҫȷ���´˽�. ���ѿ���, ��ԭʽ~(\ref{weinberg}) ��, ��~$A_\mu$ ������΢����ϵ����Ǿ��������Ͻ�һ��; �������²���ת�������
\begin{align}
B_\mu=&\cos\theta_W\cdot A_\mu-\sin\theta_W \cdot Z^0_\mu,\\
W^0_\mu=&\sin\theta_W\cdot A_\mu+\cos\theta_W \cdot Z^0_\mu,
\end{align}
�Ϳ����������
\begin{align}
gW_\mu^0+g'Y_LB_\mu=(g\sin\theta_W+g'Y_L\cos\theta_W)A_\mu+(g\cos\theta_W-g'Y_L\sin\theta)Z^0_\mu.
\end{align}
��Ȼ, Ϊ�˰�~$A_\mu$ ڹ��Ϊ����໥���õ��м䲣ɫ��~(����~$Z^0_\mu$ �ͳ��������õĵ������м䲣ɫ��, �Dz������), ���DZ���Ҫ��������΢����û����ϵ�, ��Ҫ��~$g\sin\theta_W+g'Y_L\cos\theta_W=0$; ���ǿɵ�
\begin{align}
\sin\theta_W=\frac{-g'Y_L}{\sqrt{g^2+g'^2Y_L^2}},~\cos\theta_W=\frac{g}{\sqrt{g^2+g'^2Y_L^2}}.
\end{align}
�²���ǵľ�����ֵ, ������ʵ��ó�; ���²�ֵԼΪ~$\sin^2\theta_W=0.2223(21)$. ͨ���򵥵ĵ���, �͵õ��������Ĺ淶��, �²���ת��ʵ������, ����!

����, ������������ǰ�������໥�����еĵ�Ų���; ���~(\ref{weinberg}) �о�������½�һ��:
\begin{align}
D_\mu=&\partial_\mu+ig'\frac{1}{2}Y_W\cos\theta_W A_\mu+igT^3\sin\theta_W A_\mu\nonumber\\
=&\partial_\mu+ig'\cos\theta_W(\frac{Y_W}{2}-Y_LT^3)A_\mu.
\end{align}
��������֪�ĵ�����õ�Э��΢�����Ƚ�, ���ǿɵ�~$g'\cos\theta_W:=e$ �ǻ������; ��Բ�����е�һ�������:
\begin{align}
Q=-Y_LT^3+\frac{Y_W}{2}.
\end{align}
��ʽ��ǿ�໥�����е�~Gell-Mann-Nishijima ��ʽ������ͬ����ʽ. Ҫ��������������ڵ�̬�����̬������ȷ�ĵ��������, ��~$QR=Y_R/2=-1,~QL=Q\left[\begin{array}{c}\nu_e\\e_L\end{array}\right]=
\left[\begin{array}{c}(-Y_L\frac{1}{2}+\frac{Y_L}{2})\nu_e\\(Y_L\frac{1}{2}+\frac{Y_L}{2})e_L\end{array}\right]=\left[\begin{array}{c}0\cdot\nu_e\\1\cdot e_L\end{array}\right]$, ���ǾͿɵó�
\begin{align}
Y_L=-1,~Y_R=-2.
\end{align}
�ݴ�, ���ǿɽ�ǰ�����г���~$Y_L,~Y_R$ ��ʽ�ӽ��л���, ����~$Q=T^3+\frac{Y_W}{2}$, ����
\begin{align}
e=g'\cos\theta_W=\frac{gg'}{\sqrt{g^2+g'^2}}=g\sin\theta_W.
\end{align}
����, ���ǿɽ�һ����ʽ~(\ref{weinberg}) дΪ
\begin{align}
D_\mu L_e=&\partial_\mu L_e+\frac{i}{2}\left[\begin{array}{cc}\sqrt{g^2+g'^2}Z^0_\mu&\sqrt{2}gW^+_\mu\\\sqrt{2}gW^-_\mu&-2eA_\mu+\frac{g'^2-g^2}{\sqrt{g^2+g'^2}}Z^0_\mu\end{array}\right]\left[\begin{array}{c}\nu_e\\e_L\end{array}\right].
\end{align}
д�����ж���淶����, ��������淶����ϵ����������ܶ�, ���Ǽ򵥲���������:
\begin{align}
\mathcal{L}_{\textrm{free+int}}=i\bar{L}_e\gamma^\mu D_\mu L_e+i\bar{R}_e\gamma^\mu D_\mu R_e=\mathcal{L}_{\textrm{free}}+\mathcal{L}_{\textrm{int}};
\end{align}
�����໥���ò���Ϊ
\begin{align}
\mathcal{L}_{\textrm{int}}=&-\frac{1}{2}(\nu_e^\dag,e_L^\dag)\left[\begin{array}{cc}\gamma^0&0\\0&\gamma^0\end{array}\right]
\left[\begin{array}{cc}\sqrt{g^2+g'^2}Z\!\!\!\!/~^0 &\sqrt{2}gW\!\!\!\!\!/~^+\\\sqrt{2}gW\!\!\!\!\!/~^-&-2eA\!\!\!/+\frac{g'^2-g^2}{\sqrt{g^2+g'^2}}Z\!\!\!\!/~^0\end{array}\right]
\left[\begin{array}{c}\nu_e\\e_L\end{array}\right]\nonumber\\
&+\bar{R}_e\left(eA\!\!\!/-\frac{g'^2Z\!\!\!\!/~^0}{\sqrt{g^2+g'^2}}\right)R_e=\mathcal{L}_{eA}+\mathcal{L}_{lW}+\mathcal{L}_{lZ};
\end{align}
���ı����ζ�����໥���������ܶȿɷ�Ϊ��������, ����-$W$ ������Լ�����-$Z$ �����; �������Ƿֱ�д��. ��������Ϊ
\begin{align}
\mathcal{L}_{eA}=e\bar{e}_L\gamma^\mu A_\mu e_L+e\bar{e}_R\gamma^\mu A_\mu e_R=e\sum_{l=e,\mu,\tau}\bar{\psi}_l\gamma^\mu A_\mu \psi_l,
\end{align}
����ȫ����������~QED �еĽ���. ������~$W^\pm$ ��ϵ���Ϊ
\begin{align}
\mathcal{L}_{lW}=-\frac{g}{\sqrt{2}}\sum(\bar{\nu}_e\gamma^\mu W_\mu^+ e_L+\bar{e}_L\gamma^\mu W_\mu^-\nu_e).
\end{align}
���������ҳ��������غ���, ˳������ǰ��~QED ������������, �����������ܶ�������Ϊ����淶����˵���ʽ. Ӧ��ŵ�ض������غ�������ʽ~(\ref{noether cu.}), ���ǿ���������ӳ����غ�ʸ����Ϊ
\begin{align}
j^{\mu i}=\sum \bar{L}_e\gamma^\mu\sigma^iL_e.
\end{align}
������~$j^{\mu\pm}=j^{\mu1}\pm ij^{\mu2},~\sigma^{\pm}=(\sigma^1\pm i\sigma^2)/2$, �����ǿɵ�
\begin{align}
j^{\mu\pm}=\sum 2\bar{L}_e\gamma^\mu\sigma^\pm L_e.
\end{align}
�ɴ�~$\mathcal{L}_{lW}$ �Ϳɽ�һ����Ϊ
\begin{align}
\mathcal{L}_{lW}=-\frac{g}{2\sqrt{2}}\sum(j^{\mu+}W_\mu^++j^{\mu-}W_\mu^-).
\end{align}
���������о�������, �����������Ȳ¶�����������������ʽ��������. ��������д��~$\mathcal{L}_{lZ^0}$ ��:
\begin{align}
\mathcal{L}=&-\frac{1}{2}\left( \sqrt{g^2+g'^2}\bar{\nu}_e\gamma^\mu\nu_eZ_\mu^0+ \frac{g'^2-g^2}{\sqrt{g^2+g'^2}}\bar{e}_L\gamma^\mu e_L Z_\mu^0
+ \frac{2g'^2}{\sqrt{g^2+g'^2}}\bar{e}_R\gamma^\mu e_RZ_\mu^0\right)\nonumber\\
=&-\frac{1}{2}\left( \sqrt{g^2+g'^2}\bar{L}_e\gamma^\mu\sigma^3L_eZ^0_\mu+\frac{2g'^2}{\sqrt{g^2+g'^2}}\bar{e}\gamma^\mu eZ^0_\mu\right)\nonumber\\
=&-\frac{1}{2}\frac{g}{\cos\theta_W}\sum\left(\bar{L}_e\gamma^\mu\sigma^3L_e+2\sin^2\theta_W\bar{e}\gamma^\mu e\right)Z^0_\mu;
\end{align}
%����, ���Ƕ�˭��������ͬ��ֵ. ����, һ��������ѧ�����, ��������Ϊ����, �ֿ��Լ�ָ�����������ڲ�ͬ����̬�ϵ�ֵ�Ƕ���. �����ַ�����һ�µ�.
%������������, �����ı���, һ������. �����ֱ���������ɫ�����.
ǰ��������Ԥ��~$Z^0_\mu$ �Ǵ��������õIJ�����IJ�ɫ��, ������ʽ���Ե�֤: ��������~$Z^0_\mu$ ������������������������΢�����, ����΢�Ӳ�����, �������ܵ������, ����~$Z^0_\mu$ ֻ���Ǵ��������õ��м䲣ɫ��. ����, �����������
\begin{align}
j^{\mu0}:=\bar{L}_e\gamma^\mu\sigma^3L_e+2\sin^2\theta_W\bar{e}\gamma^\mu e
\end{align}
��Ϊ���ӳ���~$Z^0_\mu$ ���Բ�ɫ����ϵ���������. ��������������������Ԥ��, ������ʵ��֤ʵ��. ��һ����, �����ʹ�����ǶԱ�����ͳһ���۵Ľ���.


ǰ���Ѿ��ᵽ��, ���ڵ�����, ���Ǽ����˲���������õij�ʼ������淶���Ӷ�����������. ��ͬ�ڴ˼���, ��ʵ��, �������ӵ�����, ���������õ��м䲣ɫ�Ӷ�����������; ���Һ��ߵ�������ʮ�ֵش�~(����淶�任���ǻ������ǰ�������ò�ɫ��, ���Ǻ��ߵĵ���, ������ʵ���м䲣ɫ����������淶���ӱ����޾�����������Ƶ�ì��, �͵��Խ⿪). ��������ǻ������, ���Ǽ�������һ����̽��.



\subsection{����淶�Գ��Ե��Է���ȱ, ϣ��˹��, Goldstone ����; ������ȱ, ϣ��˹����: �淶���Ե�~Goldstone ��ɫ�ӻ������; $W^\pm,~Z^0$ �Լ����ӵ������Ļ��}

~~~~
��������~$\phi=\chi_1+i\chi_2$ ��~$\phi^4$ ģ��Ϊ
\begin{align}
\mathcal{L}=\partial_\mu\phi^\dag\partial^\mu\phi-m^2\phi^\dag\phi-\lambda(\phi^\dag\phi)^2:=\partial_\mu\phi^\dag\partial^\mu\phi-V(\phi,\phi^\dag);
\end{align}
��ʽ��Ȼ�Ǿ�������淶~$\phi\rightarrow\phi'=e^{i\gamma}\phi$ �����Ե�. ���Ļ�̬/��ն�Ӧ�����ܵļ�С:
\begin{align}
\frac{\partial}{\partial\phi^\dag}V=m^2\phi+2\lambda\phi(\phi^\dag\phi)=0.
\end{align}
��~$m^2>0$, ��ʽ�����ļ�С��Ϊ~$\phi=\phi^\dag=0$, ������ڸ�ƽ���������, ��Ωһ��. ����~$m^2<0$, ��ƽ��������㷴���������ܼ���, �����dz������; ������մ�����
\begin{align}\label{va}
|\phi|^2=\chi_1^2+\chi_2^2=-\frac{m^2}{2\lambda}:=a^2,
\end{align}
��~$|\phi|=a$. Ҳ����˵, �ิƽ��������Ϊ~$a$ ��Բ���ϵ����е�, ���dz������; ������մ���״���������޼򲢵�.

�������Ը�����ն��ǶԳƵ�; �����������, ֻ���������е�һ��, ���پ��й淶�任~(��λ�任, ����ƽ���ڵ�ת��). ���ǰ��������, ��Ϊ��~$\phi$ ��~(��յ�) �Է��Գ���ȱ.


����, ʽ~(\ref{va}) �����, ��������ڴ�ֵΪ~$|\langle0|\phi|0\rangle|^2=a^2$. ��������ڴ��ڲ�Ϊ��ij�, ���dz�֮Ϊϣ��˹��.

\begin{figure}[!h]
\begin{center}
\includegraphics[width=5.0 cm]{pic/mexican.jpg}
\caption{``ī����ñ'' ��: ��������~$\phi^4$ ģ�͵��Է��Գ���ȱ.}
\label{Casimir}
\end{center}
\end{figure}

ʵ���ϲ⵽�ĸ�������, ���dz�����ջ����ϵļ���; ���������������Ļ�̬~$\phi=a$ �������, ����
\begin{align}
\phi=a+\frac{1}{\sqrt{2}}(h+i\rho).
\end{align}
Ҳ����˵, ��ʱ~$h,~\rho$ ���ǿ��Թ۲⵽�������ij���. ������ʽ�Ӵ���ԭʼ�����ܶ�, �ɵ�
\begin{align}
\mathcal{L}=\frac{1}{2}(\partial_\mu h)^2+\frac{1}{2}(\partial_\mu \rho)^2-\lambda v^2h^2-\lambda vh(h^2+\rho^2)-\frac{\lambda}{4}(h^2+\rho^2)^2;
\end{align}
����~$v:=\sqrt{2}a$. �����Ƶ���, Ӧ����ʽ~(\ref{va}), ����ȥ�˳�����. ����ʽ��Ȼ�ɼ�, ��~$h$ ��������
\begin{align}
m_H=\sqrt{2\lambda}v;
\end{align}
ϣ��˹������������, ��Ϊϣ��˹����. ����, �����Կ���, ��~$\rho$ ������. һ���, һ�������Գ��Ե��Է���ȱ, �ᵼ����������/���ӵĴ���, ���Ϊ~Goldstone ����; ��Ӧ�ij�/���Ӿͳ�Ϊ~Goldstone ��/����. ע��˴�ֻ��~Goldstone ������һ��������, ����֤�����ǽ����Ժ����.

��Ȼ����ģ����~$m^2<0$ Ϊ���õ�������Գ��Ե��Է���ȱ, �����Ա��������ڲ��ռ��ʸ��, ��������ȴ�ض��ǷǸ���. ������~$\phi$ ��ij~$N$ ά�ڲ��ռ����ʸ, �������ܶ�һ���дΪ~$\mathcal{L}=\partial_\mu\phi^\dag\partial_\mu\phi-V(\phi)$; ������~$\phi=\phi_0$ ���м�Сֵ, ��~$\frac{\partial V}{\partial\phi_a}\Big|_{a}=0$. ��ô, ���ǿ��ڴ���ո����ļ���, ����
\begin{align}
V(\phi)=&V(\phi_0)+\frac{1}{2}\frac{\partial^2V}{\partial_\mu\phi_a\partial\phi_b}\Big|_{\phi=\phi_0}\delta\phi_a\delta\phi_b+O(\delta\phi^3)\\
:=&V(\phi_0)+\frac{1}{2}m_{ab}\delta\phi_a\delta\phi_b+O(\delta\phi^3).
\end{align}
ֻҪ����ղ����������κ�һ��ƫ��, �����ܱ仯������ֵ, ��������ʽ�ɼ�����~$m_{ab}\geq0$. ����, �����������о��ڲ��ռ�ʸ������~Goldstone ����. ����~$\phi$ ��ij~$N$ ά�ڲ��ռ����ʸ, �������й淶�任~$\phi'=e^{i\theta_a T^a}\phi$. ����, ��������մ��Դ˹淶�任����չ��~(������֮ǰ�ij�����յ�����ƫ��), ����
\begin{align}
V(\phi_0)=V(\phi'_0)=V(\phi_0)+\frac{1}{2}m_{ab}\delta\phi_a\delta\phi_b+\cdots;
\end{align}
���ǿɵ�~$m_{ab}\delta\phi_a\delta\phi_b=0$. ����ζ��, ��~$\delta\phi_a,~\delta\phi_b$ �Բ�Ϊ��, �����~$m_{ab}=0$. ��~$\delta\phi_a$ ��Ϊ��, �⼴������ͬ��̬��Ϊ���, ��˵����շ�����, �Գ����Է���ȱ. �Ӷ�����ȷ֤, �Գ����Է���ȱ, �ص������������ӵĴ���. �˼�~Goldstone ����. ����, �����������̻�����, ��������Ϊ���ά��, ������շ����Է��Գ���ȱ��ά��, ͬʱҲ����~Goldstone ���ӵ���Ŀ~$N_G$; ��������������, ��ϣ��˹�ӵ���Ŀ,  һ��ؼ�~$N-N_G$.

ǰ���������۵�, ������淶�Գ��Ե��Է���ȱ, �����������۶���淶�Գ��Ե��Է���ȱ; �����ֻ��Ϊ�Է��淶��ȱ. �ȿ��ǰ������淶�����; �������Ը�������Ϊ��. ����淶����~$\phi^4$ �����ú�ľ��ж���淶�Գ��Ե������ܶ�Ϊ
\begin{align}
&\mathcal{L}=D_\mu\phi^\dag D^\mu\phi-m^2\phi^\dag\phi-\lambda(\phi^\dag\phi)^2-\frac{1}{4}F_{\mu\nu}F^{\mu\nu}\nonumber\\
=&\partial_\mu\phi^\dag\partial^\mu\phi-m^2\phi^\dag\phi-\lambda(\phi^\dag\phi)^2-iq\phi^\dag\overset{\leftrightarrow}{\partial}_\mu\phi A^\mu+q^2A^2\phi^\dag\phi-\frac{1}{4}F_{\mu\nu}F^{\mu\nu};
\end{align}
��ѡ��~$\phi=a+\frac{1}{\sqrt{2}}(h+i\rho)$, �������ʽ�Ϳɵ�
\begin{align}
\mathcal{L}=&\frac{1}{2}(\partial_\mu h)^2+\frac{1}{2}(\partial_\mu \rho)^2-\lambda v^2h^2-\frac{1}{4}F_{\mu\nu}F^{\mu\nu}+\frac{1}{2}q^2v^2A^2\nonumber\\
&-\lambda vh(h^2+\rho^2)-\frac{\lambda}{4}(h^2+\rho^2)^2+qv\partial_\mu\rho A\nonumber\\
&+qh\overset{\leftrightarrow}{\partial}_\mu\rho A^\mu+q^2vhA^2+\frac{1}{2}q^2(h^2+\rho^2)A^2.
\end{align}
����ʽ�ɶ���, $h,~\rho$ �����������淶������һ����; ������Ϣ��, �淶�����������~$qv$. ����ʽ�ӵڶ������һ��, ����~Goldstone ��ɫ������������. ��ʵ��, ͬ��Ϊ�������Թ����˶�������, Goldstone ��ɫ������ʱ�Dz��ɷֱ��; ����˵�����˵, Goldstone ��ɫ�ӽ��ᱻ���ӳԵ�. ��ѧ��, ����ֱ��������Ϊ��ķ�ʽ������ȥ, ��:
\begin{align}
\mathcal{L}=&\frac{1}{2}(\partial_\mu h)^2-\lambda v^2h^2-\frac{1}{4}F_{\mu\nu}F^{\mu\nu}+\frac{1}{2}q^2v^2A^2\nonumber\\
&-\lambda vh^3-\frac{\lambda}{4}h^4+q^2vhA^2+\frac{1}{2}q^2h^2A^2.
\end{align}
�ɴ�, ���dz����Է��淶��ȱ�����, �淶����ͨ���Ե�~Goldstone ��ɫ��/��ϣ��˹��ɫ�����, �����������. ��������ʹ�淶���ӻ�������Ļ���, �а���ɭ-ϣ��˹����. ��ʵ��, ʹ~Goldstone ��ɫ����ȥ, �൱��ѡȡ���ض��Ĺ淶; ���dz�������ѡ�����, �������淶�������淶. ����һ����ȥ~Goldstone ��ɫ����淶�������~$qv\partial_\mu\rho A$ ������, ��Ϊ~$R_\xi$ �淶, ���Ϊ~'t Hooft �淶, ��������~$\xi$ �淶, ������Ϊ
\begin{align}
\Omega[A_\mu]=\partial_\mu A^\mu-\xi qv\rho;
\end{align}
��˿ɵ�
\begin{align}
-\frac{1}{2\xi}(\partial_\mu A^\mu-\xi qv\rho)^2+qv\partial_\mu\rho A=-\frac{1}{2\xi}(\partial_\mu A^\mu)^2+\frac{1}{2}\xi q^2 v^2\rho^2+qv\partial_\mu(\rho A).
\end{align}
��ʽ���һ����ȫɢ����, ����ֱ����ȥ. �˹淶��, Goldstone ��ɫ����淶�����������ȥ��, ��ǰ������; ������Ϊ������~$\sqrt{\xi}qv$ �Ƿ�������, ������������ȥ. ~$R_\xi$ �淶�ij������ڸ���ʱ�ļ����Լ�����������֤��.



�������ǶԷ������˵һ˵��, �Լ��Ժ������. ��ǰ���Է��淶��ȱΪ��, ����һ�²���ȷ֤, ��ȥ~Goldstone ��ɫ�ӵ������淶, �൱����һ��ʼ��ѡ��
\begin{align}
\phi=\frac{1}{\sqrt{2}}(v+h).
\end{align}
�������������ֱ�ӳ���ʽΪ����/�����淶.

����, �������о���-Mills �淶��������. ���ǿ��Ǿ��и���ά�ڲ��ռ�~(��ͬλ���ռ�) �ĸ�������~$\phi=(\phi^+,\phi^0)^T$. ��~$m^2<0$ ʱ, �˳����������
\begin{align}\label{va}
|\phi|^2=\chi_1^2+\chi_2^2+\chi_3^2+\chi_4^2=-\frac{m^2}{2\lambda}=a^2;
\end{align}
��ʽ���ɲ�����~3 ��, ��ζ�ű������Է��Գ���ȱ������~Goldstone ��ɫ����~3 ��. ��ǰ������, ѡ��
\begin{align}
\phi=\frac{1}{\sqrt{2}}\left[\begin{array}{c}0\\v+H\end{array}\right]
\end{align}
�Ϳ���������~Goldstone ��ɫ�ӱ������õ������淶���Ե�. ���ǿɵ�~$\phi$ ������淶����~$\phi^4$ �����ú�ľ��ж���淶�Գ��Ե������ܶȼ������Ϊ
\begin{align}
&\mathcal{L}=D_\mu\phi^\dag D^\mu\phi-m^2\phi^\dag\phi-\lambda(\phi^\dag\phi)^2-\frac{1}{4}F^i_{\mu\nu}F_i^{\mu\nu}\nonumber\\
=&\partial_\mu\phi^\dag\partial^\mu\phi-i\phi^\dag\overset{\leftrightarrow}{\partial}_\mu\phi A^\mu+q^2A^2\phi^\dag\phi-m^2\phi^\dag\phi-\lambda(\phi^\dag\phi)^2-\frac{1}{4}F^i_{\mu\nu}F_i^{\mu\nu}\nonumber\\
=&\frac{1}{2}(\partial_\mu h)^2+\frac{1}{2}g^2(v+h)^2A^2-\frac{\lambda}{4}(h^4+4vh^3+4v^2h^2-v^4)-\frac{1}{4}F^i_{\mu\nu}F_i^{\mu\nu}\nonumber\\
=&-\frac{1}{4}F^i_{\mu\nu}F_i^{\mu\nu}+\frac{1}{2}g^2v^2A^2+\frac{1}{2}(\partial_\mu h)^2-\lambda v^2h^2\nonumber\\
&-\lambda vh^3-\frac{1}{4}\lambda h^4+g^2vhA^2+\frac{1}{2}g^2h^2A^2.
\end{align}
����~$D_\mu=\partial_\mu+igT^iW_\mu^i:=\partial_\mu+igA_\mu$. --Ϊ���������, Ŀǰ������δ����~U(1) ����; ���ǽ����Ժ���������. --���������һ���ǹ淶������ϣ��˹��, �ڶ���ǰ������ϣ��˹�ӵ������, ����������淶�������. �������, �淶����Ȼ���������. �������ǿ�������~Goldstone ��ɫ�ӵĴ������. ���Ǽ��賡����������~$Y_W\phi=Y_H\phi$, ��Ӧ��~Gell-Mann-Nishijima ��ϵ, �ɵ�
\begin{align}
Q\phi=\left(T^3+\frac{Y_W}{2}\right)\phi=\left(T^3+\frac{Y_H}{2}\right)\phi=\frac{1}{2}\left[\begin{array}{c}(1+Y_H)\phi^+\\(-1+Y_H)\phi^0\end{array}\right]
\end{align}
��ղ�����, ��������ʽ�ɵ�~$Y_H=1$; ��һ���ɵ�~$\phi^+$ �е��~$+1$. Ҳ����˵, ���������м䲣ɫ�ӳԵ�������~Goldstone ��ɫ��, ������������, һ��������; ��ʣ�µ����һ�����Ӽ�ϣ��˹��ɫ��, �ǵ����Ե�.

����, ����д��~U(1)$\otimes$SU(2) Э�����~$U=e^{i\beta Y_W/2+i\alpha^iT^i}$ �����ڳ�~$\phi=(\phi^+,\phi^0)^T$ �ϵ�������ʽ:
\begin{align}
D_\mu\phi=&\partial_\mu\phi+\frac{i}{2}\left[\begin{array}{cc}gW_\mu^0+g'Y_HB_\mu&\sqrt{2}gW^+_\mu\\\sqrt{2}gW^-_\mu&-gW_\mu^0+g'Y_HB_\mu\end{array}\right]
\left[\begin{array}{c}\phi^+\\\phi^0\end{array}\right]\nonumber\\
=&\partial_\mu\phi+\frac{i}{2}\left[\begin{array}{cc}gW_\mu^0+g'B_\mu&\sqrt{2}gW^+_\mu\\\sqrt{2}gW^-_\mu&-gW_\mu^0+g'B_\mu\end{array}\right]
\left[\begin{array}{c}\phi^+\\\phi^0\end{array}\right]\nonumber\\
=&\partial_\mu\phi+\frac{i}{2}\left[\begin{array}{cc}2eA_\mu-\frac{g'^2-g^2}{\sqrt{g^2+g'^2}}Z^0_\mu&\sqrt{2}gW^+_\mu\\\sqrt{2}gW^-_\mu&-\sqrt{g^2+g'^2}Z^0_\mu\end{array}\right]\left[\begin{array}{c}\phi^+\\\phi^0\end{array}\right]\nonumber\\
:=&\partial_\mu\phi+\frac{i}{2}\mathscr{D}_\mu\phi;
\end{align}
����~$\mathscr{D}_\mu^\dag=\mathscr{D}_\mu$. �������ǿɵ�
\begin{align}
(D_\mu\phi)^\dag D^\mu\phi=\partial_\mu\phi^\dag\partial^\mu\phi+\textrm{Im}(\phi^\dag\mathscr{D}_\mu\partial^\mu\phi)+\frac{1}{4}\phi^\dag\mathscr{D}_\mu\mathscr{D}^\mu\phi.
\end{align}
��ǰ��ѡ�������/Goldstone ��ɫ�ӷ���ʽ, ��ͬλ���ռ�������淶������ʽ, �Ϳɵó����Ļ�������Լ��淶��������ij���; �����һ����~$\partial_\mu\phi^\dag\partial^\mu\phi=\frac{1}{2}(\partial_\mu H)^2$ ��ϣ��˹���Ķ����ܶ�; ���߲���. ����, �淶���������������ʽ���һ��������~$(0,v/\sqrt{2})^T$ ������֮��. ����
\begin{align}
&\frac{1}{4}\left[\begin{array}{c}0\\\frac{v}{\sqrt{2}}\end{array}\right]^\dag\mathscr{D}^\mu\mathscr{D}_\mu\left[\begin{array}{c}0\\\frac{v}{\sqrt{2}}\end{array}\right]\nonumber\\
=&\frac{1}{4}\frac{v}{\sqrt{2}}\left[\begin{array}{c}\sqrt{2}gW^{+^\mu}\\-\sqrt{g^2+g'^2}Z^{0\mu}\end{array}\right]^\dag
\left[\begin{array}{c}\sqrt{2}gW^+_\mu\\-\sqrt{g^2+g'^2}Z^0_\mu\end{array}\right]\frac{v}{\sqrt{2}}\nonumber\\
=&\frac{1}{4}g^2v^2W^{-\mu}W^+_\mu+\frac{1}{8}(g^2+g'^2)v^2Z^{0\mu}Z^0_\mu.
\end{align}
�������Ǿ͵�
\begin{align}
m_W=\frac{1}{2}gv,~m_Z=\frac{1}{2}\sqrt{g^2+g'^2}v=\frac{m_W}{\cos\theta_W}.
\end{align}
������������, ����ʵ��ó�. ֵ��ע�����, �ԼӼ��㲻�ѷ���, ����������һ��������, ����û�ܲ���������, ��Ȼ���־�����Ϊ��. ����ؿ���˵, ����~Goldstone ��ɫ�ӷֱ�������õ������淶���ӳԵ���; ����û����, ����û�ܻ������. ��֮, �û�������Ļ����, ���û��������û���: ���ǵ����۵�ȷ��ǡ����ݵ�.% ͨ������, ��������;
~��˽��, ��������!



%\begin{align}
%\mathscr{D}_\mu\left[\begin{array}{c}0\\\frac{v}{\sqrt{2}}\end{array}\right]=&\left[\begin{array}{cc}2eA_\mu-\frac{g'^2-g^2}{\sqrt{g^2+g'^2}}Z^0_\mu&\sqrt{2}gW^+_\mu\\\sqrt{2}gW^-_\mu&-\sqrt{g^2+g'^2}Z^0_\mu\end{array}\right]
%\left[\begin{array}{c}0\\\frac{v}{\sqrt{2}}\end{array}\right]\nonumber\\
%=&\left[\begin{array}{c}\sqrt{2}gW^+_\mu\\-\sqrt{g^2+g'^2}Z^0_\mu\end{array}\right]\frac{v}{\sqrt{2}}
%\end{align}

�������, ������������������. ��;��, ��������ȥ��~Goldstone ��ɫ��, ����ϣ��˹���������. ����������ϣ��˹��֮�����������, ���д��
\begin{align}
\mathcal{L}_{lH}=-\sum_{e,\mu,\tau} g_e(\bar{L}_e\phi R_e+\bar{R}_e\phi^\dag L_e);
\end{align}
��Ȼǰ�����Ϊ���׹���. �������淶������ʽ, �͵�
\begin{align}
\mathcal{L}_{lH}=&-\sum_{e,\mu,\tau} g_e\bar{e}e\phi^0:=-\sum_{e,\mu,\tau}\left(m_e\bar{e}e+\frac{m_e}{v}\bar{e}eH\right);
%=-\frac{1}{\sqrt{2}}\sum g_e\left[\bar{L}_e(v+H)R_e+\bar{R}_e(v+H)L_e\right]\nonumber\\
\end{align}
�ɴ�ʽ��ɼ�, ��΢��δ����������, Ҳδ��ϣ��˹�������. ���������ӻ��������, Ϊ
\begin{align}
m_e=\frac{g_ev}{\sqrt{2}}.
\end{align}





%ע��: R_e ��ʱ���ǰ�ά��, ��ʱ��������ά��, ��������~~~~~~~~~~~~~~~~`��ʵR_e Ӧ�ǰ�ά, e_R ������ά. �����������ȽϺ�. ���е�ʱ���ֲ�������. ��Ϊ�������������, ������һ��������ά.



%��ָ���໥���õ�һЩ�����ص�.
%ѡ�淶, ����ѡ��λ. ѡ�����ĸ��ط�����.
\subsection{��˵�������, CKM ����}

~~~~
����������~$(\nu_e,e),~(\nu_\mu,\mu),~(\nu_\tau,\tau)$ ���Ƶ�, ���Ҳ��Ϊ����~$(u,d),~(c,s),~(t,b)$. ��˹��ɵ���ͬλ��˫��̬�뵥̬����
\begin{gather}
L_u=\left[\begin{array}{c}u_L\\d_L\end{array}\right],~L_c=\left[\begin{array}{c}c_L\\s_L\end{array}\right],~L_t=\left[\begin{array}{c}t_L\\b_L\end{array}\right];\\
R_u=u_R,~R_c=c_R,~R_t=t_R,\\
R_d=d_R,~R_s=s_R,~R_b=b_R.
\end{gather}
����, $u,~c,~t$ ��˾��е��~$2/3$; $d,~s,~b$ ��˾��е��~$-1/3$. ����, ��~Gell-Mann-Nishijima ��ʽ, ���Ǿ�֪
\begin{align}
QL_u=\left(T^3+\frac{Y_W}{2}\right)\left[\begin{array}{c}u_L\\d_L\end{array}\right]=\left[\begin{array}{c}
\left(\frac{1}{2}+\frac{Y_W}{2}\right)u_L\\\left(-\frac{1}{2}+\frac{Y_W}{2}\right)d_L\end{array}\right],
\end{align}
Ҳ����˵�����ͬλ������̬����������~$Y_{uL}=\frac{1}{3}$. ���Ƶ�, $QR_u=\frac{Y_W}{2}R_u=\frac{2}{3},~QR_d=\frac{Y_W}{2}R_d=-\frac{1}{3}$, ��~$u,~c,~t$ ��̬����������~$Y_{uR}=\frac{4}{3}$, ��~$d,~s,~b$ ��̬����������~$Y_{dR}=-\frac{2}{3}$.

ĿǰΪֹ, ���Ǽ�����������������. ����, �����ÿ����ϣ��˹�����, ��ʹ֮�������. �����ϣ��˹����������Ͽ�дΪ
\begin{align}
\mathcal{L}_{qH}=-\sum_{ud,cs,tb}(g_d\bar{L}_u\phi R_d+g_u\bar{L}_u\phi_CR_u+h.c.)
\end{align}
����, $\phi_C=i\sigma^2\left[\begin{array}{c}\phi^+\\\phi^0\end{array}\right]=\left[\begin{array}{c}\phi^0\\-\phi^-\end{array}\right]$ Ϊ~$\phi$ �ĵ�ɹ���̬~(����~Gell-Mann-Nishijima ��ʽ���ѷ���~$\phi_C$ ��������~$-1$). ���׷���, ����������, ֮����û�������ڶ���, ����Ϊ�������������غ�. ����, ����ѡ�������淶, ���ÿ�˳Ե�~Goldstone ��ɫ��. ע��~$\phi$ �������淶����ǰ��; ��~$\phi_C$ �Ĺ淶Ϊ~$\phi_C=\frac{1}{\sqrt{2}}\left[\begin{array}{c}v+H\\0\end{array}\right]$. �������淶��, ������
\begin{align}
\mathcal{L}_{qH}=&-\sum_{ud,cs,tb} (g_d\bar{d}_L\phi^0 d_R+g_u\bar{u}_L\phi^0u_R+h.c.)\nonumber\\
=&-\sum_qg_q\bar{q}q\phi^0:=-\sum_q\left(m_q\bar{q}{q}+\frac{m_q}{v}\bar{q}qH\right);
%=-\sum_q\left(\frac{g_qv}{\sqrt{2}}\bar{q}{q}+\frac{g_qH}{\sqrt{2}}\bar{q}q\right)\nonumber\\
\end{align}
���������һ���ǿ�˵�������, �ڶ����ǿ����ϣ��˹�������. �ɴ˼�, ��˻��������
\begin{align}
m_q=\frac{g_qv}{\sqrt{2}}.
\end{align}


��ͬ�������������õ�, ����~$B_\mu,~W^0_\mu$ ���²���ת�����Ӷ��ɵ�~$A_\mu,~Z^0_\mu$ һ��, �����δ��������������̬, �����������̬�����ݲ��뵽���໥������ȥ��. ����ζ���������ù�����, ��˿��ܻ���ζ��~(�����, �Ժ����ǽ�ȷ֤, ʹ��˷���ζ�伴������ϵ�, ������~$W^\pm_\mu$ ���粣ɫ�ӵ��໥����). �������ǿ���
\begin{align}
\left[\begin{array}{c}u_{1L/R}\\u_{2L/R}\\u_{3L/R}\end{array}\right]=U_{L/R}\left[\begin{array}{c}u_{L/R}\\c_{L/R}\\t_{L/R}\end{array}\right],~
\left[\begin{array}{c}d_{1L/R}\\d_{2L/R}\\d_{3L/R}\end{array}\right]=D_{L/R}\left[\begin{array}{c}d_{L/R}\\s_{L/R}\\b_{L/R}\end{array}\right];
\end{align}
����~$U,~D$ ��~$3\times3$ ���Ͼ���. ���Ƿֱ��~$u_i,~d_i$ ��~$u$ ������~$d$ ����. ���ͬʱ, ���乹�ɵ���ͬλ������̬�뵥̬����
\begin{gather}
L_i=\left[\begin{array}{c}u_{iL}\\d_{iL}\end{array}\right],~R_{ui}=u_{iR},~R_{di}=d_{iR}.
\end{gather}
���ѷ���, �任���̬��֮ǰ�ľ���ͬ���ĵ��, ����������ͬλ��. ����ֱ�ӿ���
\begin{gather}
\mathcal{L}_q=\sum_qi\bar{q}\gamma^\mu\partial_\mu q=\sum_{ud,cs,tb}\left(i\bar{L}_u\gamma^\mu\partial_\mu L_u+i\bar{R}_u\gamma^\mu\partial_\mu R_u+i\bar{R}_d\gamma^\mu\partial_\mu R_d\right).
\end{gather}
%ע���һ��, ���Գ�~6 �������������ʽ.
%����~$R_u$ ��ɿ��������ų�һ����ʸ. ��Ȼ, ����٤�����Ҫô��һ���Ǿ��󻯿������ƶ��ķ���, Ҫô�ǵ������µ���ʽ������ά��λ��.
������������������, �����ֿɽ�һ������ʽ��Ϊ
\begin{gather}
\mathcal{L}_q=\sum_i\left(i\bar{L}_i\gamma^\mu\partial_\mu L_i+i\bar{R}_{ui}\gamma^\mu\partial_\mu R_{ui}+i\bar{R}_{di}\gamma^\mu\partial_\mu R_{di}\right);
\end{gather}
�����õ���~$U,~D$ ���Ͼ�����һ����. ��ִ�г���һ������, ��������ǿ�໥���õľ��ǿ�˵���������̬, ǿ�����п����ζ��. ��ʵ��, ��Ȼ, ֻҪ���ҷ���û�����, ���������������~$\sum_ii\bar{R}_{ui}\gamma^\mu\partial_\mu R_{ui}=\sum_{ud,cs,tb}i\bar{R}_u\gamma^\mu\partial_\mu R_u$ ���Ľ��; Ҳ����˵��˾�������������̬����, ����˾���ζ��. �������о��Ŀ�˵ĵ�����ü������~$Z^0_\mu$ ��ɫ�ӵ�����, �Ͷ����������. ��������ϵĻ�, ��Ȼ���ǽ�����һ������, �˼�~CKM~(Cabibbo�CKobayashi�CMaskawa, ���Ȳ�-С��-�洨) ����; �⽫�ڿ����~$W^\pm_\mu$ ������ʱ����. ����, ���Ǿ����ֱ��о�.


��ƫ΢�ָijɵ������õ�Э��΢��~$D_\mu=\partial_\mu+igT^iW_\mu^i+i\frac{1}{2}g'Y_WB_\mu$, ���Ǿ͵õ��˿�˲�������໥���õ�ģ��:
\begin{align}
\mathcal{L}_{q+qWZ}=&i\bar{L}_i\gamma^\mu D_\mu L_i+i\bar{R}_{ui}\gamma^\mu D_\mu R_{ui}+i\bar{R}_{di}\gamma^\mu D_\mu R_{di}\nonumber\\
=&\mathcal{L}_q+\mathcal{L}_{qA}+\mathcal{L}_{qZ^0}+\mathcal{L}_{qW^\pm}.
\end{align}
��һ��ʽ~(\ref{weinberg}) ��Ͽ�˵�̬�����̬��������, ���ѵõ�
\begin{align}
D_\mu R_{ui}&=\left(\partial_\mu+i\frac{2}{3}g'B_\mu\right)R_{ui}\nonumber\\
&=\left(\partial_\mu+i\frac{2}{3}g'(A_\mu\cos\theta_W-Z^0_\mu\sin\theta_W)\right)R_{ui},\\
D_\mu R_{di}&=\left(\partial_\mu-i\frac{1}{3}g'B_\mu\right)R_{di}\nonumber\\
&=\left(\partial_\mu-i\frac{1}{3}g'(A_\mu\cos\theta_W-Z^0_\mu\sin\theta_W)\right)R_{di},\\
D_\mu L_i&=\partial_\mu L_i+\frac{i}{2}\left[\begin{array}{cc}gW_\mu^0+\frac{1}{3}g'B_\mu&\sqrt{2}gW^+_\mu\\\sqrt{2}gW^-_\mu&-gW_\mu^0+\frac{1}{3}g'B_\mu\end{array}\right]L_i;
\end{align}
��Ϊд������, �������ǰ��������һʽ�о����ڶԽ�����²���ת������������:
\begin{align}
&g\sin\theta_W A_\mu+gZ^0_\mu\cos\theta_W+\frac{1}{3}g'A_\mu\cos\theta_W-\frac{1}{3}g'Z^0_\mu\sin\theta_W,\\
-&g\sin\theta_W A_\mu-gZ^0_\mu\cos\theta_W+\frac{1}{3}g'A_\mu\cos\theta_W-\frac{1}{3}g'Z^0_\mu\sin\theta_W.
\end{align}
����, ���ǾͿ�ֱ���ó����໥��������. ���˵ĵ���໥������Ϊ
\begin{align}
\mathcal{L}_{qA}=&-\frac{1}{2}(\bar{u}_{iL},\bar{d}_{iL})\left[\begin{array}{cc}eA_\mu+\frac{1}{3}eA_\mu&0\\0&-eA_\mu+\frac{1}{3}eA_\mu\end{array}\right]
\left[\begin{array}{c}u_{iL}\\d_{iL}\end{array}\right]\nonumber\\
&-\frac{2}{3}e\bar{u}_{iR}\gamma^\mu u_{iR}A_\mu+\frac{1}{3}e\bar{d}_{iR}\gamma^\mu d_{iR}A_\mu\nonumber\\
=&-\frac{2}{3}e\bar{u}_{iL}\gamma^\mu u_{iL}A_\mu+\frac{1}{3}e\bar{d}_{iL}\gamma^\mu d_{iL}A_\mu-\frac{2}{3}e\bar{u}_{iR}\gamma^\mu u_{iR}A_\mu+\frac{1}{3}e\bar{d}_{iR}\gamma^\mu d_{iR}A_\mu\nonumber\\
=&\sum_i\left(-\frac{2}{3}e\bar{u}_{i}\gamma^\mu u_{i}A_\mu+\frac{1}{3}e\bar{d}_{i}\gamma^\mu d_{i}A_\mu\right)\nonumber\\
=&\sum_{ud,cs,tb}\left(-\frac{2}{3}e\bar{u}\gamma^\mu uA_\mu+\frac{1}{3}e\bar{d}\gamma^\mu dA_\mu\right)=\sum_q e\bar{q}\gamma^\mu Q qA_\mu.
\end{align}
����~$Q$ Ϊ��˵ĵ�����. ������~QED �е�һ����ʽ. ���濼������~$Z^0_\mu$ ��ɫ�ӵ����.%�����ӵ�������ʽ�������ڶ�ʽ, �Կ�˻��̬����ͻ�Ϊ�Կ��ζ�����; ��˫������ʽ�����������໥���õ�, �Կɹ鵽��˵���������̬.
\begin{align}
\mathcal{L}_{qZ^0}=&\left(\frac{2}{3}\bar{u}_{iR}\gamma^\mu u_{iR}-\frac{1}{3}\bar{d}_{iR}\gamma^\mu d_{iR}\right)g'\sin\theta_WZ^0_\mu\nonumber\\
&-\left(\frac{1}{2}g\cos\theta_W-\frac{1}{6}g'\sin\theta_W\right)\bar{u}_{iL}\gamma^\mu u_{iL}Z^0_\mu\nonumber\\
&-\left(-\frac{1}{2}g\cos\theta_W-\frac{1}{6}g'\sin\theta_W\right)\bar{d}_{iL}\gamma^\mu d_{iL}Z^0_\mu\nonumber\\
=&\bar{q}_{R}\gamma^\mu Qq_{R}g'\sin\theta_WZ^0_\mu\nonumber\\
&-\frac{g}{\cos\theta_W}\left(\frac{1}{2}-\frac{2}{3}\sin^2\theta_W\right)\bar{u}_{L}\gamma^\mu u_{L}Z^0_\mu\nonumber\\
&-\frac{g}{\cos\theta_W}\left(-\frac{1}{2}+\frac{1}{3}\sin\theta^2_W\right)\bar{d}_{L}\gamma^\mu d_{L}Z^0_\mu\nonumber\\
=&-\frac{g}{\cos\theta_W} \left[\bar{q}_{L}\gamma^\mu\left(T^3-Q\sin\theta^2_W\right) q_{L}-\bar{q}_{R}\gamma^\mu Q\sin^2\theta_W q_{R} \right] Z^0_\mu\nonumber\\
=&-\frac{g}{\cos\theta_W} \left[\bar{q}_{L}\gamma^\mu T^3q_{L}-\bar{q}\gamma^\mu Q\sin^2\theta_W q \right] Z^0_\mu
\end{align}
����ʽ������-$Z^0_\mu$ �����ʽ�����Ƶ�; ���������ǿ����ͬλ����������. �����~$W^\pm_\mu$ ������Ǻܺ��ó���:
\begin{align}
\mathcal{L}_{qW^\pm}=&-\frac{g}{\sqrt{2}}\left(\bar{u}_{iL}\gamma^\mu d_{iL}W^+_\mu+\bar{d}_{iL}\gamma^\mu u_{iL}W^-_\mu\right)\nonumber\\
=&-\frac{g}{\sqrt{2}}\left(\bar{u}_{L}\gamma^\mu V_{ud}d_{L}W^+_\mu+\bar{d}_{L}\gamma^\mu V^*_{du} u_{L}W^-_\mu\right);
\end{align}
������ά�Ͼ���
\begin{align}
V=U_L^\dag D_L
\end{align}
��Ϊ~CKM ����; ��������˲�ͬζ��Ļ��, ����Ҫ��ʵ�����ⶨ��. �����²�ֵ�ɼ��������; ���Ƶ�, ��ԼΪ
\begin{align}
V=\left[\begin{array}{ccc}
\cos\theta_c&\sin\theta_c&0\\
-\sin\theta_c&\cos\theta_c&0\\
0&0&1
\end{array}
\right];
\end{align}
����~$\theta_c$ ��Ϊ~Cabibbo ��. Ҳ����˵~CKM ����������һ����ά�ռ������ת��.





%CKM ����: �.



\subsection{���������ı�׼ģ��}


~~~~�ܽ�����, ��������ͳһ��~Glashow-Weinberg-Salam ���۵������ܶȾ���
\begin{align}
\mathcal{L}_{\textrm{GWS}}=&\mathcal{L}_{\textrm{gauge field}}+\mathcal{L}_{\textrm{lepton+G~bosons}}+\mathcal{L}_{\textrm{quark+G~Bosons}}\nonumber\\
&+\mathcal{L}_{\textrm{Higgs+$\phi^4$}}+\mathcal{L}_{\textrm{lepton+Higgs}}+\mathcal{L}_{\textrm{quark+Higgs}}\nonumber\\
=&-\frac{1}{4}B_{\mu\nu}B^{\mu\nu}-\frac{1}{4}W^i_{\mu\nu}W_i^{\mu\nu}\nonumber\\
&+\left(i\bar{L}_e\gamma^\mu D_\mu L_e+i\bar{R}_e\gamma^\mu D_\mu R_e\right)\nonumber\\
&+\left(i\bar{L}_i\gamma^\mu\partial_\mu L_i+i\bar{R}_{ui}\gamma^\mu\partial_\mu R_{ui}+i\bar{R}_{di}\gamma^\mu\partial_\mu R_{di}\right)\nonumber\\
&+D_\mu\phi^\dag D^\mu\phi-m^2\phi^\dag\phi-\lambda(\phi^\dag\phi)^2\nonumber\\
&-g_e(\bar{L}_e\phi R_e+\bar{R}_e\phi^\dag L_e)\nonumber\\
&-(g_d\bar{L}_u\phi R_d+g_u\bar{L}_u\phi_CR_u+h.c.);
\end{align}
��~$D_\mu=\partial_\mu+igT^iW_\mu^i+i\frac{1}{2}g'Y_WB_\mu$, ����������, ��Ի~SU(2)$\otimes$U(1) �Գ����µ�Э�䵼��. ���ټ��Ͻ��ӳ���������~$\mathcal{L}_{\textrm{gluon}}=-\frac{1}{4}F^a_{\mu\nu}F_a^{\mu\nu}$, ����~SU(3) �µ�Э�䵼���ӽ���, ��
\begin{align}
D_\mu=\partial_\mu+igT^iW_\mu^i+i\frac{1}{2}g'Y_WB_\mu+ig_3T^aA^\mu_a,
\end{align}
�����Ǿ͵õ���ͬʱ����ǿ, ����������������������������. ������Э�䵼�����������ܶȼ������̺���һϵ��˼��ԭ�������㷽��, ��Ϊ���������ı�׼ģ��.











%������ɵ�ɳе�, ����, ����֪�ɵ����е�; ͬ��, ����, �����������е�
%����ڵ��, ��������, ����ͬλ��.!!!!!!!!��ν����, ����ͬλ����.
%��Ϊ��һ, ���㷴����, ����ֻ��һ��, �������ܸ��ܵ���. ��ô��������, ��Ϊ�϶�, �����ܸ��ܵ�������������, һ��������. �������м䲣ɫ����һһ��ϵ�.
%ͬλ��, ����������ʸ��Ϊ��, ʹ��΢�������Ӳ������. �������ܸ��ܵ��Ļ�, �ǽ��Ǻ������.

%���˵�, ���Ǵ��˵���, �Ǵ��˵�ų�, �����ӹ淶��!!!!!!!!!!������!!!!!! ����ͻȻ������淶�й��ɵĵ���໥����, ʹ����������. ����������, ���Ǵ���
%�������м䲣ɫ��, �Ӷ�����������Ĵ�����������΢�ӷ�������˶�.
%ͬ��, ����Ҳ���Ƿ�����, ���Dz�����ų�, ������. ͨ���ļ��û�ҵ�õ�, ���Ǵ��ڵ�����ĵ�ų�/���ӹ淶����.

%ע���е�ʱ��, \gamma ������ʵ���ǰ�ά��. ����, ���һ��, ������ֱ�Ӱ�������һ����, ȴ�ֱ�����ά������.

%ͷ�ϼ�һ��, ���ǵ���������, �����ĸ�����.

\newpage



\end{CJK*}
\end{document}
